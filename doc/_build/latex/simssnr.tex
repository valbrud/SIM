%% Generated by Sphinx.
\def\sphinxdocclass{report}
\documentclass[letterpaper,10pt,english]{sphinxmanual}
\ifdefined\pdfpxdimen
   \let\sphinxpxdimen\pdfpxdimen\else\newdimen\sphinxpxdimen
\fi \sphinxpxdimen=.75bp\relax
\ifdefined\pdfimageresolution
    \pdfimageresolution= \numexpr \dimexpr1in\relax/\sphinxpxdimen\relax
\fi
%% let collapsible pdf bookmarks panel have high depth per default
\PassOptionsToPackage{bookmarksdepth=5}{hyperref}

\PassOptionsToPackage{booktabs}{sphinx}
\PassOptionsToPackage{colorrows}{sphinx}

\PassOptionsToPackage{warn}{textcomp}
\usepackage[utf8]{inputenc}
\ifdefined\DeclareUnicodeCharacter
% support both utf8 and utf8x syntaxes
  \ifdefined\DeclareUnicodeCharacterAsOptional
    \def\sphinxDUC#1{\DeclareUnicodeCharacter{"#1}}
  \else
    \let\sphinxDUC\DeclareUnicodeCharacter
  \fi
  \sphinxDUC{00A0}{\nobreakspace}
  \sphinxDUC{2500}{\sphinxunichar{2500}}
  \sphinxDUC{2502}{\sphinxunichar{2502}}
  \sphinxDUC{2514}{\sphinxunichar{2514}}
  \sphinxDUC{251C}{\sphinxunichar{251C}}
  \sphinxDUC{2572}{\textbackslash}
\fi
\usepackage{cmap}
\usepackage[T1]{fontenc}
\usepackage{amsmath,amssymb,amstext}
\usepackage{babel}



\usepackage{tgtermes}
\usepackage{tgheros}
\renewcommand{\ttdefault}{txtt}



\usepackage[Bjarne]{fncychap}
\usepackage{sphinx}

\fvset{fontsize=auto}
\usepackage{geometry}


% Include hyperref last.
\usepackage{hyperref}
% Fix anchor placement for figures with captions.
\usepackage{hypcap}% it must be loaded after hyperref.
% Set up styles of URL: it should be placed after hyperref.
\urlstyle{same}

\addto\captionsenglish{\renewcommand{\contentsname}{Contents:}}

\usepackage{sphinxmessages}
\setcounter{tocdepth}{1}



\title{SIMSSNR}
\date{Dec 11, 2024}
\release{1.0.0}
\author{Valerii Brudanin (TU Delft)}
\newcommand{\sphinxlogo}{\vbox{}}
\renewcommand{\releasename}{Release}
\makeindex
\begin{document}

\ifdefined\shorthandoff
  \ifnum\catcode`\=\string=\active\shorthandoff{=}\fi
  \ifnum\catcode`\"=\active\shorthandoff{"}\fi
\fi

\pagestyle{empty}
\sphinxmaketitle
\pagestyle{plain}
\sphinxtableofcontents
\pagestyle{normal}
\phantomsection\label{\detokenize{index::doc}}



\chapter{Welcome to SIM Documentation}
\label{\detokenize{index:welcome-to-sim-documentation}}
\sphinxstepscope


\section{SIM}
\label{\detokenize{source/modules:sim}}\label{\detokenize{source/modules::doc}}
\sphinxstepscope


\subsection{Box module}
\label{\detokenize{source/Box:module-Box}}\label{\detokenize{source/Box:box-module}}\label{\detokenize{source/Box::doc}}\index{module@\spxentry{module}!Box@\spxentry{Box}}\index{Box@\spxentry{Box}!module@\spxentry{module}}
\sphinxAtStartPar
Box.py

\sphinxAtStartPar
This module contains classes for handling simulation volume and containing fields.

\sphinxincludegraphics[]{inheritance-23ed37e289f5cb6ce570857199b0f889790b1a21.pdf}
\index{Box (class in Box)@\spxentry{Box}\spxextra{class in Box}}

\begin{fulllineitems}
\phantomsection\label{\detokenize{source/Box:Box.Box}}
\pysigstartsignatures
\pysiglinewithargsret
{\sphinxbfcode{\sphinxupquote{class\DUrole{w}{ }}}\sphinxcode{\sphinxupquote{Box.}}\sphinxbfcode{\sphinxupquote{Box}}}
{\sphinxparam{\DUrole{n}{sources}\DUrole{o}{=}\DUrole{default_value}{()}}\sphinxparamcomma \sphinxparam{\DUrole{n}{box\_size}\DUrole{o}{=}\DUrole{default_value}{10}}\sphinxparamcomma \sphinxparam{\DUrole{n}{point\_number}\DUrole{o}{=}\DUrole{default_value}{100}}\sphinxparamcomma \sphinxparam{\DUrole{n}{additional\_info}\DUrole{o}{=}\DUrole{default_value}{None}}}
{}
\pysigstopsignatures
\sphinxAtStartPar
Bases: \sphinxcode{\sphinxupquote{object}}

\sphinxAtStartPar
This class represents a simulation volume where fields and intensities are computed.
\index{info (Box.Box attribute)@\spxentry{info}\spxextra{Box.Box attribute}}

\begin{fulllineitems}
\phantomsection\label{\detokenize{source/Box:Box.Box.info}}
\pysigstartsignatures
\pysigline
{\sphinxbfcode{\sphinxupquote{info}}}
\pysigstopsignatures
\sphinxAtStartPar
Additional information about the box.
\begin{quote}\begin{description}
\sphinxlineitem{Type}
\sphinxAtStartPar
dict

\end{description}\end{quote}

\end{fulllineitems}

\index{box\_size (Box.Box attribute)@\spxentry{box\_size}\spxextra{Box.Box attribute}}

\begin{fulllineitems}
\phantomsection\label{\detokenize{source/Box:Box.Box.box_size}}
\pysigstartsignatures
\pysigline
{\sphinxbfcode{\sphinxupquote{box\_size}}}
\pysigstopsignatures
\sphinxAtStartPar
Size of the box in each dimension.
\begin{quote}\begin{description}
\sphinxlineitem{Type}
\sphinxAtStartPar
np.ndarray

\end{description}\end{quote}

\end{fulllineitems}

\index{point\_number (Box.Box attribute)@\spxentry{point\_number}\spxextra{Box.Box attribute}}

\begin{fulllineitems}
\phantomsection\label{\detokenize{source/Box:Box.Box.point_number}}
\pysigstartsignatures
\pysigline
{\sphinxbfcode{\sphinxupquote{point\_number}}}
\pysigstopsignatures
\sphinxAtStartPar
Number of points in each dimension.
\begin{quote}\begin{description}
\sphinxlineitem{Type}
\sphinxAtStartPar
np.ndarray

\end{description}\end{quote}

\end{fulllineitems}

\index{box\_volume (Box.Box attribute)@\spxentry{box\_volume}\spxextra{Box.Box attribute}}

\begin{fulllineitems}
\phantomsection\label{\detokenize{source/Box:Box.Box.box_volume}}
\pysigstartsignatures
\pysigline
{\sphinxbfcode{\sphinxupquote{box\_volume}}}
\pysigstopsignatures
\sphinxAtStartPar
Volume of the box.
\begin{quote}\begin{description}
\sphinxlineitem{Type}
\sphinxAtStartPar
float

\end{description}\end{quote}

\end{fulllineitems}

\index{fields (Box.Box attribute)@\spxentry{fields}\spxextra{Box.Box attribute}}

\begin{fulllineitems}
\phantomsection\label{\detokenize{source/Box:Box.Box.fields}}
\pysigstartsignatures
\pysigline
{\sphinxbfcode{\sphinxupquote{fields}}}
\pysigstopsignatures
\sphinxAtStartPar
List of fields in the box.
\begin{quote}\begin{description}
\sphinxlineitem{Type}
\sphinxAtStartPar
list

\end{description}\end{quote}

\end{fulllineitems}

\index{numerically\_approximated\_intensity\_fields (Box.Box attribute)@\spxentry{numerically\_approximated\_intensity\_fields}\spxextra{Box.Box attribute}}

\begin{fulllineitems}
\phantomsection\label{\detokenize{source/Box:Box.Box.numerically_approximated_intensity_fields}}
\pysigstartsignatures
\pysigline
{\sphinxbfcode{\sphinxupquote{numerically\_approximated\_intensity\_fields}}}
\pysigstopsignatures
\sphinxAtStartPar
List of numerically approximated intensity fields.
\begin{quote}\begin{description}
\sphinxlineitem{Type}
\sphinxAtStartPar
list

\end{description}\end{quote}

\end{fulllineitems}

\index{source\_identifier (Box.Box attribute)@\spxentry{source\_identifier}\spxextra{Box.Box attribute}}

\begin{fulllineitems}
\phantomsection\label{\detokenize{source/Box:Box.Box.source_identifier}}
\pysigstartsignatures
\pysigline
{\sphinxbfcode{\sphinxupquote{source\_identifier}}}
\pysigstopsignatures
\sphinxAtStartPar
Identifier for the sources.
\begin{quote}\begin{description}
\sphinxlineitem{Type}
\sphinxAtStartPar
int

\end{description}\end{quote}

\end{fulllineitems}

\index{axes (Box.Box attribute)@\spxentry{axes}\spxextra{Box.Box attribute}}

\begin{fulllineitems}
\phantomsection\label{\detokenize{source/Box:Box.Box.axes}}
\pysigstartsignatures
\pysigline
{\sphinxbfcode{\sphinxupquote{axes}}}
\pysigstopsignatures
\sphinxAtStartPar
Axes for the box.
\begin{quote}\begin{description}
\sphinxlineitem{Type}
\sphinxAtStartPar
tuple

\end{description}\end{quote}

\end{fulllineitems}

\index{frequency\_axes (Box.Box attribute)@\spxentry{frequency\_axes}\spxextra{Box.Box attribute}}

\begin{fulllineitems}
\phantomsection\label{\detokenize{source/Box:Box.Box.frequency_axes}}
\pysigstartsignatures
\pysigline
{\sphinxbfcode{\sphinxupquote{frequency\_axes}}}
\pysigstopsignatures
\sphinxAtStartPar
Frequency axes for the box.
\begin{quote}\begin{description}
\sphinxlineitem{Type}
\sphinxAtStartPar
tuple

\end{description}\end{quote}

\end{fulllineitems}

\index{grid (Box.Box attribute)@\spxentry{grid}\spxextra{Box.Box attribute}}

\begin{fulllineitems}
\phantomsection\label{\detokenize{source/Box:Box.Box.grid}}
\pysigstartsignatures
\pysigline
{\sphinxbfcode{\sphinxupquote{grid}}}
\pysigstopsignatures
\sphinxAtStartPar
Grid of points in the box.
\begin{quote}\begin{description}
\sphinxlineitem{Type}
\sphinxAtStartPar
np.ndarray

\end{description}\end{quote}

\end{fulllineitems}

\index{electric\_field (Box.Box attribute)@\spxentry{electric\_field}\spxextra{Box.Box attribute}}

\begin{fulllineitems}
\phantomsection\label{\detokenize{source/Box:Box.Box.electric_field}}
\pysigstartsignatures
\pysigline
{\sphinxbfcode{\sphinxupquote{electric\_field}}}
\pysigstopsignatures
\sphinxAtStartPar
Electric field in the box.
\begin{quote}\begin{description}
\sphinxlineitem{Type}
\sphinxAtStartPar
np.ndarray

\end{description}\end{quote}

\end{fulllineitems}

\index{intensity (Box.Box attribute)@\spxentry{intensity}\spxextra{Box.Box attribute}}

\begin{fulllineitems}
\phantomsection\label{\detokenize{source/Box:Box.Box.intensity}}
\pysigstartsignatures
\pysigline
{\sphinxbfcode{\sphinxupquote{intensity}}}
\pysigstopsignatures
\sphinxAtStartPar
Intensity in the box.
\begin{quote}\begin{description}
\sphinxlineitem{Type}
\sphinxAtStartPar
np.ndarray

\end{description}\end{quote}

\end{fulllineitems}

\index{numerically\_approximated\_intensity (Box.Box attribute)@\spxentry{numerically\_approximated\_intensity}\spxextra{Box.Box attribute}}

\begin{fulllineitems}
\phantomsection\label{\detokenize{source/Box:Box.Box.numerically_approximated_intensity}}
\pysigstartsignatures
\pysigline
{\sphinxbfcode{\sphinxupquote{numerically\_approximated\_intensity}}}
\pysigstopsignatures
\sphinxAtStartPar
Numerically approximated intensity in the box.
\begin{quote}\begin{description}
\sphinxlineitem{Type}
\sphinxAtStartPar
np.ndarray

\end{description}\end{quote}

\end{fulllineitems}

\index{intensity\_fourier\_space (Box.Box attribute)@\spxentry{intensity\_fourier\_space}\spxextra{Box.Box attribute}}

\begin{fulllineitems}
\phantomsection\label{\detokenize{source/Box:Box.Box.intensity_fourier_space}}
\pysigstartsignatures
\pysigline
{\sphinxbfcode{\sphinxupquote{intensity\_fourier\_space}}}
\pysigstopsignatures
\sphinxAtStartPar
Intensity in the Fourier space.
\begin{quote}\begin{description}
\sphinxlineitem{Type}
\sphinxAtStartPar
np.ndarray

\end{description}\end{quote}

\end{fulllineitems}

\index{numerically\_approximated\_intensity\_fourier\_space (Box.Box attribute)@\spxentry{numerically\_approximated\_intensity\_fourier\_space}\spxextra{Box.Box attribute}}

\begin{fulllineitems}
\phantomsection\label{\detokenize{source/Box:Box.Box.numerically_approximated_intensity_fourier_space}}
\pysigstartsignatures
\pysigline
{\sphinxbfcode{\sphinxupquote{numerically\_approximated\_intensity\_fourier\_space}}}
\pysigstopsignatures
\sphinxAtStartPar
Numerically approximated intensity in the Fourier space.
\begin{quote}\begin{description}
\sphinxlineitem{Type}
\sphinxAtStartPar
np.ndarray

\end{description}\end{quote}

\end{fulllineitems}

\index{analytic\_frequencies (Box.Box attribute)@\spxentry{analytic\_frequencies}\spxextra{Box.Box attribute}}

\begin{fulllineitems}
\phantomsection\label{\detokenize{source/Box:Box.Box.analytic_frequencies}}
\pysigstartsignatures
\pysigline
{\sphinxbfcode{\sphinxupquote{analytic\_frequencies}}}
\pysigstopsignatures
\sphinxAtStartPar
List of analytic frequencies.
\begin{quote}\begin{description}
\sphinxlineitem{Type}
\sphinxAtStartPar
list

\end{description}\end{quote}

\end{fulllineitems}


\sphinxincludegraphics[]{inheritance-aecaf71520ad0d9dc066ae9c8e29505ff73505a1.pdf}
\index{add\_source() (Box.Box method)@\spxentry{add\_source()}\spxextra{Box.Box method}}

\begin{fulllineitems}
\phantomsection\label{\detokenize{source/Box:Box.Box.add_source}}
\pysigstartsignatures
\pysiglinewithargsret
{\sphinxbfcode{\sphinxupquote{add\_source}}}
{\sphinxparam{\DUrole{n}{source}}}
{}
\pysigstopsignatures
\sphinxAtStartPar
Adds a source to the box. The corresponding field is added automatically.
\begin{quote}\begin{description}
\sphinxlineitem{Parameters}
\sphinxAtStartPar
\sphinxstyleliteralstrong{\sphinxupquote{source}} \textendash{} Source to add.

\end{description}\end{quote}

\end{fulllineitems}

\index{compute\_axes() (Box.Box method)@\spxentry{compute\_axes()}\spxextra{Box.Box method}}

\begin{fulllineitems}
\phantomsection\label{\detokenize{source/Box:Box.Box.compute_axes}}
\pysigstartsignatures
\pysiglinewithargsret
{\sphinxbfcode{\sphinxupquote{compute\_axes}}}
{}
{}
\pysigstopsignatures
\sphinxAtStartPar
Computes the axes and frequency axes for the box.
\begin{quote}\begin{description}
\sphinxlineitem{Returns}
\sphinxAtStartPar
Axes and frequency axes for the box.

\sphinxlineitem{Return type}
\sphinxAtStartPar
tuple

\end{description}\end{quote}

\end{fulllineitems}

\index{compute\_electric\_field() (Box.Box method)@\spxentry{compute\_electric\_field()}\spxextra{Box.Box method}}

\begin{fulllineitems}
\phantomsection\label{\detokenize{source/Box:Box.Box.compute_electric_field}}
\pysigstartsignatures
\pysiglinewithargsret
{\sphinxbfcode{\sphinxupquote{compute\_electric\_field}}}
{}
{}
\pysigstopsignatures
\sphinxAtStartPar
Computes the electric field in the box.

\end{fulllineitems}

\index{compute\_grid() (Box.Box method)@\spxentry{compute\_grid()}\spxextra{Box.Box method}}

\begin{fulllineitems}
\phantomsection\label{\detokenize{source/Box:Box.Box.compute_grid}}
\pysigstartsignatures
\pysiglinewithargsret
{\sphinxbfcode{\sphinxupquote{compute\_grid}}}
{}
{}
\pysigstopsignatures
\sphinxAtStartPar
Computes the grid of points in the box.

\end{fulllineitems}

\index{compute\_intensity\_and\_spatial\_waves\_numerically() (Box.Box method)@\spxentry{compute\_intensity\_and\_spatial\_waves\_numerically()}\spxextra{Box.Box method}}

\begin{fulllineitems}
\phantomsection\label{\detokenize{source/Box:Box.Box.compute_intensity_and_spatial_waves_numerically}}
\pysigstartsignatures
\pysiglinewithargsret
{\sphinxbfcode{\sphinxupquote{compute\_intensity\_and\_spatial\_waves\_numerically}}}
{}
{}
\pysigstopsignatures
\sphinxAtStartPar
Find approximately spatial waves from intensity in Fourier space and compute from them the approximated intensity in the box.

\end{fulllineitems}

\index{compute\_intensity\_fourier\_space() (Box.Box method)@\spxentry{compute\_intensity\_fourier\_space()}\spxextra{Box.Box method}}

\begin{fulllineitems}
\phantomsection\label{\detokenize{source/Box:Box.Box.compute_intensity_fourier_space}}
\pysigstartsignatures
\pysiglinewithargsret
{\sphinxbfcode{\sphinxupquote{compute\_intensity\_fourier\_space}}}
{}
{}
\pysigstopsignatures
\end{fulllineitems}

\index{compute\_intensity\_from\_electric\_field() (Box.Box method)@\spxentry{compute\_intensity\_from\_electric\_field()}\spxextra{Box.Box method}}

\begin{fulllineitems}
\phantomsection\label{\detokenize{source/Box:Box.Box.compute_intensity_from_electric_field}}
\pysigstartsignatures
\pysiglinewithargsret
{\sphinxbfcode{\sphinxupquote{compute\_intensity\_from\_electric\_field}}}
{}
{}
\pysigstopsignatures
\sphinxAtStartPar
Computes the intensity from the electric field.

\end{fulllineitems}

\index{compute\_intensity\_from\_spatial\_waves() (Box.Box method)@\spxentry{compute\_intensity\_from\_spatial\_waves()}\spxextra{Box.Box method}}

\begin{fulllineitems}
\phantomsection\label{\detokenize{source/Box:Box.Box.compute_intensity_from_spatial_waves}}
\pysigstartsignatures
\pysiglinewithargsret
{\sphinxbfcode{\sphinxupquote{compute\_intensity\_from\_spatial\_waves}}}
{}
{}
\pysigstopsignatures
\sphinxAtStartPar
Computes the intensity from intensity spatial waves.

\end{fulllineitems}

\index{get\_approximated\_intensity\_sources() (Box.Box method)@\spxentry{get\_approximated\_intensity\_sources()}\spxextra{Box.Box method}}

\begin{fulllineitems}
\phantomsection\label{\detokenize{source/Box:Box.Box.get_approximated_intensity_sources}}
\pysigstartsignatures
\pysiglinewithargsret
{\sphinxbfcode{\sphinxupquote{get\_approximated\_intensity\_sources}}}
{}
{}
\pysigstopsignatures
\sphinxAtStartPar
Returns a list of numerically estimated intensity sources in the box.
\begin{quote}\begin{description}
\sphinxlineitem{Returns}
\sphinxAtStartPar
List of approximated intensity sources.

\sphinxlineitem{Return type}
\sphinxAtStartPar
list

\end{description}\end{quote}

\end{fulllineitems}

\index{get\_plane\_waves() (Box.Box method)@\spxentry{get\_plane\_waves()}\spxextra{Box.Box method}}

\begin{fulllineitems}
\phantomsection\label{\detokenize{source/Box:Box.Box.get_plane_waves}}
\pysigstartsignatures
\pysiglinewithargsret
{\sphinxbfcode{\sphinxupquote{get\_plane\_waves}}}
{}
{}
\pysigstopsignatures
\sphinxAtStartPar
Returns a list of plane waves in the box.
\begin{quote}\begin{description}
\sphinxlineitem{Returns}
\sphinxAtStartPar
List of plane waves.

\sphinxlineitem{Return type}
\sphinxAtStartPar
list

\end{description}\end{quote}

\end{fulllineitems}

\index{get\_sources() (Box.Box method)@\spxentry{get\_sources()}\spxextra{Box.Box method}}

\begin{fulllineitems}
\phantomsection\label{\detokenize{source/Box:Box.Box.get_sources}}
\pysigstartsignatures
\pysiglinewithargsret
{\sphinxbfcode{\sphinxupquote{get\_sources}}}
{}
{}
\pysigstopsignatures
\sphinxAtStartPar
Returns a list of sources in the box.
\begin{quote}\begin{description}
\sphinxlineitem{Returns}
\sphinxAtStartPar
List of sources.

\sphinxlineitem{Return type}
\sphinxAtStartPar
list

\end{description}\end{quote}

\end{fulllineitems}

\index{get\_spatial\_waves() (Box.Box method)@\spxentry{get\_spatial\_waves()}\spxextra{Box.Box method}}

\begin{fulllineitems}
\phantomsection\label{\detokenize{source/Box:Box.Box.get_spatial_waves}}
\pysigstartsignatures
\pysiglinewithargsret
{\sphinxbfcode{\sphinxupquote{get\_spatial\_waves}}}
{}
{}
\pysigstopsignatures
\sphinxAtStartPar
Returns a list of spatial waves in the box.
\begin{quote}\begin{description}
\sphinxlineitem{Returns}
\sphinxAtStartPar
List of spatial waves.

\sphinxlineitem{Return type}
\sphinxAtStartPar
list

\end{description}\end{quote}

\end{fulllineitems}

\index{plot\_approximate\_intensity\_fourier\_space\_slices() (Box.Box method)@\spxentry{plot\_approximate\_intensity\_fourier\_space\_slices()}\spxextra{Box.Box method}}

\begin{fulllineitems}
\phantomsection\label{\detokenize{source/Box:Box.Box.plot_approximate_intensity_fourier_space_slices}}
\pysigstartsignatures
\pysiglinewithargsret
{\sphinxbfcode{\sphinxupquote{plot\_approximate\_intensity\_fourier\_space\_slices}}}
{\sphinxparam{\DUrole{n}{ax}\DUrole{o}{=}\DUrole{default_value}{None}}\sphinxparamcomma \sphinxparam{\DUrole{n}{slider}\DUrole{o}{=}\DUrole{default_value}{None}}}
{}
\pysigstopsignatures
\sphinxAtStartPar
Plots slices of the intensity in the Fourier space, computed from spatial waves found numerically .
\begin{quote}\begin{description}
\sphinxlineitem{Parameters}\begin{itemize}
\item {} 
\sphinxAtStartPar
\sphinxstyleliteralstrong{\sphinxupquote{ax}} (\sphinxstyleliteralemphasis{\sphinxupquote{matplotlib.axes.Axes}}\sphinxstyleliteralemphasis{\sphinxupquote{, }}\sphinxstyleliteralemphasis{\sphinxupquote{optional}}) \textendash{} Axes to plot on. Defaults to None.

\item {} 
\sphinxAtStartPar
\sphinxstyleliteralstrong{\sphinxupquote{slider}} (\sphinxstyleliteralemphasis{\sphinxupquote{matplotlib.widgets.Slider}}\sphinxstyleliteralemphasis{\sphinxupquote{, }}\sphinxstyleliteralemphasis{\sphinxupquote{optional}}) \textendash{} Slider for interactive plotting. Defaults to None.

\end{itemize}

\end{description}\end{quote}

\end{fulllineitems}

\index{plot\_approximate\_intensity\_slices() (Box.Box method)@\spxentry{plot\_approximate\_intensity\_slices()}\spxextra{Box.Box method}}

\begin{fulllineitems}
\phantomsection\label{\detokenize{source/Box:Box.Box.plot_approximate_intensity_slices}}
\pysigstartsignatures
\pysiglinewithargsret
{\sphinxbfcode{\sphinxupquote{plot\_approximate\_intensity\_slices}}}
{\sphinxparam{\DUrole{n}{ax}\DUrole{o}{=}\DUrole{default_value}{None}}\sphinxparamcomma \sphinxparam{\DUrole{n}{slider}\DUrole{o}{=}\DUrole{default_value}{None}}}
{}
\pysigstopsignatures
\sphinxAtStartPar
Plots slices of the intensity in the real space, computed from spatial waves found numerically.
\begin{quote}\begin{description}
\sphinxlineitem{Parameters}\begin{itemize}
\item {} 
\sphinxAtStartPar
\sphinxstyleliteralstrong{\sphinxupquote{ax}} (\sphinxstyleliteralemphasis{\sphinxupquote{matplotlib.axes.Axes}}\sphinxstyleliteralemphasis{\sphinxupquote{, }}\sphinxstyleliteralemphasis{\sphinxupquote{optional}}) \textendash{} Axes to plot on. Defaults to None.

\item {} 
\sphinxAtStartPar
\sphinxstyleliteralstrong{\sphinxupquote{slider}} (\sphinxstyleliteralemphasis{\sphinxupquote{matplotlib.widgets.Slider}}\sphinxstyleliteralemphasis{\sphinxupquote{, }}\sphinxstyleliteralemphasis{\sphinxupquote{optional}}) \textendash{} Slider for interactive plotting. Defaults to None.

\end{itemize}

\end{description}\end{quote}

\end{fulllineitems}

\index{plot\_intensity\_fourier\_space\_slices() (Box.Box method)@\spxentry{plot\_intensity\_fourier\_space\_slices()}\spxextra{Box.Box method}}

\begin{fulllineitems}
\phantomsection\label{\detokenize{source/Box:Box.Box.plot_intensity_fourier_space_slices}}
\pysigstartsignatures
\pysiglinewithargsret
{\sphinxbfcode{\sphinxupquote{plot\_intensity\_fourier\_space\_slices}}}
{\sphinxparam{\DUrole{n}{ax}\DUrole{o}{=}\DUrole{default_value}{None}}\sphinxparamcomma \sphinxparam{\DUrole{n}{slider}\DUrole{o}{=}\DUrole{default_value}{None}}}
{}
\pysigstopsignatures
\sphinxAtStartPar
Plots slices of the intensity in the Fourier space.
\begin{quote}\begin{description}
\sphinxlineitem{Parameters}\begin{itemize}
\item {} 
\sphinxAtStartPar
\sphinxstyleliteralstrong{\sphinxupquote{ax}} (\sphinxstyleliteralemphasis{\sphinxupquote{matplotlib.axes.Axes}}\sphinxstyleliteralemphasis{\sphinxupquote{, }}\sphinxstyleliteralemphasis{\sphinxupquote{optional}}) \textendash{} Axes to plot on. Defaults to None.

\item {} 
\sphinxAtStartPar
\sphinxstyleliteralstrong{\sphinxupquote{slider}} (\sphinxstyleliteralemphasis{\sphinxupquote{matplotlib.widgets.Slider}}\sphinxstyleliteralemphasis{\sphinxupquote{, }}\sphinxstyleliteralemphasis{\sphinxupquote{optional}}) \textendash{} Slider for interactive plotting. Defaults to None.

\end{itemize}

\end{description}\end{quote}

\end{fulllineitems}

\index{plot\_intensity\_slices() (Box.Box method)@\spxentry{plot\_intensity\_slices()}\spxextra{Box.Box method}}

\begin{fulllineitems}
\phantomsection\label{\detokenize{source/Box:Box.Box.plot_intensity_slices}}
\pysigstartsignatures
\pysiglinewithargsret
{\sphinxbfcode{\sphinxupquote{plot\_intensity\_slices}}}
{\sphinxparam{\DUrole{n}{ax}\DUrole{o}{=}\DUrole{default_value}{None}}\sphinxparamcomma \sphinxparam{\DUrole{n}{slider}\DUrole{o}{=}\DUrole{default_value}{None}}}
{}
\pysigstopsignatures
\sphinxAtStartPar
Plots slices of the intensity in the real space.
\begin{quote}\begin{description}
\sphinxlineitem{Parameters}\begin{itemize}
\item {} 
\sphinxAtStartPar
\sphinxstyleliteralstrong{\sphinxupquote{ax}} (\sphinxstyleliteralemphasis{\sphinxupquote{matplotlib.axes.Axes}}\sphinxstyleliteralemphasis{\sphinxupquote{, }}\sphinxstyleliteralemphasis{\sphinxupquote{optional}}) \textendash{} Axes to plot on. Defaults to None.

\item {} 
\sphinxAtStartPar
\sphinxstyleliteralstrong{\sphinxupquote{slider}} (\sphinxstyleliteralemphasis{\sphinxupquote{matplotlib.widgets.Slider}}\sphinxstyleliteralemphasis{\sphinxupquote{, }}\sphinxstyleliteralemphasis{\sphinxupquote{optional}}) \textendash{} Slider for interactive plotting. Defaults to None.

\end{itemize}

\end{description}\end{quote}

\end{fulllineitems}

\index{plot\_slices() (Box.Box method)@\spxentry{plot\_slices()}\spxextra{Box.Box method}}

\begin{fulllineitems}
\phantomsection\label{\detokenize{source/Box:Box.Box.plot_slices}}
\pysigstartsignatures
\pysiglinewithargsret
{\sphinxbfcode{\sphinxupquote{plot\_slices}}}
{\sphinxparam{\DUrole{n}{array3d}}\sphinxparamcomma \sphinxparam{\DUrole{n}{ax}\DUrole{o}{=}\DUrole{default_value}{None}}\sphinxparamcomma \sphinxparam{\DUrole{n}{slider}\DUrole{o}{=}\DUrole{default_value}{None}}}
{}
\pysigstopsignatures
\sphinxAtStartPar
Plots slices of a 3D array.
\begin{quote}\begin{description}
\sphinxlineitem{Parameters}\begin{itemize}
\item {} 
\sphinxAtStartPar
\sphinxstyleliteralstrong{\sphinxupquote{array3d}} (\sphinxstyleliteralemphasis{\sphinxupquote{np.ndarray}}) \textendash{} 3D array to plot.

\item {} 
\sphinxAtStartPar
\sphinxstyleliteralstrong{\sphinxupquote{ax}} (\sphinxstyleliteralemphasis{\sphinxupquote{matplotlib.axes.Axes}}\sphinxstyleliteralemphasis{\sphinxupquote{, }}\sphinxstyleliteralemphasis{\sphinxupquote{optional}}) \textendash{} Axes to plot on. Defaults to None.

\item {} 
\sphinxAtStartPar
\sphinxstyleliteralstrong{\sphinxupquote{slider}} (\sphinxstyleliteralemphasis{\sphinxupquote{matplotlib.widgets.Slider}}\sphinxstyleliteralemphasis{\sphinxupquote{, }}\sphinxstyleliteralemphasis{\sphinxupquote{optional}}) \textendash{} Slider for interactive plotting. Defaults to None.

\end{itemize}

\end{description}\end{quote}

\end{fulllineitems}

\index{remove\_source() (Box.Box method)@\spxentry{remove\_source()}\spxextra{Box.Box method}}

\begin{fulllineitems}
\phantomsection\label{\detokenize{source/Box:Box.Box.remove_source}}
\pysigstartsignatures
\pysiglinewithargsret
{\sphinxbfcode{\sphinxupquote{remove\_source}}}
{\sphinxparam{\DUrole{n}{source\_identifier}}}
{}
\pysigstopsignatures
\sphinxAtStartPar
Removes a source from the box by its identifier. The corresponding field is removed as well.
\begin{quote}\begin{description}
\sphinxlineitem{Parameters}
\sphinxAtStartPar
\sphinxstyleliteralstrong{\sphinxupquote{source\_identifier}} (\sphinxstyleliteralemphasis{\sphinxupquote{int}}) \textendash{} Identifier of the source to remove.

\end{description}\end{quote}

\end{fulllineitems}


\end{fulllineitems}

\index{BoxSIM (class in Box)@\spxentry{BoxSIM}\spxextra{class in Box}}

\begin{fulllineitems}
\phantomsection\label{\detokenize{source/Box:Box.BoxSIM}}
\pysigstartsignatures
\pysiglinewithargsret
{\sphinxbfcode{\sphinxupquote{class\DUrole{w}{ }}}\sphinxcode{\sphinxupquote{Box.}}\sphinxbfcode{\sphinxupquote{BoxSIM}}}
{\sphinxparam{\DUrole{n}{illumination: \textasciitilde{}Illumination.Illumination = \textless{}Illumination.Illumination object\textgreater{}}}\sphinxparamcomma \sphinxparam{\DUrole{n}{box\_size=10}}\sphinxparamcomma \sphinxparam{\DUrole{n}{point\_number=100}}\sphinxparamcomma \sphinxparam{\DUrole{n}{additional\_info=None}}}
{}
\pysigstopsignatures
\sphinxAtStartPar
Bases: {\hyperref[\detokenize{source/Box:Box.Box}]{\sphinxcrossref{\sphinxcode{\sphinxupquote{Box}}}}}

\sphinxAtStartPar
This class is an extension of the class Box that supports SIM specific operations,
such as illumination shifts.
\index{illumination (Box.BoxSIM attribute)@\spxentry{illumination}\spxextra{Box.BoxSIM attribute}}

\begin{fulllineitems}
\phantomsection\label{\detokenize{source/Box:Box.BoxSIM.illumination}}
\pysigstartsignatures
\pysigline
{\sphinxbfcode{\sphinxupquote{illumination}}}
\pysigstopsignatures
\sphinxAtStartPar
The illumination configuration.
\begin{quote}\begin{description}
\sphinxlineitem{Type}
\sphinxAtStartPar
IlluminationConfiguration

\end{description}\end{quote}

\end{fulllineitems}

\index{illuminations\_shifted (Box.BoxSIM attribute)@\spxentry{illuminations\_shifted}\spxextra{Box.BoxSIM attribute}}

\begin{fulllineitems}
\phantomsection\label{\detokenize{source/Box:Box.BoxSIM.illuminations_shifted}}
\pysigstartsignatures
\pysigline
{\sphinxbfcode{\sphinxupquote{illuminations\_shifted}}}
\pysigstopsignatures
\sphinxAtStartPar
Array of shifted illuminations for different angles and shifts.
\begin{quote}\begin{description}
\sphinxlineitem{Type}
\sphinxAtStartPar
np.ndarray

\end{description}\end{quote}

\end{fulllineitems}


\sphinxincludegraphics[]{inheritance-84910c6934282eedfd2d81e6afd6acd44710a201.pdf}
\index{compute\_total\_illumination() (Box.BoxSIM method)@\spxentry{compute\_total\_illumination()}\spxextra{Box.BoxSIM method}}

\begin{fulllineitems}
\phantomsection\label{\detokenize{source/Box:Box.BoxSIM.compute_total_illumination}}
\pysigstartsignatures
\pysiglinewithargsret
{\sphinxbfcode{\sphinxupquote{compute\_total\_illumination}}}
{}
{{ $\rightarrow$ ndarray}}
\pysigstopsignatures
\end{fulllineitems}

\index{get\_intensity() (Box.BoxSIM method)@\spxentry{get\_intensity()}\spxextra{Box.BoxSIM method}}

\begin{fulllineitems}
\phantomsection\label{\detokenize{source/Box:Box.BoxSIM.get_intensity}}
\pysigstartsignatures
\pysiglinewithargsret
{\sphinxbfcode{\sphinxupquote{get\_intensity}}}
{\sphinxparam{\DUrole{n}{r}\DUrole{p}{:}\DUrole{w}{ }\DUrole{n}{int}}\sphinxparamcomma \sphinxparam{\DUrole{n}{n}\DUrole{p}{:}\DUrole{w}{ }\DUrole{n}{int}}}
{{ $\rightarrow$ ndarray}}
\pysigstopsignatures
\end{fulllineitems}


\end{fulllineitems}

\index{Field (class in Box)@\spxentry{Field}\spxextra{class in Box}}

\begin{fulllineitems}
\phantomsection\label{\detokenize{source/Box:Box.Field}}
\pysigstartsignatures
\pysiglinewithargsret
{\sphinxbfcode{\sphinxupquote{class\DUrole{w}{ }}}\sphinxcode{\sphinxupquote{Box.}}\sphinxbfcode{\sphinxupquote{Field}}}
{\sphinxparam{\DUrole{n}{source}\DUrole{p}{:}\DUrole{w}{ }\DUrole{n}{{\hyperref[\detokenize{source/Sources:Sources.Source}]{\sphinxcrossref{Source}}}}}\sphinxparamcomma \sphinxparam{\DUrole{n}{grid}\DUrole{p}{:}\DUrole{w}{ }\DUrole{n}{ndarray\DUrole{p}{{[}}tuple\DUrole{p}{{[}}int\DUrole{p}{,}\DUrole{w}{ }int\DUrole{p}{,}\DUrole{w}{ }int\DUrole{p}{,}\DUrole{w}{ }\DUrole{m}{3}\DUrole{p}{{]}}\DUrole{p}{,}\DUrole{w}{ }float64\DUrole{p}{{]}}}}\sphinxparamcomma \sphinxparam{\DUrole{n}{identifier}\DUrole{p}{:}\DUrole{w}{ }\DUrole{n}{int}}}
{}
\pysigstopsignatures
\sphinxAtStartPar
Bases: \sphinxcode{\sphinxupquote{object}}

\sphinxAtStartPar
This class keeps field values within a given numeric volume.
\index{identifier (Box.Field attribute)@\spxentry{identifier}\spxextra{Box.Field attribute}}

\begin{fulllineitems}
\phantomsection\label{\detokenize{source/Box:Box.Field.identifier}}
\pysigstartsignatures
\pysigline
{\sphinxbfcode{\sphinxupquote{identifier}}}
\pysigstopsignatures
\sphinxAtStartPar
Unique identifier for the field.
\begin{quote}\begin{description}
\sphinxlineitem{Type}
\sphinxAtStartPar
int

\end{description}\end{quote}

\end{fulllineitems}

\index{field\_type (Box.Field attribute)@\spxentry{field\_type}\spxextra{Box.Field attribute}}

\begin{fulllineitems}
\phantomsection\label{\detokenize{source/Box:Box.Field.field_type}}
\pysigstartsignatures
\pysigline
{\sphinxbfcode{\sphinxupquote{field\_type}}}
\pysigstopsignatures
\sphinxAtStartPar
Type of the field (either “ElectricField” or “Intensity”).
\begin{quote}\begin{description}
\sphinxlineitem{Type}
\sphinxAtStartPar
str

\end{description}\end{quote}

\end{fulllineitems}

\index{source (Box.Field attribute)@\spxentry{source}\spxextra{Box.Field attribute}}

\begin{fulllineitems}
\phantomsection\label{\detokenize{source/Box:Box.Field.source}}
\pysigstartsignatures
\pysigline
{\sphinxbfcode{\sphinxupquote{source}}}
\pysigstopsignatures
\sphinxAtStartPar
The source that produces the field.
\begin{quote}\begin{description}
\sphinxlineitem{Type}
\sphinxAtStartPar
{\hyperref[\detokenize{source/Sources:Sources.Source}]{\sphinxcrossref{Source}}}

\end{description}\end{quote}

\end{fulllineitems}

\index{field (Box.Field attribute)@\spxentry{field}\spxextra{Box.Field attribute}}

\begin{fulllineitems}
\phantomsection\label{\detokenize{source/Box:Box.Field.field}}
\pysigstartsignatures
\pysigline
{\sphinxbfcode{\sphinxupquote{field}}}
\pysigstopsignatures
\sphinxAtStartPar
The computed field values.
\begin{quote}\begin{description}
\sphinxlineitem{Type}
\sphinxAtStartPar
np.ndarray

\end{description}\end{quote}

\end{fulllineitems}


\sphinxincludegraphics[]{inheritance-ddec035ca2f99d62c1b8dcf4547a2f570a80231b.pdf}

\end{fulllineitems}


\sphinxstepscope


\subsection{GUI module}
\label{\detokenize{source/GUI:module-GUI}}\label{\detokenize{source/GUI:gui-module}}\label{\detokenize{source/GUI::doc}}\index{module@\spxentry{module}!GUI@\spxentry{GUI}}\index{GUI@\spxentry{GUI}!module@\spxentry{module}}
\sphinxAtStartPar
GUI.py

\sphinxAtStartPar
This module contains the main graphical user interface (GUI) components of the application.

\sphinxAtStartPar
This module and related ones is currently a demo\sphinxhyphen{}version of the user\sphinxhyphen{}interface, and will
possibly be sufficiently modified or replaced in the future. For this reason, no in\sphinxhyphen{}depth
documentation is provided.

\sphinxincludegraphics[]{inheritance-6471f3fa7c04c0591308714ff1557a9eb660c9e2.pdf}
\index{MainWindow (class in GUI)@\spxentry{MainWindow}\spxextra{class in GUI}}

\begin{fulllineitems}
\phantomsection\label{\detokenize{source/GUI:GUI.MainWindow}}
\pysigstartsignatures
\pysiglinewithargsret
{\sphinxbfcode{\sphinxupquote{class\DUrole{w}{ }}}\sphinxcode{\sphinxupquote{GUI.}}\sphinxbfcode{\sphinxupquote{MainWindow}}}
{\sphinxparam{\DUrole{n}{box}\DUrole{o}{=}\DUrole{default_value}{None}}}
{}
\pysigstopsignatures
\sphinxAtStartPar
Bases: \sphinxcode{\sphinxupquote{QMainWindow}}

\sphinxincludegraphics[]{inheritance-64a91cb9f1991cb19db9ef9ec7419c1488dca1cc.pdf}
\index{add\_intensity\_plane\_wave() (GUI.MainWindow method)@\spxentry{add\_intensity\_plane\_wave()}\spxextra{GUI.MainWindow method}}

\begin{fulllineitems}
\phantomsection\label{\detokenize{source/GUI:GUI.MainWindow.add_intensity_plane_wave}}
\pysigstartsignatures
\pysiglinewithargsret
{\sphinxbfcode{\sphinxupquote{add\_intensity\_plane\_wave}}}
{\sphinxparam{\DUrole{n}{ipw}\DUrole{o}{=}\DUrole{default_value}{None}}}
{}
\pysigstopsignatures
\end{fulllineitems}

\index{add\_plane\_wave() (GUI.MainWindow method)@\spxentry{add\_plane\_wave()}\spxextra{GUI.MainWindow method}}

\begin{fulllineitems}
\phantomsection\label{\detokenize{source/GUI:GUI.MainWindow.add_plane_wave}}
\pysigstartsignatures
\pysiglinewithargsret
{\sphinxbfcode{\sphinxupquote{add\_plane\_wave}}}
{\sphinxparam{\DUrole{n}{ipw}\DUrole{o}{=}\DUrole{default_value}{None}}}
{}
\pysigstopsignatures
\end{fulllineitems}

\index{add\_point\_source() (GUI.MainWindow method)@\spxentry{add\_point\_source()}\spxextra{GUI.MainWindow method}}

\begin{fulllineitems}
\phantomsection\label{\detokenize{source/GUI:GUI.MainWindow.add_point_source}}
\pysigstartsignatures
\pysiglinewithargsret
{\sphinxbfcode{\sphinxupquote{add\_point\_source}}}
{}
{}
\pysigstopsignatures
\end{fulllineitems}

\index{add\_source() (GUI.MainWindow method)@\spxentry{add\_source()}\spxextra{GUI.MainWindow method}}

\begin{fulllineitems}
\phantomsection\label{\detokenize{source/GUI:GUI.MainWindow.add_source}}
\pysigstartsignatures
\pysiglinewithargsret
{\sphinxbfcode{\sphinxupquote{add\_source}}}
{\sphinxparam{\DUrole{n}{source}}}
{}
\pysigstopsignatures
\end{fulllineitems}

\index{add\_to\_box() (GUI.MainWindow method)@\spxentry{add\_to\_box()}\spxextra{GUI.MainWindow method}}

\begin{fulllineitems}
\phantomsection\label{\detokenize{source/GUI:GUI.MainWindow.add_to_box}}
\pysigstartsignatures
\pysiglinewithargsret
{\sphinxbfcode{\sphinxupquote{add\_to\_box}}}
{\sphinxparam{\DUrole{n}{initialized}}\sphinxparamcomma \sphinxparam{\DUrole{n}{source}}}
{}
\pysigstopsignatures
\end{fulllineitems}

\index{change\_plotting\_mode() (GUI.MainWindow method)@\spxentry{change\_plotting\_mode()}\spxextra{GUI.MainWindow method}}

\begin{fulllineitems}
\phantomsection\label{\detokenize{source/GUI:GUI.MainWindow.change_plotting_mode}}
\pysigstartsignatures
\pysiglinewithargsret
{\sphinxbfcode{\sphinxupquote{change\_plotting\_mode}}}
{}
{}
\pysigstopsignatures
\end{fulllineitems}

\index{change\_view3d() (GUI.MainWindow method)@\spxentry{change\_view3d()}\spxextra{GUI.MainWindow method}}

\begin{fulllineitems}
\phantomsection\label{\detokenize{source/GUI:GUI.MainWindow.change_view3d}}
\pysigstartsignatures
\pysiglinewithargsret
{\sphinxbfcode{\sphinxupquote{change\_view3d}}}
{}
{}
\pysigstopsignatures
\end{fulllineitems}

\index{choose\_plotting\_mode() (GUI.MainWindow method)@\spxentry{choose\_plotting\_mode()}\spxextra{GUI.MainWindow method}}

\begin{fulllineitems}
\phantomsection\label{\detokenize{source/GUI:GUI.MainWindow.choose_plotting_mode}}
\pysigstartsignatures
\pysiglinewithargsret
{\sphinxbfcode{\sphinxupquote{choose\_plotting\_mode}}}
{\sphinxparam{\DUrole{n}{Z}}}
{}
\pysigstopsignatures
\end{fulllineitems}

\index{choose\_view3d() (GUI.MainWindow method)@\spxentry{choose\_view3d()}\spxextra{GUI.MainWindow method}}

\begin{fulllineitems}
\phantomsection\label{\detokenize{source/GUI:GUI.MainWindow.choose_view3d}}
\pysigstartsignatures
\pysiglinewithargsret
{\sphinxbfcode{\sphinxupquote{choose\_view3d}}}
{\sphinxparam{\DUrole{n}{array}}\sphinxparamcomma \sphinxparam{\DUrole{n}{number}}}
{}
\pysigstopsignatures
\end{fulllineitems}

\index{clear\_layout() (GUI.MainWindow method)@\spxentry{clear\_layout()}\spxextra{GUI.MainWindow method}}

\begin{fulllineitems}
\phantomsection\label{\detokenize{source/GUI:GUI.MainWindow.clear_layout}}
\pysigstartsignatures
\pysiglinewithargsret
{\sphinxbfcode{\sphinxupquote{clear\_layout}}}
{\sphinxparam{\DUrole{n}{layout}}}
{}
\pysigstopsignatures
\end{fulllineitems}

\index{compute\_and\_plot\_fourier\_space() (GUI.MainWindow method)@\spxentry{compute\_and\_plot\_fourier\_space()}\spxextra{GUI.MainWindow method}}

\begin{fulllineitems}
\phantomsection\label{\detokenize{source/GUI:GUI.MainWindow.compute_and_plot_fourier_space}}
\pysigstartsignatures
\pysiglinewithargsret
{\sphinxbfcode{\sphinxupquote{compute\_and\_plot\_fourier\_space}}}
{}
{}
\pysigstopsignatures
\end{fulllineitems}

\index{compute\_and\_plot\_from\_electric\_field() (GUI.MainWindow method)@\spxentry{compute\_and\_plot\_from\_electric\_field()}\spxextra{GUI.MainWindow method}}

\begin{fulllineitems}
\phantomsection\label{\detokenize{source/GUI:GUI.MainWindow.compute_and_plot_from_electric_field}}
\pysigstartsignatures
\pysiglinewithargsret
{\sphinxbfcode{\sphinxupquote{compute\_and\_plot\_from\_electric\_field}}}
{}
{}
\pysigstopsignatures
\end{fulllineitems}

\index{compute\_and\_plot\_from\_intensity\_sources() (GUI.MainWindow method)@\spxentry{compute\_and\_plot\_from\_intensity\_sources()}\spxextra{GUI.MainWindow method}}

\begin{fulllineitems}
\phantomsection\label{\detokenize{source/GUI:GUI.MainWindow.compute_and_plot_from_intensity_sources}}
\pysigstartsignatures
\pysiglinewithargsret
{\sphinxbfcode{\sphinxupquote{compute\_and\_plot\_from\_intensity\_sources}}}
{}
{}
\pysigstopsignatures
\end{fulllineitems}

\index{compute\_next\_shift() (GUI.MainWindow method)@\spxentry{compute\_next\_shift()}\spxextra{GUI.MainWindow method}}

\begin{fulllineitems}
\phantomsection\label{\detokenize{source/GUI:GUI.MainWindow.compute_next_shift}}
\pysigstartsignatures
\pysiglinewithargsret
{\sphinxbfcode{\sphinxupquote{compute\_next\_shift}}}
{}
{}
\pysigstopsignatures
\end{fulllineitems}

\index{compute\_numerically\_approximated\_intensities() (GUI.MainWindow method)@\spxentry{compute\_numerically\_approximated\_intensities()}\spxextra{GUI.MainWindow method}}

\begin{fulllineitems}
\phantomsection\label{\detokenize{source/GUI:GUI.MainWindow.compute_numerically_approximated_intensities}}
\pysigstartsignatures
\pysiglinewithargsret
{\sphinxbfcode{\sphinxupquote{compute\_numerically\_approximated\_intensities}}}
{}
{}
\pysigstopsignatures
\end{fulllineitems}

\index{compute\_total\_intensity() (GUI.MainWindow method)@\spxentry{compute\_total\_intensity()}\spxextra{GUI.MainWindow method}}

\begin{fulllineitems}
\phantomsection\label{\detokenize{source/GUI:GUI.MainWindow.compute_total_intensity}}
\pysigstartsignatures
\pysiglinewithargsret
{\sphinxbfcode{\sphinxupquote{compute\_total\_intensity}}}
{}
{}
\pysigstopsignatures
\end{fulllineitems}

\index{get\_ipw\_from\_pw() (GUI.MainWindow method)@\spxentry{get\_ipw\_from\_pw()}\spxextra{GUI.MainWindow method}}

\begin{fulllineitems}
\phantomsection\label{\detokenize{source/GUI:GUI.MainWindow.get_ipw_from_pw}}
\pysigstartsignatures
\pysiglinewithargsret
{\sphinxbfcode{\sphinxupquote{get\_ipw\_from\_pw}}}
{}
{}
\pysigstopsignatures
\end{fulllineitems}

\index{init\_ui() (GUI.MainWindow method)@\spxentry{init\_ui()}\spxextra{GUI.MainWindow method}}

\begin{fulllineitems}
\phantomsection\label{\detokenize{source/GUI:GUI.MainWindow.init_ui}}
\pysigstartsignatures
\pysiglinewithargsret
{\sphinxbfcode{\sphinxupquote{init\_ui}}}
{}
{}
\pysigstopsignatures
\end{fulllineitems}

\index{load\_config() (GUI.MainWindow method)@\spxentry{load\_config()}\spxextra{GUI.MainWindow method}}

\begin{fulllineitems}
\phantomsection\label{\detokenize{source/GUI:GUI.MainWindow.load_config}}
\pysigstartsignatures
\pysiglinewithargsret
{\sphinxbfcode{\sphinxupquote{load\_config}}}
{}
{}
\pysigstopsignatures
\end{fulllineitems}

\index{load\_illumination() (GUI.MainWindow method)@\spxentry{load\_illumination()}\spxextra{GUI.MainWindow method}}

\begin{fulllineitems}
\phantomsection\label{\detokenize{source/GUI:GUI.MainWindow.load_illumination}}
\pysigstartsignatures
\pysiglinewithargsret
{\sphinxbfcode{\sphinxupquote{load\_illumination}}}
{}
{}
\pysigstopsignatures
\end{fulllineitems}

\index{on\_option\_selected() (GUI.MainWindow method)@\spxentry{on\_option\_selected()}\spxextra{GUI.MainWindow method}}

\begin{fulllineitems}
\phantomsection\label{\detokenize{source/GUI:GUI.MainWindow.on_option_selected}}
\pysigstartsignatures
\pysiglinewithargsret
{\sphinxbfcode{\sphinxupquote{on\_option\_selected}}}
{\sphinxparam{\DUrole{n}{index}}}
{}
\pysigstopsignatures
\end{fulllineitems}

\index{plot\_fourier\_space\_slices() (GUI.MainWindow method)@\spxentry{plot\_fourier\_space\_slices()}\spxextra{GUI.MainWindow method}}

\begin{fulllineitems}
\phantomsection\label{\detokenize{source/GUI:GUI.MainWindow.plot_fourier_space_slices}}
\pysigstartsignatures
\pysiglinewithargsret
{\sphinxbfcode{\sphinxupquote{plot\_fourier\_space\_slices}}}
{\sphinxparam{\DUrole{n}{intensity}\DUrole{o}{=}\DUrole{default_value}{None}}}
{}
\pysigstopsignatures
\end{fulllineitems}

\index{plot\_intensity\_slices() (GUI.MainWindow method)@\spxentry{plot\_intensity\_slices()}\spxextra{GUI.MainWindow method}}

\begin{fulllineitems}
\phantomsection\label{\detokenize{source/GUI:GUI.MainWindow.plot_intensity_slices}}
\pysigstartsignatures
\pysiglinewithargsret
{\sphinxbfcode{\sphinxupquote{plot\_intensity\_slices}}}
{\sphinxparam{\DUrole{n}{intensity}\DUrole{o}{=}\DUrole{default_value}{None}}}
{}
\pysigstopsignatures
\end{fulllineitems}

\index{plot\_numerically\_approximated\_intensity() (GUI.MainWindow method)@\spxentry{plot\_numerically\_approximated\_intensity()}\spxextra{GUI.MainWindow method}}

\begin{fulllineitems}
\phantomsection\label{\detokenize{source/GUI:GUI.MainWindow.plot_numerically_approximated_intensity}}
\pysigstartsignatures
\pysiglinewithargsret
{\sphinxbfcode{\sphinxupquote{plot\_numerically\_approximated\_intensity}}}
{}
{}
\pysigstopsignatures
\end{fulllineitems}

\index{plot\_numerically\_approximated\_intensity\_fourier\_space() (GUI.MainWindow method)@\spxentry{plot\_numerically\_approximated\_intensity\_fourier\_space()}\spxextra{GUI.MainWindow method}}

\begin{fulllineitems}
\phantomsection\label{\detokenize{source/GUI:GUI.MainWindow.plot_numerically_approximated_intensity_fourier_space}}
\pysigstartsignatures
\pysiglinewithargsret
{\sphinxbfcode{\sphinxupquote{plot\_numerically\_approximated\_intensity\_fourier\_space}}}
{}
{}
\pysigstopsignatures
\end{fulllineitems}

\index{plot\_shift\_arrow() (GUI.MainWindow method)@\spxentry{plot\_shift\_arrow()}\spxextra{GUI.MainWindow method}}

\begin{fulllineitems}
\phantomsection\label{\detokenize{source/GUI:GUI.MainWindow.plot_shift_arrow}}
\pysigstartsignatures
\pysiglinewithargsret
{\sphinxbfcode{\sphinxupquote{plot\_shift\_arrow}}}
{}
{}
\pysigstopsignatures
\end{fulllineitems}

\index{remove\_source() (GUI.MainWindow method)@\spxentry{remove\_source()}\spxextra{GUI.MainWindow method}}

\begin{fulllineitems}
\phantomsection\label{\detokenize{source/GUI:GUI.MainWindow.remove_source}}
\pysigstartsignatures
\pysiglinewithargsret
{\sphinxbfcode{\sphinxupquote{remove\_source}}}
{\sphinxparam{\DUrole{n}{initializer}}}
{}
\pysigstopsignatures
\end{fulllineitems}

\index{save\_config() (GUI.MainWindow method)@\spxentry{save\_config()}\spxextra{GUI.MainWindow method}}

\begin{fulllineitems}
\phantomsection\label{\detokenize{source/GUI:GUI.MainWindow.save_config}}
\pysigstartsignatures
\pysiglinewithargsret
{\sphinxbfcode{\sphinxupquote{save\_config}}}
{}
{}
\pysigstopsignatures
\end{fulllineitems}


\end{fulllineitems}

\index{PlottingMode (class in GUI)@\spxentry{PlottingMode}\spxextra{class in GUI}}

\begin{fulllineitems}
\phantomsection\label{\detokenize{source/GUI:GUI.PlottingMode}}
\pysigstartsignatures
\pysiglinewithargsret
{\sphinxbfcode{\sphinxupquote{class\DUrole{w}{ }}}\sphinxcode{\sphinxupquote{GUI.}}\sphinxbfcode{\sphinxupquote{PlottingMode}}}
{\sphinxparam{\DUrole{n}{value}}\sphinxparamcomma \sphinxparam{\DUrole{n}{names=\textless{}not given\textgreater{}}}\sphinxparamcomma \sphinxparam{\DUrole{n}{*values}}\sphinxparamcomma \sphinxparam{\DUrole{n}{module=None}}\sphinxparamcomma \sphinxparam{\DUrole{n}{qualname=None}}\sphinxparamcomma \sphinxparam{\DUrole{n}{type=None}}\sphinxparamcomma \sphinxparam{\DUrole{n}{start=1}}\sphinxparamcomma \sphinxparam{\DUrole{n}{boundary=None}}}
{}
\pysigstopsignatures
\sphinxAtStartPar
Bases: \sphinxcode{\sphinxupquote{Enum}}

\sphinxincludegraphics[]{inheritance-ed5509469ffb87e1442b1866e34d63f835877588.pdf}
\index{linear (GUI.PlottingMode attribute)@\spxentry{linear}\spxextra{GUI.PlottingMode attribute}}

\begin{fulllineitems}
\phantomsection\label{\detokenize{source/GUI:GUI.PlottingMode.linear}}
\pysigstartsignatures
\pysigline
{\sphinxbfcode{\sphinxupquote{linear}}\sphinxbfcode{\sphinxupquote{\DUrole{w}{ }\DUrole{p}{=}\DUrole{w}{ }0}}}
\pysigstopsignatures
\end{fulllineitems}

\index{logarithmic (GUI.PlottingMode attribute)@\spxentry{logarithmic}\spxextra{GUI.PlottingMode attribute}}

\begin{fulllineitems}
\phantomsection\label{\detokenize{source/GUI:GUI.PlottingMode.logarithmic}}
\pysigstartsignatures
\pysigline
{\sphinxbfcode{\sphinxupquote{logarithmic}}\sphinxbfcode{\sphinxupquote{\DUrole{w}{ }\DUrole{p}{=}\DUrole{w}{ }1}}}
\pysigstopsignatures
\end{fulllineitems}

\index{mixed (GUI.PlottingMode attribute)@\spxentry{mixed}\spxextra{GUI.PlottingMode attribute}}

\begin{fulllineitems}
\phantomsection\label{\detokenize{source/GUI:GUI.PlottingMode.mixed}}
\pysigstartsignatures
\pysigline
{\sphinxbfcode{\sphinxupquote{mixed}}\sphinxbfcode{\sphinxupquote{\DUrole{w}{ }\DUrole{p}{=}\DUrole{w}{ }2}}}
\pysigstopsignatures
\end{fulllineitems}


\end{fulllineitems}

\index{View (class in GUI)@\spxentry{View}\spxextra{class in GUI}}

\begin{fulllineitems}
\phantomsection\label{\detokenize{source/GUI:GUI.View}}
\pysigstartsignatures
\pysiglinewithargsret
{\sphinxbfcode{\sphinxupquote{class\DUrole{w}{ }}}\sphinxcode{\sphinxupquote{GUI.}}\sphinxbfcode{\sphinxupquote{View}}}
{\sphinxparam{\DUrole{n}{value}}\sphinxparamcomma \sphinxparam{\DUrole{n}{names=\textless{}not given\textgreater{}}}\sphinxparamcomma \sphinxparam{\DUrole{n}{*values}}\sphinxparamcomma \sphinxparam{\DUrole{n}{module=None}}\sphinxparamcomma \sphinxparam{\DUrole{n}{qualname=None}}\sphinxparamcomma \sphinxparam{\DUrole{n}{type=None}}\sphinxparamcomma \sphinxparam{\DUrole{n}{start=1}}\sphinxparamcomma \sphinxparam{\DUrole{n}{boundary=None}}}
{}
\pysigstopsignatures
\sphinxAtStartPar
Bases: \sphinxcode{\sphinxupquote{Enum}}

\sphinxincludegraphics[]{inheritance-903cb7554ad99d4ede99ecfd8c5c4d6b7a1ac160.pdf}
\index{XY (GUI.View attribute)@\spxentry{XY}\spxextra{GUI.View attribute}}

\begin{fulllineitems}
\phantomsection\label{\detokenize{source/GUI:GUI.View.XY}}
\pysigstartsignatures
\pysigline
{\sphinxbfcode{\sphinxupquote{XY}}\sphinxbfcode{\sphinxupquote{\DUrole{w}{ }\DUrole{p}{=}\DUrole{w}{ }0}}}
\pysigstopsignatures
\end{fulllineitems}

\index{XZ (GUI.View attribute)@\spxentry{XZ}\spxextra{GUI.View attribute}}

\begin{fulllineitems}
\phantomsection\label{\detokenize{source/GUI:GUI.View.XZ}}
\pysigstartsignatures
\pysigline
{\sphinxbfcode{\sphinxupquote{XZ}}\sphinxbfcode{\sphinxupquote{\DUrole{w}{ }\DUrole{p}{=}\DUrole{w}{ }2}}}
\pysigstopsignatures
\end{fulllineitems}

\index{YZ (GUI.View attribute)@\spxentry{YZ}\spxextra{GUI.View attribute}}

\begin{fulllineitems}
\phantomsection\label{\detokenize{source/GUI:GUI.View.YZ}}
\pysigstartsignatures
\pysigline
{\sphinxbfcode{\sphinxupquote{YZ}}\sphinxbfcode{\sphinxupquote{\DUrole{w}{ }\DUrole{p}{=}\DUrole{w}{ }1}}}
\pysigstopsignatures
\end{fulllineitems}


\end{fulllineitems}


\sphinxstepscope


\subsection{GUIInitializationWidgets module}
\label{\detokenize{source/GUIInitializationWidgets:module-GUIInitializationWidgets}}\label{\detokenize{source/GUIInitializationWidgets:guiinitializationwidgets-module}}\label{\detokenize{source/GUIInitializationWidgets::doc}}\index{module@\spxentry{module}!GUIInitializationWidgets@\spxentry{GUIInitializationWidgets}}\index{GUIInitializationWidgets@\spxentry{GUIInitializationWidgets}!module@\spxentry{module}}
\sphinxAtStartPar
GUIInitializationWidgets.py

\sphinxAtStartPar
This module contains classes and functions for initializing GUI widgets.

\sphinxAtStartPar
This module and related ones is currently a demo\sphinxhyphen{}version of the user\sphinxhyphen{}interface, and will
possibly be sufficiently modified or replaced in the future. For this reason, no in\sphinxhyphen{}depth
documentation is provided.

\sphinxincludegraphics[]{inheritance-b1be19f8ffa9990df20e96794f8574a1ffde96aa.pdf}
\index{InitializationWidget (class in GUIInitializationWidgets)@\spxentry{InitializationWidget}\spxextra{class in GUIInitializationWidgets}}

\begin{fulllineitems}
\phantomsection\label{\detokenize{source/GUIInitializationWidgets:GUIInitializationWidgets.InitializationWidget}}
\pysigstartsignatures
\pysigline
{\sphinxbfcode{\sphinxupquote{class\DUrole{w}{ }}}\sphinxcode{\sphinxupquote{GUIInitializationWidgets.}}\sphinxbfcode{\sphinxupquote{InitializationWidget}}}
\pysigstopsignatures
\sphinxAtStartPar
Bases: \sphinxcode{\sphinxupquote{QWidget}}

\sphinxincludegraphics[]{inheritance-e6d984e8752d91dad692b7c632190ef93565cd54.pdf}
\index{on\_click\_ok() (GUIInitializationWidgets.InitializationWidget method)@\spxentry{on\_click\_ok()}\spxextra{GUIInitializationWidgets.InitializationWidget method}}

\begin{fulllineitems}
\phantomsection\label{\detokenize{source/GUIInitializationWidgets:GUIInitializationWidgets.InitializationWidget.on_click_ok}}
\pysigstartsignatures
\pysiglinewithargsret
{\sphinxbfcode{\sphinxupquote{abstract\DUrole{w}{ }}}\sphinxbfcode{\sphinxupquote{on\_click\_ok}}}
{}
{}
\pysigstopsignatures
\end{fulllineitems}

\index{request\_data() (GUIInitializationWidgets.InitializationWidget method)@\spxentry{request\_data()}\spxextra{GUIInitializationWidgets.InitializationWidget method}}

\begin{fulllineitems}
\phantomsection\label{\detokenize{source/GUIInitializationWidgets:GUIInitializationWidgets.InitializationWidget.request_data}}
\pysigstartsignatures
\pysiglinewithargsret
{\sphinxbfcode{\sphinxupquote{abstract\DUrole{w}{ }}}\sphinxbfcode{\sphinxupquote{request\_data}}}
{}
{}
\pysigstopsignatures
\end{fulllineitems}


\end{fulllineitems}

\index{IntensityPlaneWaveInitializationWidget (class in GUIInitializationWidgets)@\spxentry{IntensityPlaneWaveInitializationWidget}\spxextra{class in GUIInitializationWidgets}}

\begin{fulllineitems}
\phantomsection\label{\detokenize{source/GUIInitializationWidgets:GUIInitializationWidgets.IntensityPlaneWaveInitializationWidget}}
\pysigstartsignatures
\pysigline
{\sphinxbfcode{\sphinxupquote{class\DUrole{w}{ }}}\sphinxcode{\sphinxupquote{GUIInitializationWidgets.}}\sphinxbfcode{\sphinxupquote{IntensityPlaneWaveInitializationWidget}}}
\pysigstopsignatures
\sphinxAtStartPar
Bases: {\hyperref[\detokenize{source/GUIInitializationWidgets:GUIInitializationWidgets.InitializationWidget}]{\sphinxcrossref{\sphinxcode{\sphinxupquote{InitializationWidget}}}}}

\sphinxincludegraphics[]{inheritance-127f6aa82ac696ee723baf5da0b60f1eca0ca563.pdf}
\index{on\_click\_ok() (GUIInitializationWidgets.IntensityPlaneWaveInitializationWidget method)@\spxentry{on\_click\_ok()}\spxextra{GUIInitializationWidgets.IntensityPlaneWaveInitializationWidget method}}

\begin{fulllineitems}
\phantomsection\label{\detokenize{source/GUIInitializationWidgets:GUIInitializationWidgets.IntensityPlaneWaveInitializationWidget.on_click_ok}}
\pysigstartsignatures
\pysiglinewithargsret
{\sphinxbfcode{\sphinxupquote{on\_click\_ok}}}
{}
{}
\pysigstopsignatures
\end{fulllineitems}

\index{request\_data() (GUIInitializationWidgets.IntensityPlaneWaveInitializationWidget method)@\spxentry{request\_data()}\spxextra{GUIInitializationWidgets.IntensityPlaneWaveInitializationWidget method}}

\begin{fulllineitems}
\phantomsection\label{\detokenize{source/GUIInitializationWidgets:GUIInitializationWidgets.IntensityPlaneWaveInitializationWidget.request_data}}
\pysigstartsignatures
\pysiglinewithargsret
{\sphinxbfcode{\sphinxupquote{request\_data}}}
{}
{}
\pysigstopsignatures
\end{fulllineitems}

\index{sendInfo (GUIInitializationWidgets.IntensityPlaneWaveInitializationWidget attribute)@\spxentry{sendInfo}\spxextra{GUIInitializationWidgets.IntensityPlaneWaveInitializationWidget attribute}}

\begin{fulllineitems}
\phantomsection\label{\detokenize{source/GUIInitializationWidgets:GUIInitializationWidgets.IntensityPlaneWaveInitializationWidget.sendInfo}}
\pysigstartsignatures
\pysigline
{\sphinxbfcode{\sphinxupquote{sendInfo}}}
\pysigstopsignatures
\sphinxAtStartPar
int = …, arguments: Sequence = …) \sphinxhyphen{}\textgreater{} PYQT\_SIGNAL

\sphinxAtStartPar
types is normally a sequence of individual types.  Each type is either a
type object or a string that is the name of a C++ type.  Alternatively
each type could itself be a sequence of types each describing a different
overloaded signal.
name is the optional C++ name of the signal.  If it is not specified then
the name of the class attribute that is bound to the signal is used.
revision is the optional revision of the signal that is exported to QML.
If it is not specified then 0 is used.
arguments is the optional sequence of the names of the signal’s arguments.
\begin{quote}\begin{description}
\sphinxlineitem{Type}
\sphinxAtStartPar
pyqtSignal({\color{red}\bfseries{}*}types, name

\sphinxlineitem{Type}
\sphinxAtStartPar
str = …, revision

\end{description}\end{quote}

\end{fulllineitems}


\end{fulllineitems}

\index{PlaneWaveInitializationWidget (class in GUIInitializationWidgets)@\spxentry{PlaneWaveInitializationWidget}\spxextra{class in GUIInitializationWidgets}}

\begin{fulllineitems}
\phantomsection\label{\detokenize{source/GUIInitializationWidgets:GUIInitializationWidgets.PlaneWaveInitializationWidget}}
\pysigstartsignatures
\pysigline
{\sphinxbfcode{\sphinxupquote{class\DUrole{w}{ }}}\sphinxcode{\sphinxupquote{GUIInitializationWidgets.}}\sphinxbfcode{\sphinxupquote{PlaneWaveInitializationWidget}}}
\pysigstopsignatures
\sphinxAtStartPar
Bases: {\hyperref[\detokenize{source/GUIInitializationWidgets:GUIInitializationWidgets.InitializationWidget}]{\sphinxcrossref{\sphinxcode{\sphinxupquote{InitializationWidget}}}}}

\sphinxincludegraphics[]{inheritance-c6a3d96226dfad4b3cdf165b055676c81de490fb.pdf}
\index{on\_click\_ok() (GUIInitializationWidgets.PlaneWaveInitializationWidget method)@\spxentry{on\_click\_ok()}\spxextra{GUIInitializationWidgets.PlaneWaveInitializationWidget method}}

\begin{fulllineitems}
\phantomsection\label{\detokenize{source/GUIInitializationWidgets:GUIInitializationWidgets.PlaneWaveInitializationWidget.on_click_ok}}
\pysigstartsignatures
\pysiglinewithargsret
{\sphinxbfcode{\sphinxupquote{on\_click\_ok}}}
{}
{}
\pysigstopsignatures
\end{fulllineitems}

\index{request\_data() (GUIInitializationWidgets.PlaneWaveInitializationWidget method)@\spxentry{request\_data()}\spxextra{GUIInitializationWidgets.PlaneWaveInitializationWidget method}}

\begin{fulllineitems}
\phantomsection\label{\detokenize{source/GUIInitializationWidgets:GUIInitializationWidgets.PlaneWaveInitializationWidget.request_data}}
\pysigstartsignatures
\pysiglinewithargsret
{\sphinxbfcode{\sphinxupquote{request\_data}}}
{}
{}
\pysigstopsignatures
\end{fulllineitems}

\index{sendInfo (GUIInitializationWidgets.PlaneWaveInitializationWidget attribute)@\spxentry{sendInfo}\spxextra{GUIInitializationWidgets.PlaneWaveInitializationWidget attribute}}

\begin{fulllineitems}
\phantomsection\label{\detokenize{source/GUIInitializationWidgets:GUIInitializationWidgets.PlaneWaveInitializationWidget.sendInfo}}
\pysigstartsignatures
\pysigline
{\sphinxbfcode{\sphinxupquote{sendInfo}}}
\pysigstopsignatures
\sphinxAtStartPar
int = …, arguments: Sequence = …) \sphinxhyphen{}\textgreater{} PYQT\_SIGNAL

\sphinxAtStartPar
types is normally a sequence of individual types.  Each type is either a
type object or a string that is the name of a C++ type.  Alternatively
each type could itself be a sequence of types each describing a different
overloaded signal.
name is the optional C++ name of the signal.  If it is not specified then
the name of the class attribute that is bound to the signal is used.
revision is the optional revision of the signal that is exported to QML.
If it is not specified then 0 is used.
arguments is the optional sequence of the names of the signal’s arguments.
\begin{quote}\begin{description}
\sphinxlineitem{Type}
\sphinxAtStartPar
pyqtSignal({\color{red}\bfseries{}*}types, name

\sphinxlineitem{Type}
\sphinxAtStartPar
str = …, revision

\end{description}\end{quote}

\end{fulllineitems}


\end{fulllineitems}


\sphinxstepscope


\subsection{GUIWidgets module}
\label{\detokenize{source/GUIWidgets:module-GUIWidgets}}\label{\detokenize{source/GUIWidgets:guiwidgets-module}}\label{\detokenize{source/GUIWidgets::doc}}\index{module@\spxentry{module}!GUIWidgets@\spxentry{GUIWidgets}}\index{GUIWidgets@\spxentry{GUIWidgets}!module@\spxentry{module}}
\sphinxAtStartPar
GUIWidgets.py

\sphinxAtStartPar
This module contains utility widgets for the GUI components.

\sphinxAtStartPar
This module and related ones is currently a demo\sphinxhyphen{}version of the user\sphinxhyphen{}interface, and will
possibly be sufficiently modified or replaced in the future. For this reason, no in\sphinxhyphen{}depth
documentation is provided.

\sphinxincludegraphics[]{inheritance-332bca98f1381428ff17a1f65bfb660615ac58b0.pdf}
\index{IntensityPlaneWaveWidget (class in GUIWidgets)@\spxentry{IntensityPlaneWaveWidget}\spxextra{class in GUIWidgets}}

\begin{fulllineitems}
\phantomsection\label{\detokenize{source/GUIWidgets:GUIWidgets.IntensityPlaneWaveWidget}}
\pysigstartsignatures
\pysiglinewithargsret
{\sphinxbfcode{\sphinxupquote{class\DUrole{w}{ }}}\sphinxcode{\sphinxupquote{GUIWidgets.}}\sphinxbfcode{\sphinxupquote{IntensityPlaneWaveWidget}}}
{\sphinxparam{\DUrole{n}{ipw}}}
{}
\pysigstopsignatures
\sphinxAtStartPar
Bases: {\hyperref[\detokenize{source/GUIWidgets:GUIWidgets.SourceWidget}]{\sphinxcrossref{\sphinxcode{\sphinxupquote{SourceWidget}}}}}

\sphinxincludegraphics[]{inheritance-2292c555b561971213a62a46354458a920adf314.pdf}
\index{change\_widget() (GUIWidgets.IntensityPlaneWaveWidget method)@\spxentry{change\_widget()}\spxextra{GUIWidgets.IntensityPlaneWaveWidget method}}

\begin{fulllineitems}
\phantomsection\label{\detokenize{source/GUIWidgets:GUIWidgets.IntensityPlaneWaveWidget.change_widget}}
\pysigstartsignatures
\pysiglinewithargsret
{\sphinxbfcode{\sphinxupquote{change\_widget}}}
{}
{}
\pysigstopsignatures
\end{fulllineitems}

\index{contextMenuEvent() (GUIWidgets.IntensityPlaneWaveWidget method)@\spxentry{contextMenuEvent()}\spxextra{GUIWidgets.IntensityPlaneWaveWidget method}}

\begin{fulllineitems}
\phantomsection\label{\detokenize{source/GUIWidgets:GUIWidgets.IntensityPlaneWaveWidget.contextMenuEvent}}
\pysigstartsignatures
\pysiglinewithargsret
{\sphinxbfcode{\sphinxupquote{contextMenuEvent}}}
{\sphinxparam{\DUrole{n}{self}}\sphinxparamcomma \sphinxparam{\DUrole{n}{a0}\DUrole{p}{:}\DUrole{w}{ }\DUrole{n}{QContextMenuEvent\DUrole{w}{ }\DUrole{p}{|}\DUrole{w}{ }None}}}
{}
\pysigstopsignatures
\end{fulllineitems}

\index{init\_ui() (GUIWidgets.IntensityPlaneWaveWidget method)@\spxentry{init\_ui()}\spxextra{GUIWidgets.IntensityPlaneWaveWidget method}}

\begin{fulllineitems}
\phantomsection\label{\detokenize{source/GUIWidgets:GUIWidgets.IntensityPlaneWaveWidget.init_ui}}
\pysigstartsignatures
\pysiglinewithargsret
{\sphinxbfcode{\sphinxupquote{init\_ui}}}
{\sphinxparam{\DUrole{n}{ipw}}}
{}
\pysigstopsignatures
\end{fulllineitems}

\index{isSet (GUIWidgets.IntensityPlaneWaveWidget attribute)@\spxentry{isSet}\spxextra{GUIWidgets.IntensityPlaneWaveWidget attribute}}

\begin{fulllineitems}
\phantomsection\label{\detokenize{source/GUIWidgets:GUIWidgets.IntensityPlaneWaveWidget.isSet}}
\pysigstartsignatures
\pysigline
{\sphinxbfcode{\sphinxupquote{isSet}}}
\pysigstopsignatures
\sphinxAtStartPar
int = …, arguments: Sequence = …) \sphinxhyphen{}\textgreater{} PYQT\_SIGNAL

\sphinxAtStartPar
types is normally a sequence of individual types.  Each type is either a
type object or a string that is the name of a C++ type.  Alternatively
each type could itself be a sequence of types each describing a different
overloaded signal.
name is the optional C++ name of the signal.  If it is not specified then
the name of the class attribute that is bound to the signal is used.
revision is the optional revision of the signal that is exported to QML.
If it is not specified then 0 is used.
arguments is the optional sequence of the names of the signal’s arguments.
\begin{quote}\begin{description}
\sphinxlineitem{Type}
\sphinxAtStartPar
pyqtSignal({\color{red}\bfseries{}*}types, name

\sphinxlineitem{Type}
\sphinxAtStartPar
str = …, revision

\end{description}\end{quote}

\end{fulllineitems}

\index{on\_receive\_info() (GUIWidgets.IntensityPlaneWaveWidget method)@\spxentry{on\_receive\_info()}\spxextra{GUIWidgets.IntensityPlaneWaveWidget method}}

\begin{fulllineitems}
\phantomsection\label{\detokenize{source/GUIWidgets:GUIWidgets.IntensityPlaneWaveWidget.on_receive_info}}
\pysigstartsignatures
\pysiglinewithargsret
{\sphinxbfcode{\sphinxupquote{on\_receive\_info}}}
{\sphinxparam{\DUrole{n}{info}}}
{}
\pysigstopsignatures
\end{fulllineitems}


\end{fulllineitems}

\index{PlaneWaveWidget (class in GUIWidgets)@\spxentry{PlaneWaveWidget}\spxextra{class in GUIWidgets}}

\begin{fulllineitems}
\phantomsection\label{\detokenize{source/GUIWidgets:GUIWidgets.PlaneWaveWidget}}
\pysigstartsignatures
\pysiglinewithargsret
{\sphinxbfcode{\sphinxupquote{class\DUrole{w}{ }}}\sphinxcode{\sphinxupquote{GUIWidgets.}}\sphinxbfcode{\sphinxupquote{PlaneWaveWidget}}}
{\sphinxparam{\DUrole{n}{pw}\DUrole{o}{=}\DUrole{default_value}{None}}}
{}
\pysigstopsignatures
\sphinxAtStartPar
Bases: {\hyperref[\detokenize{source/GUIWidgets:GUIWidgets.SourceWidget}]{\sphinxcrossref{\sphinxcode{\sphinxupquote{SourceWidget}}}}}

\sphinxincludegraphics[]{inheritance-05ac39742e4fafc36e49b07dcd046f897c1d5f9d.pdf}
\index{change\_widget() (GUIWidgets.PlaneWaveWidget method)@\spxentry{change\_widget()}\spxextra{GUIWidgets.PlaneWaveWidget method}}

\begin{fulllineitems}
\phantomsection\label{\detokenize{source/GUIWidgets:GUIWidgets.PlaneWaveWidget.change_widget}}
\pysigstartsignatures
\pysiglinewithargsret
{\sphinxbfcode{\sphinxupquote{change\_widget}}}
{}
{}
\pysigstopsignatures
\end{fulllineitems}

\index{contextMenuEvent() (GUIWidgets.PlaneWaveWidget method)@\spxentry{contextMenuEvent()}\spxextra{GUIWidgets.PlaneWaveWidget method}}

\begin{fulllineitems}
\phantomsection\label{\detokenize{source/GUIWidgets:GUIWidgets.PlaneWaveWidget.contextMenuEvent}}
\pysigstartsignatures
\pysiglinewithargsret
{\sphinxbfcode{\sphinxupquote{contextMenuEvent}}}
{\sphinxparam{\DUrole{n}{self}}\sphinxparamcomma \sphinxparam{\DUrole{n}{a0}\DUrole{p}{:}\DUrole{w}{ }\DUrole{n}{QContextMenuEvent\DUrole{w}{ }\DUrole{p}{|}\DUrole{w}{ }None}}}
{}
\pysigstopsignatures
\end{fulllineitems}

\index{init\_ui() (GUIWidgets.PlaneWaveWidget method)@\spxentry{init\_ui()}\spxextra{GUIWidgets.PlaneWaveWidget method}}

\begin{fulllineitems}
\phantomsection\label{\detokenize{source/GUIWidgets:GUIWidgets.PlaneWaveWidget.init_ui}}
\pysigstartsignatures
\pysiglinewithargsret
{\sphinxbfcode{\sphinxupquote{init\_ui}}}
{\sphinxparam{\DUrole{n}{pw}}}
{}
\pysigstopsignatures
\end{fulllineitems}

\index{isDeleted (GUIWidgets.PlaneWaveWidget attribute)@\spxentry{isDeleted}\spxextra{GUIWidgets.PlaneWaveWidget attribute}}

\begin{fulllineitems}
\phantomsection\label{\detokenize{source/GUIWidgets:GUIWidgets.PlaneWaveWidget.isDeleted}}
\pysigstartsignatures
\pysigline
{\sphinxbfcode{\sphinxupquote{isDeleted}}}
\pysigstopsignatures
\sphinxAtStartPar
int = …, arguments: Sequence = …) \sphinxhyphen{}\textgreater{} PYQT\_SIGNAL

\sphinxAtStartPar
types is normally a sequence of individual types.  Each type is either a
type object or a string that is the name of a C++ type.  Alternatively
each type could itself be a sequence of types each describing a different
overloaded signal.
name is the optional C++ name of the signal.  If it is not specified then
the name of the class attribute that is bound to the signal is used.
revision is the optional revision of the signal that is exported to QML.
If it is not specified then 0 is used.
arguments is the optional sequence of the names of the signal’s arguments.
\begin{quote}\begin{description}
\sphinxlineitem{Type}
\sphinxAtStartPar
pyqtSignal({\color{red}\bfseries{}*}types, name

\sphinxlineitem{Type}
\sphinxAtStartPar
str = …, revision

\end{description}\end{quote}

\end{fulllineitems}

\index{isSet (GUIWidgets.PlaneWaveWidget attribute)@\spxentry{isSet}\spxextra{GUIWidgets.PlaneWaveWidget attribute}}

\begin{fulllineitems}
\phantomsection\label{\detokenize{source/GUIWidgets:GUIWidgets.PlaneWaveWidget.isSet}}
\pysigstartsignatures
\pysigline
{\sphinxbfcode{\sphinxupquote{isSet}}}
\pysigstopsignatures
\sphinxAtStartPar
int = …, arguments: Sequence = …) \sphinxhyphen{}\textgreater{} PYQT\_SIGNAL

\sphinxAtStartPar
types is normally a sequence of individual types.  Each type is either a
type object or a string that is the name of a C++ type.  Alternatively
each type could itself be a sequence of types each describing a different
overloaded signal.
name is the optional C++ name of the signal.  If it is not specified then
the name of the class attribute that is bound to the signal is used.
revision is the optional revision of the signal that is exported to QML.
If it is not specified then 0 is used.
arguments is the optional sequence of the names of the signal’s arguments.
\begin{quote}\begin{description}
\sphinxlineitem{Type}
\sphinxAtStartPar
pyqtSignal({\color{red}\bfseries{}*}types, name

\sphinxlineitem{Type}
\sphinxAtStartPar
str = …, revision

\end{description}\end{quote}

\end{fulllineitems}

\index{on\_receive\_info() (GUIWidgets.PlaneWaveWidget method)@\spxentry{on\_receive\_info()}\spxextra{GUIWidgets.PlaneWaveWidget method}}

\begin{fulllineitems}
\phantomsection\label{\detokenize{source/GUIWidgets:GUIWidgets.PlaneWaveWidget.on_receive_info}}
\pysigstartsignatures
\pysiglinewithargsret
{\sphinxbfcode{\sphinxupquote{on\_receive\_info}}}
{\sphinxparam{\DUrole{n}{info}}}
{}
\pysigstopsignatures
\end{fulllineitems}


\end{fulllineitems}

\index{PointSourceInitializationWidget (class in GUIWidgets)@\spxentry{PointSourceInitializationWidget}\spxextra{class in GUIWidgets}}

\begin{fulllineitems}
\phantomsection\label{\detokenize{source/GUIWidgets:GUIWidgets.PointSourceInitializationWidget}}
\pysigstartsignatures
\pysigline
{\sphinxbfcode{\sphinxupquote{class\DUrole{w}{ }}}\sphinxcode{\sphinxupquote{GUIWidgets.}}\sphinxbfcode{\sphinxupquote{PointSourceInitializationWidget}}}
\pysigstopsignatures
\sphinxAtStartPar
Bases: {\hyperref[\detokenize{source/GUIInitializationWidgets:GUIInitializationWidgets.InitializationWidget}]{\sphinxcrossref{\sphinxcode{\sphinxupquote{InitializationWidget}}}}}

\sphinxincludegraphics[]{inheritance-3d5f66bf5333cf55c85ea7abf05a15fae55611b5.pdf}
\index{non\_numbers (GUIWidgets.PointSourceInitializationWidget attribute)@\spxentry{non\_numbers}\spxextra{GUIWidgets.PointSourceInitializationWidget attribute}}

\begin{fulllineitems}
\phantomsection\label{\detokenize{source/GUIWidgets:GUIWidgets.PointSourceInitializationWidget.non_numbers}}
\pysigstartsignatures
\pysigline
{\sphinxbfcode{\sphinxupquote{non\_numbers}}\sphinxbfcode{\sphinxupquote{\DUrole{w}{ }\DUrole{p}{=}\DUrole{w}{ }{[}\textquotesingle{}\textquotesingle{}, \textquotesingle{}\sphinxhyphen{}\textquotesingle{}{]}}}}
\pysigstopsignatures
\end{fulllineitems}

\index{request\_data() (GUIWidgets.PointSourceInitializationWidget method)@\spxentry{request\_data()}\spxextra{GUIWidgets.PointSourceInitializationWidget method}}

\begin{fulllineitems}
\phantomsection\label{\detokenize{source/GUIWidgets:GUIWidgets.PointSourceInitializationWidget.request_data}}
\pysigstartsignatures
\pysiglinewithargsret
{\sphinxbfcode{\sphinxupquote{request\_data}}}
{}
{}
\pysigstopsignatures
\end{fulllineitems}

\index{sendBrightness (GUIWidgets.PointSourceInitializationWidget attribute)@\spxentry{sendBrightness}\spxextra{GUIWidgets.PointSourceInitializationWidget attribute}}

\begin{fulllineitems}
\phantomsection\label{\detokenize{source/GUIWidgets:GUIWidgets.PointSourceInitializationWidget.sendBrightness}}
\pysigstartsignatures
\pysigline
{\sphinxbfcode{\sphinxupquote{sendBrightness}}}
\pysigstopsignatures
\sphinxAtStartPar
int = …, arguments: Sequence = …) \sphinxhyphen{}\textgreater{} PYQT\_SIGNAL

\sphinxAtStartPar
types is normally a sequence of individual types.  Each type is either a
type object or a string that is the name of a C++ type.  Alternatively
each type could itself be a sequence of types each describing a different
overloaded signal.
name is the optional C++ name of the signal.  If it is not specified then
the name of the class attribute that is bound to the signal is used.
revision is the optional revision of the signal that is exported to QML.
If it is not specified then 0 is used.
arguments is the optional sequence of the names of the signal’s arguments.
\begin{quote}\begin{description}
\sphinxlineitem{Type}
\sphinxAtStartPar
pyqtSignal({\color{red}\bfseries{}*}types, name

\sphinxlineitem{Type}
\sphinxAtStartPar
str = …, revision

\end{description}\end{quote}

\end{fulllineitems}

\index{sendCoordinates (GUIWidgets.PointSourceInitializationWidget attribute)@\spxentry{sendCoordinates}\spxextra{GUIWidgets.PointSourceInitializationWidget attribute}}

\begin{fulllineitems}
\phantomsection\label{\detokenize{source/GUIWidgets:GUIWidgets.PointSourceInitializationWidget.sendCoordinates}}
\pysigstartsignatures
\pysigline
{\sphinxbfcode{\sphinxupquote{sendCoordinates}}}
\pysigstopsignatures
\sphinxAtStartPar
int = …, arguments: Sequence = …) \sphinxhyphen{}\textgreater{} PYQT\_SIGNAL

\sphinxAtStartPar
types is normally a sequence of individual types.  Each type is either a
type object or a string that is the name of a C++ type.  Alternatively
each type could itself be a sequence of types each describing a different
overloaded signal.
name is the optional C++ name of the signal.  If it is not specified then
the name of the class attribute that is bound to the signal is used.
revision is the optional revision of the signal that is exported to QML.
If it is not specified then 0 is used.
arguments is the optional sequence of the names of the signal’s arguments.
\begin{quote}\begin{description}
\sphinxlineitem{Type}
\sphinxAtStartPar
pyqtSignal({\color{red}\bfseries{}*}types, name

\sphinxlineitem{Type}
\sphinxAtStartPar
str = …, revision

\end{description}\end{quote}

\end{fulllineitems}

\index{send\_brightness() (GUIWidgets.PointSourceInitializationWidget method)@\spxentry{send\_brightness()}\spxextra{GUIWidgets.PointSourceInitializationWidget method}}

\begin{fulllineitems}
\phantomsection\label{\detokenize{source/GUIWidgets:GUIWidgets.PointSourceInitializationWidget.send_brightness}}
\pysigstartsignatures
\pysiglinewithargsret
{\sphinxbfcode{\sphinxupquote{send\_brightness}}}
{}
{}
\pysigstopsignatures
\end{fulllineitems}

\index{send\_coordinates() (GUIWidgets.PointSourceInitializationWidget method)@\spxentry{send\_coordinates()}\spxextra{GUIWidgets.PointSourceInitializationWidget method}}

\begin{fulllineitems}
\phantomsection\label{\detokenize{source/GUIWidgets:GUIWidgets.PointSourceInitializationWidget.send_coordinates}}
\pysigstartsignatures
\pysiglinewithargsret
{\sphinxbfcode{\sphinxupquote{send\_coordinates}}}
{}
{}
\pysigstopsignatures
\end{fulllineitems}


\end{fulllineitems}

\index{PointSourceWidget (class in GUIWidgets)@\spxentry{PointSourceWidget}\spxextra{class in GUIWidgets}}

\begin{fulllineitems}
\phantomsection\label{\detokenize{source/GUIWidgets:GUIWidgets.PointSourceWidget}}
\pysigstartsignatures
\pysigline
{\sphinxbfcode{\sphinxupquote{class\DUrole{w}{ }}}\sphinxcode{\sphinxupquote{GUIWidgets.}}\sphinxbfcode{\sphinxupquote{PointSourceWidget}}}
\pysigstopsignatures
\sphinxAtStartPar
Bases: {\hyperref[\detokenize{source/GUIWidgets:GUIWidgets.SourceWidget}]{\sphinxcrossref{\sphinxcode{\sphinxupquote{SourceWidget}}}}}

\sphinxincludegraphics[]{inheritance-4bc94e9e8f18c5056e55b204b526f4451f03682d.pdf}
\index{change\_widget() (GUIWidgets.PointSourceWidget method)@\spxentry{change\_widget()}\spxextra{GUIWidgets.PointSourceWidget method}}

\begin{fulllineitems}
\phantomsection\label{\detokenize{source/GUIWidgets:GUIWidgets.PointSourceWidget.change_widget}}
\pysigstartsignatures
\pysiglinewithargsret
{\sphinxbfcode{\sphinxupquote{change\_widget}}}
{}
{}
\pysigstopsignatures
\end{fulllineitems}

\index{contextMenuEvent() (GUIWidgets.PointSourceWidget method)@\spxentry{contextMenuEvent()}\spxextra{GUIWidgets.PointSourceWidget method}}

\begin{fulllineitems}
\phantomsection\label{\detokenize{source/GUIWidgets:GUIWidgets.PointSourceWidget.contextMenuEvent}}
\pysigstartsignatures
\pysiglinewithargsret
{\sphinxbfcode{\sphinxupquote{contextMenuEvent}}}
{\sphinxparam{\DUrole{n}{self}}\sphinxparamcomma \sphinxparam{\DUrole{n}{a0}\DUrole{p}{:}\DUrole{w}{ }\DUrole{n}{QContextMenuEvent\DUrole{w}{ }\DUrole{p}{|}\DUrole{w}{ }None}}}
{}
\pysigstopsignatures
\end{fulllineitems}

\index{init\_ui() (GUIWidgets.PointSourceWidget method)@\spxentry{init\_ui()}\spxextra{GUIWidgets.PointSourceWidget method}}

\begin{fulllineitems}
\phantomsection\label{\detokenize{source/GUIWidgets:GUIWidgets.PointSourceWidget.init_ui}}
\pysigstartsignatures
\pysiglinewithargsret
{\sphinxbfcode{\sphinxupquote{init\_ui}}}
{}
{}
\pysigstopsignatures
\end{fulllineitems}

\index{isSet (GUIWidgets.PointSourceWidget attribute)@\spxentry{isSet}\spxextra{GUIWidgets.PointSourceWidget attribute}}

\begin{fulllineitems}
\phantomsection\label{\detokenize{source/GUIWidgets:GUIWidgets.PointSourceWidget.isSet}}
\pysigstartsignatures
\pysigline
{\sphinxbfcode{\sphinxupquote{isSet}}}
\pysigstopsignatures
\sphinxAtStartPar
int = …, arguments: Sequence = …) \sphinxhyphen{}\textgreater{} PYQT\_SIGNAL

\sphinxAtStartPar
types is normally a sequence of individual types.  Each type is either a
type object or a string that is the name of a C++ type.  Alternatively
each type could itself be a sequence of types each describing a different
overloaded signal.
name is the optional C++ name of the signal.  If it is not specified then
the name of the class attribute that is bound to the signal is used.
revision is the optional revision of the signal that is exported to QML.
If it is not specified then 0 is used.
arguments is the optional sequence of the names of the signal’s arguments.
\begin{quote}\begin{description}
\sphinxlineitem{Type}
\sphinxAtStartPar
pyqtSignal({\color{red}\bfseries{}*}types, name

\sphinxlineitem{Type}
\sphinxAtStartPar
str = …, revision

\end{description}\end{quote}

\end{fulllineitems}

\index{on\_click\_ok() (GUIWidgets.PointSourceWidget method)@\spxentry{on\_click\_ok()}\spxextra{GUIWidgets.PointSourceWidget method}}

\begin{fulllineitems}
\phantomsection\label{\detokenize{source/GUIWidgets:GUIWidgets.PointSourceWidget.on_click_ok}}
\pysigstartsignatures
\pysiglinewithargsret
{\sphinxbfcode{\sphinxupquote{on\_click\_ok}}}
{}
{}
\pysigstopsignatures
\end{fulllineitems}

\index{on\_receive\_brightness() (GUIWidgets.PointSourceWidget method)@\spxentry{on\_receive\_brightness()}\spxextra{GUIWidgets.PointSourceWidget method}}

\begin{fulllineitems}
\phantomsection\label{\detokenize{source/GUIWidgets:GUIWidgets.PointSourceWidget.on_receive_brightness}}
\pysigstartsignatures
\pysiglinewithargsret
{\sphinxbfcode{\sphinxupquote{on\_receive\_brightness}}}
{\sphinxparam{\DUrole{n}{brightness}}}
{}
\pysigstopsignatures
\end{fulllineitems}

\index{on\_receive\_coordinates() (GUIWidgets.PointSourceWidget method)@\spxentry{on\_receive\_coordinates()}\spxextra{GUIWidgets.PointSourceWidget method}}

\begin{fulllineitems}
\phantomsection\label{\detokenize{source/GUIWidgets:GUIWidgets.PointSourceWidget.on_receive_coordinates}}
\pysigstartsignatures
\pysiglinewithargsret
{\sphinxbfcode{\sphinxupquote{on\_receive\_coordinates}}}
{\sphinxparam{\DUrole{n}{coordinates}}}
{}
\pysigstopsignatures
\end{fulllineitems}


\end{fulllineitems}

\index{SourceWidget (class in GUIWidgets)@\spxentry{SourceWidget}\spxextra{class in GUIWidgets}}

\begin{fulllineitems}
\phantomsection\label{\detokenize{source/GUIWidgets:GUIWidgets.SourceWidget}}
\pysigstartsignatures
\pysiglinewithargsret
{\sphinxbfcode{\sphinxupquote{class\DUrole{w}{ }}}\sphinxcode{\sphinxupquote{GUIWidgets.}}\sphinxbfcode{\sphinxupquote{SourceWidget}}}
{\sphinxparam{\DUrole{n}{source}}}
{}
\pysigstopsignatures
\sphinxAtStartPar
Bases: \sphinxcode{\sphinxupquote{QWidget}}

\sphinxincludegraphics[]{inheritance-3fd6fdfaeb08c33c9c5d2b3ff33ae7ebc3d7b9d4.pdf}
\index{change\_widget() (GUIWidgets.SourceWidget method)@\spxentry{change\_widget()}\spxextra{GUIWidgets.SourceWidget method}}

\begin{fulllineitems}
\phantomsection\label{\detokenize{source/GUIWidgets:GUIWidgets.SourceWidget.change_widget}}
\pysigstartsignatures
\pysiglinewithargsret
{\sphinxbfcode{\sphinxupquote{abstract\DUrole{w}{ }}}\sphinxbfcode{\sphinxupquote{change\_widget}}}
{}
{}
\pysigstopsignatures
\end{fulllineitems}

\index{contextMenuEvent() (GUIWidgets.SourceWidget method)@\spxentry{contextMenuEvent()}\spxextra{GUIWidgets.SourceWidget method}}

\begin{fulllineitems}
\phantomsection\label{\detokenize{source/GUIWidgets:GUIWidgets.SourceWidget.contextMenuEvent}}
\pysigstartsignatures
\pysiglinewithargsret
{\sphinxbfcode{\sphinxupquote{abstract\DUrole{w}{ }}}\sphinxbfcode{\sphinxupquote{contextMenuEvent}}}
{\sphinxparam{\DUrole{n}{self}}\sphinxparamcomma \sphinxparam{\DUrole{n}{a0}\DUrole{p}{:}\DUrole{w}{ }\DUrole{n}{QContextMenuEvent\DUrole{w}{ }\DUrole{p}{|}\DUrole{w}{ }None}}}
{}
\pysigstopsignatures
\end{fulllineitems}

\index{identifier (GUIWidgets.SourceWidget attribute)@\spxentry{identifier}\spxextra{GUIWidgets.SourceWidget attribute}}

\begin{fulllineitems}
\phantomsection\label{\detokenize{source/GUIWidgets:GUIWidgets.SourceWidget.identifier}}
\pysigstartsignatures
\pysigline
{\sphinxbfcode{\sphinxupquote{identifier}}\sphinxbfcode{\sphinxupquote{\DUrole{w}{ }\DUrole{p}{=}\DUrole{w}{ }0}}}
\pysigstopsignatures
\end{fulllineitems}

\index{init\_ui() (GUIWidgets.SourceWidget method)@\spxentry{init\_ui()}\spxextra{GUIWidgets.SourceWidget method}}

\begin{fulllineitems}
\phantomsection\label{\detokenize{source/GUIWidgets:GUIWidgets.SourceWidget.init_ui}}
\pysigstartsignatures
\pysiglinewithargsret
{\sphinxbfcode{\sphinxupquote{abstract\DUrole{w}{ }}}\sphinxbfcode{\sphinxupquote{init\_ui}}}
{\sphinxparam{\DUrole{n}{source}}}
{}
\pysigstopsignatures
\end{fulllineitems}

\index{isDeleted (GUIWidgets.SourceWidget attribute)@\spxentry{isDeleted}\spxextra{GUIWidgets.SourceWidget attribute}}

\begin{fulllineitems}
\phantomsection\label{\detokenize{source/GUIWidgets:GUIWidgets.SourceWidget.isDeleted}}
\pysigstartsignatures
\pysigline
{\sphinxbfcode{\sphinxupquote{isDeleted}}}
\pysigstopsignatures
\sphinxAtStartPar
int = …, arguments: Sequence = …) \sphinxhyphen{}\textgreater{} PYQT\_SIGNAL

\sphinxAtStartPar
types is normally a sequence of individual types.  Each type is either a
type object or a string that is the name of a C++ type.  Alternatively
each type could itself be a sequence of types each describing a different
overloaded signal.
name is the optional C++ name of the signal.  If it is not specified then
the name of the class attribute that is bound to the signal is used.
revision is the optional revision of the signal that is exported to QML.
If it is not specified then 0 is used.
arguments is the optional sequence of the names of the signal’s arguments.
\begin{quote}\begin{description}
\sphinxlineitem{Type}
\sphinxAtStartPar
pyqtSignal({\color{red}\bfseries{}*}types, name

\sphinxlineitem{Type}
\sphinxAtStartPar
str = …, revision

\end{description}\end{quote}

\end{fulllineitems}

\index{remove\_widget() (GUIWidgets.SourceWidget method)@\spxentry{remove\_widget()}\spxextra{GUIWidgets.SourceWidget method}}

\begin{fulllineitems}
\phantomsection\label{\detokenize{source/GUIWidgets:GUIWidgets.SourceWidget.remove_widget}}
\pysigstartsignatures
\pysiglinewithargsret
{\sphinxbfcode{\sphinxupquote{remove\_widget}}}
{}
{}
\pysigstopsignatures
\end{fulllineitems}


\end{fulllineitems}


\sphinxstepscope


\subsection{Illumination module}
\label{\detokenize{source/Illumination:module-Illumination}}\label{\detokenize{source/Illumination:illumination-module}}\label{\detokenize{source/Illumination::doc}}\index{module@\spxentry{module}!Illumination@\spxentry{Illumination}}\index{Illumination@\spxentry{Illumination}!module@\spxentry{module}}
\sphinxAtStartPar
Illumination.py

\sphinxAtStartPar
This module contains the Illumination class, which handles the simulation and analysis of illumination patterns in optical systems.
\begin{description}
\sphinxlineitem{Classes:}
\sphinxAtStartPar
Illumination: Manages the properties and behavior of illumination patterns, including wavevectors and spatial shifts.

\end{description}

\sphinxincludegraphics[]{inheritance-3001b81258d3eaafe9cd8200fa95c28de917c3f9.pdf}
\index{Illumination (class in Illumination)@\spxentry{Illumination}\spxextra{class in Illumination}}

\begin{fulllineitems}
\phantomsection\label{\detokenize{source/Illumination:Illumination.Illumination}}
\pysigstartsignatures
\pysiglinewithargsret
{\sphinxbfcode{\sphinxupquote{class\DUrole{w}{ }}}\sphinxcode{\sphinxupquote{Illumination.}}\sphinxbfcode{\sphinxupquote{Illumination}}}
{\sphinxparam{\DUrole{n}{intensity\_plane\_waves\_dict}\DUrole{p}{:}\DUrole{w}{ }\DUrole{n}{dict\DUrole{p}{{[}}tuple\DUrole{p}{{[}}int\DUrole{p}{,}\DUrole{w}{ }int\DUrole{p}{,}\DUrole{w}{ }int\DUrole{p}{{]}}\DUrole{p}{,}\DUrole{w}{ }{\hyperref[\detokenize{source/Sources:Sources.IntensityPlaneWave}]{\sphinxcrossref{IntensityPlaneWave}}}\DUrole{p}{{]}}}}\sphinxparamcomma \sphinxparam{\DUrole{n}{Mr}\DUrole{o}{=}\DUrole{default_value}{1}}}
{}
\pysigstopsignatures
\sphinxAtStartPar
Bases: \sphinxcode{\sphinxupquote{object}}

\sphinxAtStartPar
Manages the properties and behavior of illumination patterns, including wavevectors and spatial shifts.
\index{angles (Illumination.Illumination attribute)@\spxentry{angles}\spxextra{Illumination.Illumination attribute}}

\begin{fulllineitems}
\phantomsection\label{\detokenize{source/Illumination:Illumination.Illumination.angles}}
\pysigstartsignatures
\pysigline
{\sphinxbfcode{\sphinxupquote{angles}}}
\pysigstopsignatures
\sphinxAtStartPar
Array of rotation angles.
\begin{quote}\begin{description}
\sphinxlineitem{Type}
\sphinxAtStartPar
np.ndarray

\end{description}\end{quote}

\end{fulllineitems}

\index{\_spatial\_shifts (Illumination.Illumination attribute)@\spxentry{\_spatial\_shifts}\spxextra{Illumination.Illumination attribute}}

\begin{fulllineitems}
\phantomsection\label{\detokenize{source/Illumination:Illumination.Illumination._spatial_shifts}}
\pysigstartsignatures
\pysigline
{\sphinxbfcode{\sphinxupquote{\_spatial\_shifts}}}
\pysigstopsignatures
\sphinxAtStartPar
List of spatial shifts.
\begin{quote}\begin{description}
\sphinxlineitem{Type}
\sphinxAtStartPar
list

\end{description}\end{quote}

\end{fulllineitems}

\index{\_Mr (Illumination.Illumination attribute)@\spxentry{\_Mr}\spxextra{Illumination.Illumination attribute}}

\begin{fulllineitems}
\phantomsection\label{\detokenize{source/Illumination:Illumination.Illumination._Mr}}
\pysigstartsignatures
\pysigline
{\sphinxbfcode{\sphinxupquote{\_Mr}}}
\pysigstopsignatures
\sphinxAtStartPar
Number of rotations.
\begin{quote}\begin{description}
\sphinxlineitem{Type}
\sphinxAtStartPar
int

\end{description}\end{quote}

\end{fulllineitems}

\index{Mt (Illumination.Illumination attribute)@\spxentry{Mt}\spxextra{Illumination.Illumination attribute}}

\begin{fulllineitems}
\phantomsection\label{\detokenize{source/Illumination:Illumination.Illumination.Mt}}
\pysigstartsignatures
\pysigline
{\sphinxbfcode{\sphinxupquote{Mt}}}
\pysigstopsignatures
\sphinxAtStartPar
Number of spatial shifts.
\begin{quote}\begin{description}
\sphinxlineitem{Type}
\sphinxAtStartPar
int

\end{description}\end{quote}

\end{fulllineitems}

\index{waves (Illumination.Illumination attribute)@\spxentry{waves}\spxextra{Illumination.Illumination attribute}}

\begin{fulllineitems}
\phantomsection\label{\detokenize{source/Illumination:Illumination.Illumination.waves}}
\pysigstartsignatures
\pysigline
{\sphinxbfcode{\sphinxupquote{waves}}}
\pysigstopsignatures
\sphinxAtStartPar
Dictionary of intensity plane waves.
\begin{quote}\begin{description}
\sphinxlineitem{Type}
\sphinxAtStartPar
dict

\end{description}\end{quote}

\end{fulllineitems}

\index{wavevectors2d (Illumination.Illumination attribute)@\spxentry{wavevectors2d}\spxextra{Illumination.Illumination attribute}}

\begin{fulllineitems}
\phantomsection\label{\detokenize{source/Illumination:Illumination.Illumination.wavevectors2d}}
\pysigstartsignatures
\pysigline
{\sphinxbfcode{\sphinxupquote{wavevectors2d}}}
\pysigstopsignatures
\sphinxAtStartPar
List of 2D wavevectors.
\begin{quote}\begin{description}
\sphinxlineitem{Type}
\sphinxAtStartPar
list

\end{description}\end{quote}

\end{fulllineitems}

\index{indices2d (Illumination.Illumination attribute)@\spxentry{indices2d}\spxextra{Illumination.Illumination attribute}}

\begin{fulllineitems}
\phantomsection\label{\detokenize{source/Illumination:Illumination.Illumination.indices2d}}
\pysigstartsignatures
\pysigline
{\sphinxbfcode{\sphinxupquote{indices2d}}}
\pysigstopsignatures
\sphinxAtStartPar
List of 2D indices.
\begin{quote}\begin{description}
\sphinxlineitem{Type}
\sphinxAtStartPar
list

\end{description}\end{quote}

\end{fulllineitems}

\index{wavevectors3d (Illumination.Illumination attribute)@\spxentry{wavevectors3d}\spxextra{Illumination.Illumination attribute}}

\begin{fulllineitems}
\phantomsection\label{\detokenize{source/Illumination:Illumination.Illumination.wavevectors3d}}
\pysigstartsignatures
\pysigline
{\sphinxbfcode{\sphinxupquote{wavevectors3d}}}
\pysigstopsignatures
\sphinxAtStartPar
List of 3D wavevectors.
\begin{quote}\begin{description}
\sphinxlineitem{Type}
\sphinxAtStartPar
list

\end{description}\end{quote}

\end{fulllineitems}

\index{indices3d (Illumination.Illumination attribute)@\spxentry{indices3d}\spxextra{Illumination.Illumination attribute}}

\begin{fulllineitems}
\phantomsection\label{\detokenize{source/Illumination:Illumination.Illumination.indices3d}}
\pysigstartsignatures
\pysigline
{\sphinxbfcode{\sphinxupquote{indices3d}}}
\pysigstopsignatures
\sphinxAtStartPar
List of 3D indices.
\begin{quote}\begin{description}
\sphinxlineitem{Type}
\sphinxAtStartPar
list

\end{description}\end{quote}

\end{fulllineitems}

\index{rearranged\_indices (Illumination.Illumination attribute)@\spxentry{rearranged\_indices}\spxextra{Illumination.Illumination attribute}}

\begin{fulllineitems}
\phantomsection\label{\detokenize{source/Illumination:Illumination.Illumination.rearranged_indices}}
\pysigstartsignatures
\pysigline
{\sphinxbfcode{\sphinxupquote{rearranged\_indices}}}
\pysigstopsignatures
\sphinxAtStartPar
Dictionary of rearranged indices.
\begin{quote}\begin{description}
\sphinxlineitem{Type}
\sphinxAtStartPar
dict

\end{description}\end{quote}

\end{fulllineitems}

\index{xy\_fourier\_peaks (Illumination.Illumination attribute)@\spxentry{xy\_fourier\_peaks}\spxextra{Illumination.Illumination attribute}}

\begin{fulllineitems}
\phantomsection\label{\detokenize{source/Illumination:Illumination.Illumination.xy_fourier_peaks}}
\pysigstartsignatures
\pysigline
{\sphinxbfcode{\sphinxupquote{xy\_fourier\_peaks}}}
\pysigstopsignatures
\sphinxAtStartPar
Set of 2D Fourier peaks.
\begin{quote}\begin{description}
\sphinxlineitem{Type}
\sphinxAtStartPar
set

\end{description}\end{quote}

\end{fulllineitems}

\index{phase\_matrix (Illumination.Illumination attribute)@\spxentry{phase\_matrix}\spxextra{Illumination.Illumination attribute}}

\begin{fulllineitems}
\phantomsection\label{\detokenize{source/Illumination:Illumination.Illumination.phase_matrix}}
\pysigstartsignatures
\pysigline
{\sphinxbfcode{\sphinxupquote{phase\_matrix}}}
\pysigstopsignatures
\sphinxAtStartPar
Dictionary of all phase the relevant phase shifts.
\begin{quote}\begin{description}
\sphinxlineitem{Type}
\sphinxAtStartPar
dict

\end{description}\end{quote}

\end{fulllineitems}


\sphinxincludegraphics[]{inheritance-70b8c9dae226e61be78c1c705e49f39b360dfb75.pdf}
\index{Mr (Illumination.Illumination property)@\spxentry{Mr}\spxextra{Illumination.Illumination property}}

\begin{fulllineitems}
\phantomsection\label{\detokenize{source/Illumination:Illumination.Illumination.Mr}}
\pysigstartsignatures
\pysigline
{\sphinxbfcode{\sphinxupquote{property\DUrole{w}{ }}}\sphinxbfcode{\sphinxupquote{Mr}}}
\pysigstopsignatures
\end{fulllineitems}

\index{compute\_expanded\_lattice2d() (Illumination.Illumination method)@\spxentry{compute\_expanded\_lattice2d()}\spxextra{Illumination.Illumination method}}

\begin{fulllineitems}
\phantomsection\label{\detokenize{source/Illumination:Illumination.Illumination.compute_expanded_lattice2d}}
\pysigstartsignatures
\pysiglinewithargsret
{\sphinxbfcode{\sphinxupquote{compute\_expanded\_lattice2d}}}
{}
{{ $\rightarrow$ set\DUrole{p}{{[}}tuple\DUrole{p}{{[}}int\DUrole{p}{,}\DUrole{w}{ }int\DUrole{p}{{]}}\DUrole{p}{{]}}}}
\pysigstopsignatures\begin{description}
\sphinxlineitem{Compute the expanded 2D lattice of Fourier peaks}
\sphinxAtStartPar
(autoconvoluiton of Fourier transform of the illumination pattern).

\end{description}
\begin{quote}\begin{description}
\sphinxlineitem{Returns}
\sphinxAtStartPar
Set of expanded 2D lattice peaks.

\sphinxlineitem{Return type}
\sphinxAtStartPar
set

\end{description}\end{quote}

\end{fulllineitems}

\index{compute\_expanded\_lattice3d() (Illumination.Illumination method)@\spxentry{compute\_expanded\_lattice3d()}\spxextra{Illumination.Illumination method}}

\begin{fulllineitems}
\phantomsection\label{\detokenize{source/Illumination:Illumination.Illumination.compute_expanded_lattice3d}}
\pysigstartsignatures
\pysiglinewithargsret
{\sphinxbfcode{\sphinxupquote{compute\_expanded\_lattice3d}}}
{}
{{ $\rightarrow$ set\DUrole{p}{{[}}tuple\DUrole{p}{{[}}int\DUrole{p}{,}\DUrole{w}{ }int\DUrole{p}{,}\DUrole{w}{ }int\DUrole{p}{{]}}\DUrole{p}{{]}}}}
\pysigstopsignatures\begin{description}
\sphinxlineitem{Compute the expanded 3D lattice of Fourier peaks}
\sphinxAtStartPar
(autoconvoluiton of Fourier transform of the illumination pattern).

\end{description}
\begin{quote}\begin{description}
\sphinxlineitem{Returns}
\sphinxAtStartPar
Set of expanded 3D lattice peaks.

\sphinxlineitem{Return type}
\sphinxAtStartPar
set

\end{description}\end{quote}

\end{fulllineitems}

\index{compute\_phase\_matrix() (Illumination.Illumination method)@\spxentry{compute\_phase\_matrix()}\spxextra{Illumination.Illumination method}}

\begin{fulllineitems}
\phantomsection\label{\detokenize{source/Illumination:Illumination.Illumination.compute_phase_matrix}}
\pysigstartsignatures
\pysiglinewithargsret
{\sphinxbfcode{\sphinxupquote{compute\_phase\_matrix}}}
{}
{}
\pysigstopsignatures\begin{description}
\sphinxlineitem{Compute the dictionary of all the relevant phase shifts}
\sphinxAtStartPar
(products of spatial shifts and illumination pattern spatial frequencies).

\end{description}

\end{fulllineitems}

\index{find\_ipw\_from\_pw() (Illumination.Illumination static method)@\spxentry{find\_ipw\_from\_pw()}\spxextra{Illumination.Illumination static method}}

\begin{fulllineitems}
\phantomsection\label{\detokenize{source/Illumination:Illumination.Illumination.find_ipw_from_pw}}
\pysigstartsignatures
\pysiglinewithargsret
{\sphinxbfcode{\sphinxupquote{static\DUrole{w}{ }}}\sphinxbfcode{\sphinxupquote{find\_ipw\_from\_pw}}}
{\sphinxparam{\DUrole{n}{plane\_waves}}}
{{ $\rightarrow$ list\DUrole{p}{{[}}{\hyperref[\detokenize{source/Sources:Sources.IntensityPlaneWave}]{\sphinxcrossref{IntensityPlaneWave}}}\DUrole{p}{{]}}}}
\pysigstopsignatures\begin{description}
\sphinxlineitem{Static method to find intensity plane waves}
\sphinxAtStartPar
(i.e. Fourier transform of the illumination pattern) from plane waves.

\end{description}
\begin{quote}\begin{description}
\sphinxlineitem{Parameters}
\sphinxAtStartPar
\sphinxstyleliteralstrong{\sphinxupquote{plane\_waves}} (\sphinxstyleliteralemphasis{\sphinxupquote{list}}) \textendash{} List of plane waves.

\sphinxlineitem{Returns}
\sphinxAtStartPar
List of intensity plane waves.

\sphinxlineitem{Return type}
\sphinxAtStartPar
list

\end{description}\end{quote}

\end{fulllineitems}

\index{get\_all\_wavevectors() (Illumination.Illumination method)@\spxentry{get\_all\_wavevectors()}\spxextra{Illumination.Illumination method}}

\begin{fulllineitems}
\phantomsection\label{\detokenize{source/Illumination:Illumination.Illumination.get_all_wavevectors}}
\pysigstartsignatures
\pysiglinewithargsret
{\sphinxbfcode{\sphinxupquote{get\_all\_wavevectors}}}
{}
{{ $\rightarrow$ list\DUrole{p}{{[}}ndarray\DUrole{p}{{]}}}}
\pysigstopsignatures
\sphinxAtStartPar
Get all wavevectors for all rotations.
\begin{quote}\begin{description}
\sphinxlineitem{Returns}
\sphinxAtStartPar
List of all wavevectors.

\sphinxlineitem{Return type}
\sphinxAtStartPar
list

\end{description}\end{quote}

\end{fulllineitems}

\index{get\_all\_wavevectors\_projected() (Illumination.Illumination method)@\spxentry{get\_all\_wavevectors\_projected()}\spxextra{Illumination.Illumination method}}

\begin{fulllineitems}
\phantomsection\label{\detokenize{source/Illumination:Illumination.Illumination.get_all_wavevectors_projected}}
\pysigstartsignatures
\pysiglinewithargsret
{\sphinxbfcode{\sphinxupquote{get\_all\_wavevectors\_projected}}}
{}
{}
\pysigstopsignatures
\sphinxAtStartPar
Get all projected wavevectors for all rotations.
\begin{quote}\begin{description}
\sphinxlineitem{Returns}
\sphinxAtStartPar
List of all projected wavevectors.

\sphinxlineitem{Return type}
\sphinxAtStartPar
list

\end{description}\end{quote}

\end{fulllineitems}

\index{get\_wavevectors() (Illumination.Illumination method)@\spxentry{get\_wavevectors()}\spxextra{Illumination.Illumination method}}

\begin{fulllineitems}
\phantomsection\label{\detokenize{source/Illumination:Illumination.Illumination.get_wavevectors}}
\pysigstartsignatures
\pysiglinewithargsret
{\sphinxbfcode{\sphinxupquote{get\_wavevectors}}}
{\sphinxparam{\DUrole{n}{r}\DUrole{p}{:}\DUrole{w}{ }\DUrole{n}{int}}}
{{ $\rightarrow$ tuple\DUrole{p}{{[}}list\DUrole{p}{{[}}ndarray\DUrole{p}{{]}}\DUrole{p}{,}\DUrole{w}{ }list\DUrole{p}{{[}}tuple\DUrole{p}{{[}}int\DUrole{p}{{]}}\DUrole{p}{{]}}\DUrole{p}{{]}}}}
\pysigstopsignatures
\sphinxAtStartPar
Get the wavevectors and indices for a given rotation.
\begin{quote}\begin{description}
\sphinxlineitem{Parameters}
\sphinxAtStartPar
\sphinxstyleliteralstrong{\sphinxupquote{r}} (\sphinxstyleliteralemphasis{\sphinxupquote{int}}) \textendash{} Rotation index.

\sphinxlineitem{Returns}
\sphinxAtStartPar
List of wavevectors and list of indices.

\sphinxlineitem{Return type}
\sphinxAtStartPar
tuple

\end{description}\end{quote}

\end{fulllineitems}

\index{get\_wavevectors\_projected() (Illumination.Illumination method)@\spxentry{get\_wavevectors\_projected()}\spxextra{Illumination.Illumination method}}

\begin{fulllineitems}
\phantomsection\label{\detokenize{source/Illumination:Illumination.Illumination.get_wavevectors_projected}}
\pysigstartsignatures
\pysiglinewithargsret
{\sphinxbfcode{\sphinxupquote{get\_wavevectors\_projected}}}
{\sphinxparam{\DUrole{n}{r}\DUrole{p}{:}\DUrole{w}{ }\DUrole{n}{int}}}
{{ $\rightarrow$ tuple\DUrole{p}{{[}}list\DUrole{p}{{[}}ndarray\DUrole{p}{{]}}\DUrole{p}{,}\DUrole{w}{ }list\DUrole{p}{{[}}tuple\DUrole{p}{{[}}int\DUrole{p}{{]}}\DUrole{p}{{]}}\DUrole{p}{{]}}}}
\pysigstopsignatures
\sphinxAtStartPar
Get the projected wavevectors and indices for a given rotation.
\begin{quote}\begin{description}
\sphinxlineitem{Parameters}
\sphinxAtStartPar
\sphinxstyleliteralstrong{\sphinxupquote{r}} (\sphinxstyleliteralemphasis{\sphinxupquote{int}}) \textendash{} Rotation index.

\sphinxlineitem{Returns}
\sphinxAtStartPar
List of projected wavevectors and list of indices.

\sphinxlineitem{Return type}
\sphinxAtStartPar
tuple

\end{description}\end{quote}

\end{fulllineitems}

\index{index\_frequencies() (Illumination.Illumination static method)@\spxentry{index\_frequencies()}\spxextra{Illumination.Illumination static method}}

\begin{fulllineitems}
\phantomsection\label{\detokenize{source/Illumination:Illumination.Illumination.index_frequencies}}
\pysigstartsignatures
\pysiglinewithargsret
{\sphinxbfcode{\sphinxupquote{static\DUrole{w}{ }}}\sphinxbfcode{\sphinxupquote{index\_frequencies}}}
{\sphinxparam{\DUrole{n}{waves\_list}\DUrole{p}{:}\DUrole{w}{ }\DUrole{n}{list\DUrole{p}{{[}}{\hyperref[\detokenize{source/Sources:Sources.IntensityPlaneWave}]{\sphinxcrossref{IntensityPlaneWave}}}\DUrole{p}{{]}}}}\sphinxparamcomma \sphinxparam{\DUrole{n}{base\_vector\_lengths}\DUrole{p}{:}\DUrole{w}{ }\DUrole{n}{tuple\DUrole{p}{{[}}float\DUrole{p}{,}\DUrole{w}{ }float\DUrole{p}{,}\DUrole{w}{ }float\DUrole{p}{{]}}}}}
{{ $\rightarrow$ dict\DUrole{p}{{[}}tuple\DUrole{p}{{[}}int\DUrole{p}{,}\DUrole{w}{ }int\DUrole{p}{,}\DUrole{w}{ }int\DUrole{p}{{]}}\DUrole{p}{,}\DUrole{w}{ }{\hyperref[\detokenize{source/Sources:Sources.IntensityPlaneWave}]{\sphinxcrossref{IntensityPlaneWave}}}\DUrole{p}{{]}}}}
\pysigstopsignatures
\sphinxAtStartPar
Static method to automatically index intensity plane waves given corresponding base vector lengths.
\begin{quote}\begin{description}
\sphinxlineitem{Parameters}\begin{itemize}
\item {} 
\sphinxAtStartPar
\sphinxstyleliteralstrong{\sphinxupquote{waves\_list}} (\sphinxstyleliteralemphasis{\sphinxupquote{list}}) \textendash{} List of plane waves.

\item {} 
\sphinxAtStartPar
\sphinxstyleliteralstrong{\sphinxupquote{base\_vector\_lengths}} (\sphinxstyleliteralemphasis{\sphinxupquote{tuple}}) \textendash{} Base vector lengths.

\end{itemize}

\sphinxlineitem{Returns}
\sphinxAtStartPar
Dictionary of indexed frequencies.

\sphinxlineitem{Return type}
\sphinxAtStartPar
dict

\end{description}\end{quote}

\end{fulllineitems}

\index{init\_from\_list() (Illumination.Illumination class method)@\spxentry{init\_from\_list()}\spxextra{Illumination.Illumination class method}}

\begin{fulllineitems}
\phantomsection\label{\detokenize{source/Illumination:Illumination.Illumination.init_from_list}}
\pysigstartsignatures
\pysiglinewithargsret
{\sphinxbfcode{\sphinxupquote{classmethod\DUrole{w}{ }}}\sphinxbfcode{\sphinxupquote{init\_from\_list}}}
{\sphinxparam{\DUrole{n}{intensity\_plane\_waves\_list}\DUrole{p}{:}\DUrole{w}{ }\DUrole{n}{dict\DUrole{p}{{[}}tuple\DUrole{p}{{[}}int\DUrole{p}{,}\DUrole{w}{ }int\DUrole{p}{,}\DUrole{w}{ }int\DUrole{p}{{]}}\DUrole{p}{,}\DUrole{w}{ }{\hyperref[\detokenize{source/Sources:Sources.IntensityPlaneWave}]{\sphinxcrossref{IntensityPlaneWave}}}\DUrole{p}{{]}}}}\sphinxparamcomma \sphinxparam{\DUrole{n}{base\_vector\_lengths}\DUrole{p}{:}\DUrole{w}{ }\DUrole{n}{tuple\DUrole{p}{{[}}float\DUrole{p}{,}\DUrole{w}{ }float\DUrole{p}{,}\DUrole{w}{ }float\DUrole{p}{{]}}}}\sphinxparamcomma \sphinxparam{\DUrole{n}{Mr}\DUrole{o}{=}\DUrole{default_value}{1}}}
{}
\pysigstopsignatures
\sphinxAtStartPar
Class method to initialize Illumination from a list of intensity plane waves.
\begin{quote}\begin{description}
\sphinxlineitem{Parameters}\begin{itemize}
\item {} 
\sphinxAtStartPar
\sphinxstyleliteralstrong{\sphinxupquote{intensity\_plane\_waves\_list}} (\sphinxstyleliteralemphasis{\sphinxupquote{list}}) \textendash{} List of intensity plane waves.

\item {} 
\sphinxAtStartPar
\sphinxstyleliteralstrong{\sphinxupquote{base\_vector\_lengths}} (\sphinxstyleliteralemphasis{\sphinxupquote{tuple}}) \textendash{} Base vector lengths of the illumination Fourier space Bravais lattice.

\item {} 
\sphinxAtStartPar
\sphinxstyleliteralstrong{\sphinxupquote{Mr}} (\sphinxstyleliteralemphasis{\sphinxupquote{int}}) \textendash{} Number of rotations.

\end{itemize}

\sphinxlineitem{Returns}
\sphinxAtStartPar
Initialized Illumination object.

\sphinxlineitem{Return type}
\sphinxAtStartPar
{\hyperref[\detokenize{source/Illumination:Illumination.Illumination}]{\sphinxcrossref{Illumination}}}

\end{description}\end{quote}

\end{fulllineitems}

\index{normalize\_spatial\_waves() (Illumination.Illumination method)@\spxentry{normalize\_spatial\_waves()}\spxextra{Illumination.Illumination method}}

\begin{fulllineitems}
\phantomsection\label{\detokenize{source/Illumination:Illumination.Illumination.normalize_spatial_waves}}
\pysigstartsignatures
\pysiglinewithargsret
{\sphinxbfcode{\sphinxupquote{normalize\_spatial\_waves}}}
{}
{}
\pysigstopsignatures
\sphinxAtStartPar
Normalize the spatial waves on zero peak (i.e., a0 = 1).
\begin{quote}\begin{description}
\sphinxlineitem{Raises}
\sphinxAtStartPar
\sphinxstyleliteralstrong{\sphinxupquote{AttributeError}} \textendash{} If zero wavevector is not found.

\end{description}\end{quote}

\end{fulllineitems}

\index{set\_spatial\_shifts\_diagonally() (Illumination.Illumination method)@\spxentry{set\_spatial\_shifts\_diagonally()}\spxextra{Illumination.Illumination method}}

\begin{fulllineitems}
\phantomsection\label{\detokenize{source/Illumination:Illumination.Illumination.set_spatial_shifts_diagonally}}
\pysigstartsignatures
\pysiglinewithargsret
{\sphinxbfcode{\sphinxupquote{set\_spatial\_shifts\_diagonally}}}
{\sphinxparam{\DUrole{n}{number}\DUrole{p}{:}\DUrole{w}{ }\DUrole{n}{int}}\sphinxparamcomma \sphinxparam{\DUrole{n}{base\_vectors}\DUrole{p}{:}\DUrole{w}{ }\DUrole{n}{tuple\DUrole{p}{{[}}float\DUrole{p}{,}\DUrole{w}{ }float\DUrole{p}{,}\DUrole{w}{ }float\DUrole{p}{{]}}}}}
{}
\pysigstopsignatures
\sphinxAtStartPar
Set the spatial shifts diagonally (i.e., all the spatial shifts are assumed to be on the same lin).
This is the most common use in practice.
Appropriate shifts for a given illumination pattern can be computed in the module ‘compute\_optimal\_lattices.py’
\begin{quote}\begin{description}
\sphinxlineitem{Parameters}\begin{itemize}
\item {} 
\sphinxAtStartPar
\sphinxstyleliteralstrong{\sphinxupquote{number}} (\sphinxstyleliteralemphasis{\sphinxupquote{int}}) \textendash{} Number of shifts.

\item {} 
\sphinxAtStartPar
\sphinxstyleliteralstrong{\sphinxupquote{base\_vectors}} (\sphinxstyleliteralemphasis{\sphinxupquote{tuple}}) \textendash{} Base vectors for the shifts.

\end{itemize}

\end{description}\end{quote}

\end{fulllineitems}

\index{spatial\_shifts (Illumination.Illumination property)@\spxentry{spatial\_shifts}\spxextra{Illumination.Illumination property}}

\begin{fulllineitems}
\phantomsection\label{\detokenize{source/Illumination:Illumination.Illumination.spatial_shifts}}
\pysigstartsignatures
\pysigline
{\sphinxbfcode{\sphinxupquote{property\DUrole{w}{ }}}\sphinxbfcode{\sphinxupquote{spatial\_shifts}}}
\pysigstopsignatures
\end{fulllineitems}


\end{fulllineitems}


\sphinxstepscope


\subsection{OpticalSystems module}
\label{\detokenize{source/OpticalSystems:module-OpticalSystems}}\label{\detokenize{source/OpticalSystems:opticalsystems-module}}\label{\detokenize{source/OpticalSystems::doc}}\index{module@\spxentry{module}!OpticalSystems@\spxentry{OpticalSystems}}\index{OpticalSystems@\spxentry{OpticalSystems}!module@\spxentry{module}}
\sphinxAtStartPar
OpticalSystems.py

\sphinxAtStartPar
This module contains classes for simulating and analyzing optical systems.

\sphinxAtStartPar
Note: More reasonable interface for accessing and calculating of the PSF and OTF is expected in the future.
For this reason the detailed documentation on the computations is not provided yet.

\sphinxincludegraphics[]{inheritance-20b13fd43ba62aaa01b995203466baeb923f2606.pdf}
\index{OpticalSystem (class in OpticalSystems)@\spxentry{OpticalSystem}\spxextra{class in OpticalSystems}}

\begin{fulllineitems}
\phantomsection\label{\detokenize{source/OpticalSystems:OpticalSystems.OpticalSystem}}
\pysigstartsignatures
\pysiglinewithargsret
{\sphinxbfcode{\sphinxupquote{class\DUrole{w}{ }}}\sphinxcode{\sphinxupquote{OpticalSystems.}}\sphinxbfcode{\sphinxupquote{OpticalSystem}}}
{\sphinxparam{\DUrole{n}{interpolation\_method}\DUrole{p}{:}\DUrole{w}{ }\DUrole{n}{str}}}
{}
\pysigstopsignatures
\sphinxAtStartPar
Bases: \sphinxcode{\sphinxupquote{object}}

\sphinxAtStartPar
Base class for optical systems, providing common functionality.
\index{supported\_interpolation\_methods (OpticalSystems.OpticalSystem attribute)@\spxentry{supported\_interpolation\_methods}\spxextra{OpticalSystems.OpticalSystem attribute}}

\begin{fulllineitems}
\phantomsection\label{\detokenize{source/OpticalSystems:OpticalSystems.OpticalSystem.supported_interpolation_methods}}
\pysigstartsignatures
\pysigline
{\sphinxbfcode{\sphinxupquote{supported\_interpolation\_methods}}}
\pysigstopsignatures
\sphinxAtStartPar
List of supported interpolation methods.
\begin{quote}\begin{description}
\sphinxlineitem{Type}
\sphinxAtStartPar
list

\end{description}\end{quote}

\end{fulllineitems}

\index{psf (OpticalSystems.OpticalSystem attribute)@\spxentry{psf}\spxextra{OpticalSystems.OpticalSystem attribute}}

\begin{fulllineitems}
\phantomsection\label{\detokenize{source/OpticalSystems:OpticalSystems.OpticalSystem.psf}}
\pysigstartsignatures
\pysigline
{\sphinxbfcode{\sphinxupquote{psf}}}
\pysigstopsignatures
\sphinxAtStartPar
Point Spread Function.
\begin{quote}\begin{description}
\sphinxlineitem{Type}
\sphinxAtStartPar
np.ndarray

\end{description}\end{quote}

\end{fulllineitems}

\index{otf (OpticalSystems.OpticalSystem attribute)@\spxentry{otf}\spxextra{OpticalSystems.OpticalSystem attribute}}

\begin{fulllineitems}
\phantomsection\label{\detokenize{source/OpticalSystems:OpticalSystems.OpticalSystem.otf}}
\pysigstartsignatures
\pysigline
{\sphinxbfcode{\sphinxupquote{otf}}}
\pysigstopsignatures
\sphinxAtStartPar
Optical Transfer Function.
\begin{quote}\begin{description}
\sphinxlineitem{Type}
\sphinxAtStartPar
np.ndarray

\end{description}\end{quote}

\end{fulllineitems}

\index{interpolator (OpticalSystems.OpticalSystem attribute)@\spxentry{interpolator}\spxextra{OpticalSystems.OpticalSystem attribute}}

\begin{fulllineitems}
\phantomsection\label{\detokenize{source/OpticalSystems:OpticalSystems.OpticalSystem.interpolator}}
\pysigstartsignatures
\pysigline
{\sphinxbfcode{\sphinxupquote{interpolator}}}
\pysigstopsignatures
\sphinxAtStartPar
Interpolator for OTF.
\begin{quote}\begin{description}
\sphinxlineitem{Type}
\sphinxAtStartPar
scipy.interpolate.RegularGridInterpolator

\end{description}\end{quote}

\end{fulllineitems}

\index{\_otf\_frequencies (OpticalSystems.OpticalSystem attribute)@\spxentry{\_otf\_frequencies}\spxextra{OpticalSystems.OpticalSystem attribute}}

\begin{fulllineitems}
\phantomsection\label{\detokenize{source/OpticalSystems:OpticalSystems.OpticalSystem._otf_frequencies}}
\pysigstartsignatures
\pysigline
{\sphinxbfcode{\sphinxupquote{\_otf\_frequencies}}}
\pysigstopsignatures
\sphinxAtStartPar
Frequencies for OTF.
\begin{quote}\begin{description}
\sphinxlineitem{Type}
\sphinxAtStartPar
np.ndarray

\end{description}\end{quote}

\end{fulllineitems}

\index{\_psf\_coordinates (OpticalSystems.OpticalSystem attribute)@\spxentry{\_psf\_coordinates}\spxextra{OpticalSystems.OpticalSystem attribute}}

\begin{fulllineitems}
\phantomsection\label{\detokenize{source/OpticalSystems:OpticalSystems.OpticalSystem._psf_coordinates}}
\pysigstartsignatures
\pysigline
{\sphinxbfcode{\sphinxupquote{\_psf\_coordinates}}}
\pysigstopsignatures
\sphinxAtStartPar
Coordinates for PSF.
\begin{quote}\begin{description}
\sphinxlineitem{Type}
\sphinxAtStartPar
np.ndarray

\end{description}\end{quote}

\end{fulllineitems}

\index{\_interpolation\_method (OpticalSystems.OpticalSystem attribute)@\spxentry{\_interpolation\_method}\spxextra{OpticalSystems.OpticalSystem attribute}}

\begin{fulllineitems}
\phantomsection\label{\detokenize{source/OpticalSystems:OpticalSystems.OpticalSystem._interpolation_method}}
\pysigstartsignatures
\pysigline
{\sphinxbfcode{\sphinxupquote{\_interpolation\_method}}}
\pysigstopsignatures
\sphinxAtStartPar
Interpolation method.
\begin{quote}\begin{description}
\sphinxlineitem{Type}
\sphinxAtStartPar
str

\end{description}\end{quote}

\end{fulllineitems}


\sphinxincludegraphics[]{inheritance-a102f1a2805ccede33f452ba15b4ea60096d1dd4.pdf}
\index{compute\_psf\_and\_otf() (OpticalSystems.OpticalSystem method)@\spxentry{compute\_psf\_and\_otf()}\spxextra{OpticalSystems.OpticalSystem method}}

\begin{fulllineitems}
\phantomsection\label{\detokenize{source/OpticalSystems:OpticalSystems.OpticalSystem.compute_psf_and_otf}}
\pysigstartsignatures
\pysiglinewithargsret
{\sphinxbfcode{\sphinxupquote{abstract\DUrole{w}{ }}}\sphinxbfcode{\sphinxupquote{compute\_psf\_and\_otf}}}
{}
{{ $\rightarrow$ tuple\DUrole{p}{{[}}ndarray\DUrole{p}{{[}}tuple\DUrole{p}{{[}}int\DUrole{p}{,}\DUrole{w}{ }int\DUrole{p}{,}\DUrole{w}{ }int\DUrole{p}{{]}}\DUrole{p}{,}\DUrole{w}{ }float64\DUrole{p}{{]}}\DUrole{p}{,}\DUrole{w}{ }ndarray\DUrole{p}{{[}}tuple\DUrole{p}{{[}}int\DUrole{p}{,}\DUrole{w}{ }int\DUrole{p}{,}\DUrole{w}{ }int\DUrole{p}{{]}}\DUrole{p}{,}\DUrole{w}{ }float64\DUrole{p}{{]}}\DUrole{p}{{]}}}}
\pysigstopsignatures
\sphinxAtStartPar
Compute the PSF and OTF.

\end{fulllineitems}

\index{compute\_psf\_and\_otf\_cordinates() (OpticalSystems.OpticalSystem method)@\spxentry{compute\_psf\_and\_otf\_cordinates()}\spxextra{OpticalSystems.OpticalSystem method}}

\begin{fulllineitems}
\phantomsection\label{\detokenize{source/OpticalSystems:OpticalSystems.OpticalSystem.compute_psf_and_otf_cordinates}}
\pysigstartsignatures
\pysiglinewithargsret
{\sphinxbfcode{\sphinxupquote{abstract\DUrole{w}{ }}}\sphinxbfcode{\sphinxupquote{compute\_psf\_and\_otf\_cordinates}}}
{\sphinxparam{\DUrole{n}{psf\_size}\DUrole{p}{:}\DUrole{w}{ }\DUrole{n}{tuple\DUrole{p}{{[}}int\DUrole{p}{{]}}}}\sphinxparamcomma \sphinxparam{\DUrole{n}{N}\DUrole{p}{:}\DUrole{w}{ }\DUrole{n}{int}}}
{}
\pysigstopsignatures
\sphinxAtStartPar
Compute the PSF and OTF coordinate axes.
\begin{quote}\begin{description}
\sphinxlineitem{Parameters}\begin{itemize}
\item {} 
\sphinxAtStartPar
\sphinxstyleliteralstrong{\sphinxupquote{psf\_size}} (\sphinxstyleliteralemphasis{\sphinxupquote{tuple}}) \textendash{} Size of the PSF.

\item {} 
\sphinxAtStartPar
\sphinxstyleliteralstrong{\sphinxupquote{N}} (\sphinxstyleliteralemphasis{\sphinxupquote{int}}) \textendash{} Number of points.

\end{itemize}

\end{description}\end{quote}

\end{fulllineitems}

\index{compute\_q\_grid() (OpticalSystems.OpticalSystem method)@\spxentry{compute\_q\_grid()}\spxextra{OpticalSystems.OpticalSystem method}}

\begin{fulllineitems}
\phantomsection\label{\detokenize{source/OpticalSystems:OpticalSystems.OpticalSystem.compute_q_grid}}
\pysigstartsignatures
\pysiglinewithargsret
{\sphinxbfcode{\sphinxupquote{abstract\DUrole{w}{ }}}\sphinxbfcode{\sphinxupquote{compute\_q\_grid}}}
{}
{{ $\rightarrow$ ndarray\DUrole{p}{{[}}tuple\DUrole{p}{{[}}int\DUrole{p}{,}\DUrole{w}{ }int\DUrole{p}{,}\DUrole{w}{ }int\DUrole{p}{,}\DUrole{w}{ }\DUrole{m}{3}\DUrole{p}{{]}}\DUrole{p}{,}\DUrole{w}{ }float64\DUrole{p}{{]}}}}
\pysigstopsignatures
\sphinxAtStartPar
Compute the q\sphinxhyphen{}grid for the OTF.
\begin{quote}\begin{description}
\sphinxlineitem{Returns}
\sphinxAtStartPar
Computed q\sphinxhyphen{}grid.

\sphinxlineitem{Return type}
\sphinxAtStartPar
np.ndarray

\end{description}\end{quote}

\end{fulllineitems}

\index{compute\_x\_grid() (OpticalSystems.OpticalSystem method)@\spxentry{compute\_x\_grid()}\spxextra{OpticalSystems.OpticalSystem method}}

\begin{fulllineitems}
\phantomsection\label{\detokenize{source/OpticalSystems:OpticalSystems.OpticalSystem.compute_x_grid}}
\pysigstartsignatures
\pysiglinewithargsret
{\sphinxbfcode{\sphinxupquote{abstract\DUrole{w}{ }}}\sphinxbfcode{\sphinxupquote{compute\_x\_grid}}}
{}
{{ $\rightarrow$ ndarray\DUrole{p}{{[}}tuple\DUrole{p}{{[}}int\DUrole{p}{,}\DUrole{w}{ }int\DUrole{p}{,}\DUrole{w}{ }int\DUrole{p}{,}\DUrole{w}{ }\DUrole{m}{3}\DUrole{p}{{]}}\DUrole{p}{,}\DUrole{w}{ }float64\DUrole{p}{{]}}}}
\pysigstopsignatures
\sphinxAtStartPar
Compute the x\sphinxhyphen{}grid for the PSF.
\begin{quote}\begin{description}
\sphinxlineitem{Returns}
\sphinxAtStartPar
Computed x\sphinxhyphen{}grid.

\sphinxlineitem{Return type}
\sphinxAtStartPar
np.ndarray

\end{description}\end{quote}

\end{fulllineitems}

\index{interpolate\_otf() (OpticalSystems.OpticalSystem method)@\spxentry{interpolate\_otf()}\spxextra{OpticalSystems.OpticalSystem method}}

\begin{fulllineitems}
\phantomsection\label{\detokenize{source/OpticalSystems:OpticalSystems.OpticalSystem.interpolate_otf}}
\pysigstartsignatures
\pysiglinewithargsret
{\sphinxbfcode{\sphinxupquote{interpolate\_otf}}}
{\sphinxparam{\DUrole{n}{k\_shift}\DUrole{p}{:}\DUrole{w}{ }\DUrole{n}{ndarray\DUrole{p}{{[}}\DUrole{m}{3}\DUrole{p}{,}\DUrole{w}{ }float64\DUrole{p}{{]}}}}}
{{ $\rightarrow$ ndarray\DUrole{p}{{[}}tuple\DUrole{p}{{[}}int\DUrole{p}{,}\DUrole{w}{ }int\DUrole{p}{,}\DUrole{w}{ }int\DUrole{p}{{]}}\DUrole{p}{,}\DUrole{w}{ }float64\DUrole{p}{{]}}}}
\pysigstopsignatures
\sphinxAtStartPar
Interpolate the OTF with a given shift.
\begin{quote}\begin{description}
\sphinxlineitem{Parameters}
\sphinxAtStartPar
\sphinxstyleliteralstrong{\sphinxupquote{k\_shift}} (\sphinxstyleliteralemphasis{\sphinxupquote{np.ndarray}}) \textendash{} Shift vector for interpolation.

\sphinxlineitem{Returns}
\sphinxAtStartPar
Interpolated OTF.

\sphinxlineitem{Return type}
\sphinxAtStartPar
np.ndarray

\end{description}\end{quote}

\end{fulllineitems}

\index{interpolation\_method (OpticalSystems.OpticalSystem property)@\spxentry{interpolation\_method}\spxextra{OpticalSystems.OpticalSystem property}}

\begin{fulllineitems}
\phantomsection\label{\detokenize{source/OpticalSystems:OpticalSystems.OpticalSystem.interpolation_method}}
\pysigstartsignatures
\pysigline
{\sphinxbfcode{\sphinxupquote{property\DUrole{w}{ }}}\sphinxbfcode{\sphinxupquote{interpolation\_method}}}
\pysigstopsignatures
\end{fulllineitems}

\index{otf\_frequencies (OpticalSystems.OpticalSystem property)@\spxentry{otf\_frequencies}\spxextra{OpticalSystems.OpticalSystem property}}

\begin{fulllineitems}
\phantomsection\label{\detokenize{source/OpticalSystems:OpticalSystems.OpticalSystem.otf_frequencies}}
\pysigstartsignatures
\pysigline
{\sphinxbfcode{\sphinxupquote{property\DUrole{w}{ }}}\sphinxbfcode{\sphinxupquote{otf\_frequencies}}}
\pysigstopsignatures
\end{fulllineitems}

\index{psf\_coordinates (OpticalSystems.OpticalSystem property)@\spxentry{psf\_coordinates}\spxextra{OpticalSystems.OpticalSystem property}}

\begin{fulllineitems}
\phantomsection\label{\detokenize{source/OpticalSystems:OpticalSystems.OpticalSystem.psf_coordinates}}
\pysigstartsignatures
\pysigline
{\sphinxbfcode{\sphinxupquote{abstract\DUrole{w}{ }property\DUrole{w}{ }}}\sphinxbfcode{\sphinxupquote{psf\_coordinates}}}
\pysigstopsignatures
\end{fulllineitems}

\index{supported\_interpolation\_methods (OpticalSystems.OpticalSystem attribute)@\spxentry{supported\_interpolation\_methods}\spxextra{OpticalSystems.OpticalSystem attribute}}

\begin{fulllineitems}
\phantomsection\label{\detokenize{source/OpticalSystems:id0}}
\pysigstartsignatures
\pysigline
{\sphinxbfcode{\sphinxupquote{supported\_interpolation\_methods}}\sphinxbfcode{\sphinxupquote{\DUrole{w}{ }\DUrole{p}{=}\DUrole{w}{ }{[}\textquotesingle{}linear\textquotesingle{}, \textquotesingle{}Fourier\textquotesingle{}{]}}}}
\pysigstopsignatures
\sphinxAtStartPar
A list of currently supported interpolation methods.
Other scipy interpolation methods are not directly supported due to high memory usage.
Add them to the list if needed.
Currently, meaningless, but changes are expected.
Fourier is interpolation is used for SIM by default.
Linear interpolation is available with the “interpolate\_OTF” method if needed.

\end{fulllineitems}


\end{fulllineitems}

\index{OpticalSystem2D (class in OpticalSystems)@\spxentry{OpticalSystem2D}\spxextra{class in OpticalSystems}}

\begin{fulllineitems}
\phantomsection\label{\detokenize{source/OpticalSystems:OpticalSystems.OpticalSystem2D}}
\pysigstartsignatures
\pysiglinewithargsret
{\sphinxbfcode{\sphinxupquote{class\DUrole{w}{ }}}\sphinxcode{\sphinxupquote{OpticalSystems.}}\sphinxbfcode{\sphinxupquote{OpticalSystem2D}}}
{\sphinxparam{\DUrole{n}{interpolation\_method}}}
{}
\pysigstopsignatures
\sphinxAtStartPar
Bases: {\hyperref[\detokenize{source/OpticalSystems:OpticalSystems.OpticalSystem}]{\sphinxcrossref{\sphinxcode{\sphinxupquote{OpticalSystem}}}}}

\sphinxincludegraphics[]{inheritance-f9c5e725c0244ae6b5ba7dd52d4c8865d2dd7163.pdf}
\index{compute\_effective\_otfs\_2dSIM() (OpticalSystems.OpticalSystem2D method)@\spxentry{compute\_effective\_otfs\_2dSIM()}\spxextra{OpticalSystems.OpticalSystem2D method}}

\begin{fulllineitems}
\phantomsection\label{\detokenize{source/OpticalSystems:OpticalSystems.OpticalSystem2D.compute_effective_otfs_2dSIM}}
\pysigstartsignatures
\pysiglinewithargsret
{\sphinxbfcode{\sphinxupquote{compute\_effective\_otfs\_2dSIM}}}
{\sphinxparam{\DUrole{n}{illumination}\DUrole{p}{:}\DUrole{w}{ }\DUrole{n}{{\hyperref[\detokenize{source/Illumination:Illumination.Illumination}]{\sphinxcrossref{Illumination}}}}}}
{{ $\rightarrow$ dict\DUrole{p}{{[}}tuple\DUrole{p}{{[}}int\DUrole{p}{,}\DUrole{w}{ }tuple\DUrole{p}{{[}}int\DUrole{p}{{]}}\DUrole{p}{{]}}\DUrole{p}{,}\DUrole{w}{ }float64\DUrole{p}{{]}}}}
\pysigstopsignatures
\sphinxAtStartPar
Compute the effective OTFs for 2D SIM illumination
(in the case of 2D SIM they are just shifted).
\begin{quote}\begin{description}
\sphinxlineitem{Parameters}
\sphinxAtStartPar
\sphinxstyleliteralstrong{\sphinxupquote{illumination}} \textendash{} Illumination object containing wave information.

\sphinxlineitem{Returns}
\sphinxAtStartPar
Effective OTFs.

\sphinxlineitem{Return type}
\sphinxAtStartPar
dict

\end{description}\end{quote}

\end{fulllineitems}

\index{compute\_psf\_and\_otf() (OpticalSystems.OpticalSystem2D method)@\spxentry{compute\_psf\_and\_otf()}\spxextra{OpticalSystems.OpticalSystem2D method}}

\begin{fulllineitems}
\phantomsection\label{\detokenize{source/OpticalSystems:OpticalSystems.OpticalSystem2D.compute_psf_and_otf}}
\pysigstartsignatures
\pysiglinewithargsret
{\sphinxbfcode{\sphinxupquote{compute\_psf\_and\_otf}}}
{}
{{ $\rightarrow$ tuple\DUrole{p}{{[}}float64\DUrole{p}{,}\DUrole{w}{ }float64\DUrole{p}{{]}}}}
\pysigstopsignatures
\sphinxAtStartPar
Compute the PSF and OTF.

\end{fulllineitems}

\index{compute\_psf\_and\_otf\_cordinates() (OpticalSystems.OpticalSystem2D method)@\spxentry{compute\_psf\_and\_otf\_cordinates()}\spxextra{OpticalSystems.OpticalSystem2D method}}

\begin{fulllineitems}
\phantomsection\label{\detokenize{source/OpticalSystems:OpticalSystems.OpticalSystem2D.compute_psf_and_otf_cordinates}}
\pysigstartsignatures
\pysiglinewithargsret
{\sphinxbfcode{\sphinxupquote{compute\_psf\_and\_otf\_cordinates}}}
{\sphinxparam{\DUrole{n}{psf\_size}\DUrole{p}{:}\DUrole{w}{ }\DUrole{n}{tuple\DUrole{p}{{[}}float\DUrole{p}{{]}}}}\sphinxparamcomma \sphinxparam{\DUrole{n}{N}\DUrole{p}{:}\DUrole{w}{ }\DUrole{n}{int}}}
{}
\pysigstopsignatures
\sphinxAtStartPar
Compute the PSF and OTF coordinate axes.
\begin{quote}\begin{description}
\sphinxlineitem{Parameters}\begin{itemize}
\item {} 
\sphinxAtStartPar
\sphinxstyleliteralstrong{\sphinxupquote{psf\_size}} (\sphinxstyleliteralemphasis{\sphinxupquote{tuple}}) \textendash{} Size of the PSF.

\item {} 
\sphinxAtStartPar
\sphinxstyleliteralstrong{\sphinxupquote{N}} (\sphinxstyleliteralemphasis{\sphinxupquote{int}}) \textendash{} Number of points.

\end{itemize}

\end{description}\end{quote}

\end{fulllineitems}

\index{compute\_q\_grid() (OpticalSystems.OpticalSystem2D method)@\spxentry{compute\_q\_grid()}\spxextra{OpticalSystems.OpticalSystem2D method}}

\begin{fulllineitems}
\phantomsection\label{\detokenize{source/OpticalSystems:OpticalSystems.OpticalSystem2D.compute_q_grid}}
\pysigstartsignatures
\pysiglinewithargsret
{\sphinxbfcode{\sphinxupquote{compute\_q\_grid}}}
{}
{{ $\rightarrow$ ndarray\DUrole{p}{{[}}tuple\DUrole{p}{{[}}int\DUrole{p}{,}\DUrole{w}{ }int\DUrole{p}{,}\DUrole{w}{ }\DUrole{m}{2}\DUrole{p}{{]}}\DUrole{p}{,}\DUrole{w}{ }float64\DUrole{p}{{]}}}}
\pysigstopsignatures
\sphinxAtStartPar
Compute the q\sphinxhyphen{}grid for the OTF.
\begin{quote}\begin{description}
\sphinxlineitem{Returns}
\sphinxAtStartPar
Computed q\sphinxhyphen{}grid.

\sphinxlineitem{Return type}
\sphinxAtStartPar
np.ndarray

\end{description}\end{quote}

\end{fulllineitems}

\index{compute\_x\_grid() (OpticalSystems.OpticalSystem2D method)@\spxentry{compute\_x\_grid()}\spxextra{OpticalSystems.OpticalSystem2D method}}

\begin{fulllineitems}
\phantomsection\label{\detokenize{source/OpticalSystems:OpticalSystems.OpticalSystem2D.compute_x_grid}}
\pysigstartsignatures
\pysiglinewithargsret
{\sphinxbfcode{\sphinxupquote{compute\_x\_grid}}}
{}
{{ $\rightarrow$ ndarray\DUrole{p}{{[}}tuple\DUrole{p}{{[}}int\DUrole{p}{,}\DUrole{w}{ }int\DUrole{p}{,}\DUrole{w}{ }\DUrole{m}{2}\DUrole{p}{{]}}\DUrole{p}{,}\DUrole{w}{ }float64\DUrole{p}{{]}}}}
\pysigstopsignatures
\sphinxAtStartPar
Compute the x\sphinxhyphen{}grid for the PSF.
\begin{quote}\begin{description}
\sphinxlineitem{Returns}
\sphinxAtStartPar
Computed x\sphinxhyphen{}grid.

\sphinxlineitem{Return type}
\sphinxAtStartPar
np.ndarray

\end{description}\end{quote}

\end{fulllineitems}

\index{interpolate\_otf() (OpticalSystems.OpticalSystem2D method)@\spxentry{interpolate\_otf()}\spxextra{OpticalSystems.OpticalSystem2D method}}

\begin{fulllineitems}
\phantomsection\label{\detokenize{source/OpticalSystems:OpticalSystems.OpticalSystem2D.interpolate_otf}}
\pysigstartsignatures
\pysiglinewithargsret
{\sphinxbfcode{\sphinxupquote{interpolate\_otf}}}
{\sphinxparam{\DUrole{n}{k\_shift}\DUrole{p}{:}\DUrole{w}{ }\DUrole{n}{ndarray\DUrole{p}{{[}}\DUrole{m}{3}\DUrole{p}{,}\DUrole{w}{ }float64\DUrole{p}{{]}}}}}
{{ $\rightarrow$ ndarray\DUrole{p}{{[}}tuple\DUrole{p}{{[}}int\DUrole{p}{,}\DUrole{w}{ }int\DUrole{p}{,}\DUrole{w}{ }int\DUrole{p}{{]}}\DUrole{p}{,}\DUrole{w}{ }float64\DUrole{p}{{]}}}}
\pysigstopsignatures
\sphinxAtStartPar
Interpolate the OTF with a given shift.
\begin{quote}\begin{description}
\sphinxlineitem{Parameters}
\sphinxAtStartPar
\sphinxstyleliteralstrong{\sphinxupquote{k\_shift}} (\sphinxstyleliteralemphasis{\sphinxupquote{np.ndarray}}) \textendash{} Shift vector for interpolation.

\sphinxlineitem{Returns}
\sphinxAtStartPar
Interpolated OTF.

\sphinxlineitem{Return type}
\sphinxAtStartPar
np.ndarray

\end{description}\end{quote}

\end{fulllineitems}

\index{psf\_coordinates (OpticalSystems.OpticalSystem2D property)@\spxentry{psf\_coordinates}\spxextra{OpticalSystems.OpticalSystem2D property}}

\begin{fulllineitems}
\phantomsection\label{\detokenize{source/OpticalSystems:OpticalSystems.OpticalSystem2D.psf_coordinates}}
\pysigstartsignatures
\pysigline
{\sphinxbfcode{\sphinxupquote{property\DUrole{w}{ }}}\sphinxbfcode{\sphinxupquote{psf\_coordinates}}}
\pysigstopsignatures
\end{fulllineitems}


\end{fulllineitems}

\index{OpticalSystem3D (class in OpticalSystems)@\spxentry{OpticalSystem3D}\spxextra{class in OpticalSystems}}

\begin{fulllineitems}
\phantomsection\label{\detokenize{source/OpticalSystems:OpticalSystems.OpticalSystem3D}}
\pysigstartsignatures
\pysiglinewithargsret
{\sphinxbfcode{\sphinxupquote{class\DUrole{w}{ }}}\sphinxcode{\sphinxupquote{OpticalSystems.}}\sphinxbfcode{\sphinxupquote{OpticalSystem3D}}}
{\sphinxparam{\DUrole{n}{interpolation\_method}}}
{}
\pysigstopsignatures
\sphinxAtStartPar
Bases: {\hyperref[\detokenize{source/OpticalSystems:OpticalSystems.OpticalSystem}]{\sphinxcrossref{\sphinxcode{\sphinxupquote{OpticalSystem}}}}}

\sphinxincludegraphics[]{inheritance-9897bed5996dc999a21655be623b80a256b9402a.pdf}
\index{compute\_effective\_otfs\_projective\_3dSIM() (OpticalSystems.OpticalSystem3D method)@\spxentry{compute\_effective\_otfs\_projective\_3dSIM()}\spxextra{OpticalSystems.OpticalSystem3D method}}

\begin{fulllineitems}
\phantomsection\label{\detokenize{source/OpticalSystems:OpticalSystems.OpticalSystem3D.compute_effective_otfs_projective_3dSIM}}
\pysigstartsignatures
\pysiglinewithargsret
{\sphinxbfcode{\sphinxupquote{compute\_effective\_otfs\_projective\_3dSIM}}}
{\sphinxparam{\DUrole{n}{illumination}\DUrole{p}{:}\DUrole{w}{ }\DUrole{n}{{\hyperref[\detokenize{source/Illumination:Illumination.Illumination}]{\sphinxcrossref{Illumination}}}}}}
{{ $\rightarrow$ dict\DUrole{p}{{[}}tuple\DUrole{p}{{[}}int\DUrole{p}{,}\DUrole{w}{ }tuple\DUrole{p}{{[}}int\DUrole{p}{{]}}\DUrole{p}{{]}}\DUrole{p}{,}\DUrole{w}{ }float64\DUrole{p}{{]}}}}
\pysigstopsignatures
\sphinxAtStartPar
Compute the effective OTFs for projective 3D SIM illumination.
\begin{quote}\begin{description}
\sphinxlineitem{Parameters}
\sphinxAtStartPar
\sphinxstyleliteralstrong{\sphinxupquote{illumination}} \textendash{} Illumination object containing wave information.

\sphinxlineitem{Returns}
\sphinxAtStartPar
Effective OTFs.

\sphinxlineitem{Return type}
\sphinxAtStartPar
dict

\end{description}\end{quote}

\end{fulllineitems}

\index{compute\_effective\_otfs\_true\_3dSIM() (OpticalSystems.OpticalSystem3D method)@\spxentry{compute\_effective\_otfs\_true\_3dSIM()}\spxextra{OpticalSystems.OpticalSystem3D method}}

\begin{fulllineitems}
\phantomsection\label{\detokenize{source/OpticalSystems:OpticalSystems.OpticalSystem3D.compute_effective_otfs_true_3dSIM}}
\pysigstartsignatures
\pysiglinewithargsret
{\sphinxbfcode{\sphinxupquote{compute\_effective\_otfs\_true\_3dSIM}}}
{\sphinxparam{\DUrole{n}{illumination}\DUrole{p}{:}\DUrole{w}{ }\DUrole{n}{{\hyperref[\detokenize{source/Illumination:Illumination.Illumination}]{\sphinxcrossref{Illumination}}}}}}
{{ $\rightarrow$ dict\DUrole{p}{{[}}tuple\DUrole{p}{{[}}int\DUrole{p}{,}\DUrole{w}{ }tuple\DUrole{p}{{[}}int\DUrole{p}{{]}}\DUrole{p}{{]}}\DUrole{p}{,}\DUrole{w}{ }float64\DUrole{p}{{]}}}}
\pysigstopsignatures
\sphinxAtStartPar
Compute the effective OTFs for true 3D SIM
(in the case of true 3D SIM, they are just shifted).
\begin{quote}\begin{description}
\sphinxlineitem{Parameters}
\sphinxAtStartPar
\sphinxstyleliteralstrong{\sphinxupquote{illumination}} \textendash{} Illumination object containing wave information.

\sphinxlineitem{Returns}
\sphinxAtStartPar
Effective PSFs and OTFs.

\sphinxlineitem{Return type}
\sphinxAtStartPar
tuple

\end{description}\end{quote}

\end{fulllineitems}

\index{compute\_psf\_and\_otf() (OpticalSystems.OpticalSystem3D method)@\spxentry{compute\_psf\_and\_otf()}\spxextra{OpticalSystems.OpticalSystem3D method}}

\begin{fulllineitems}
\phantomsection\label{\detokenize{source/OpticalSystems:OpticalSystems.OpticalSystem3D.compute_psf_and_otf}}
\pysigstartsignatures
\pysiglinewithargsret
{\sphinxbfcode{\sphinxupquote{compute\_psf\_and\_otf}}}
{}
{{ $\rightarrow$ tuple\DUrole{p}{{[}}float64\DUrole{p}{,}\DUrole{w}{ }float64\DUrole{p}{{]}}}}
\pysigstopsignatures
\sphinxAtStartPar
Compute the PSF and OTF.

\end{fulllineitems}

\index{compute\_psf\_and\_otf\_cordinates() (OpticalSystems.OpticalSystem3D method)@\spxentry{compute\_psf\_and\_otf\_cordinates()}\spxextra{OpticalSystems.OpticalSystem3D method}}

\begin{fulllineitems}
\phantomsection\label{\detokenize{source/OpticalSystems:OpticalSystems.OpticalSystem3D.compute_psf_and_otf_cordinates}}
\pysigstartsignatures
\pysiglinewithargsret
{\sphinxbfcode{\sphinxupquote{compute\_psf\_and\_otf\_cordinates}}}
{\sphinxparam{\DUrole{n}{psf\_size}}\sphinxparamcomma \sphinxparam{\DUrole{n}{N}}}
{}
\pysigstopsignatures
\sphinxAtStartPar
Compute the PSF and OTF coordinate axes.
\begin{quote}\begin{description}
\sphinxlineitem{Parameters}\begin{itemize}
\item {} 
\sphinxAtStartPar
\sphinxstyleliteralstrong{\sphinxupquote{psf\_size}} (\sphinxstyleliteralemphasis{\sphinxupquote{tuple}}) \textendash{} Size of the PSF.

\item {} 
\sphinxAtStartPar
\sphinxstyleliteralstrong{\sphinxupquote{N}} (\sphinxstyleliteralemphasis{\sphinxupquote{int}}) \textendash{} Number of points.

\end{itemize}

\end{description}\end{quote}

\end{fulllineitems}

\index{compute\_q\_grid() (OpticalSystems.OpticalSystem3D method)@\spxentry{compute\_q\_grid()}\spxextra{OpticalSystems.OpticalSystem3D method}}

\begin{fulllineitems}
\phantomsection\label{\detokenize{source/OpticalSystems:OpticalSystems.OpticalSystem3D.compute_q_grid}}
\pysigstartsignatures
\pysiglinewithargsret
{\sphinxbfcode{\sphinxupquote{compute\_q\_grid}}}
{}
{{ $\rightarrow$ ndarray\DUrole{p}{{[}}tuple\DUrole{p}{{[}}int\DUrole{p}{,}\DUrole{w}{ }int\DUrole{p}{,}\DUrole{w}{ }int\DUrole{p}{,}\DUrole{w}{ }int\DUrole{p}{{]}}\DUrole{p}{,}\DUrole{w}{ }float64\DUrole{p}{{]}}}}
\pysigstopsignatures
\sphinxAtStartPar
Compute the q\sphinxhyphen{}grid for the OTF.
\begin{quote}\begin{description}
\sphinxlineitem{Returns}
\sphinxAtStartPar
Computed q\sphinxhyphen{}grid.

\sphinxlineitem{Return type}
\sphinxAtStartPar
np.ndarray

\end{description}\end{quote}

\end{fulllineitems}

\index{compute\_x\_grid() (OpticalSystems.OpticalSystem3D method)@\spxentry{compute\_x\_grid()}\spxextra{OpticalSystems.OpticalSystem3D method}}

\begin{fulllineitems}
\phantomsection\label{\detokenize{source/OpticalSystems:OpticalSystems.OpticalSystem3D.compute_x_grid}}
\pysigstartsignatures
\pysiglinewithargsret
{\sphinxbfcode{\sphinxupquote{compute\_x\_grid}}}
{}
{{ $\rightarrow$ ndarray\DUrole{p}{{[}}tuple\DUrole{p}{{[}}int\DUrole{p}{,}\DUrole{w}{ }int\DUrole{p}{,}\DUrole{w}{ }int\DUrole{p}{,}\DUrole{w}{ }int\DUrole{p}{{]}}\DUrole{p}{,}\DUrole{w}{ }float64\DUrole{p}{{]}}}}
\pysigstopsignatures
\sphinxAtStartPar
Compute the x\sphinxhyphen{}grid for the PSF.
\begin{quote}\begin{description}
\sphinxlineitem{Returns}
\sphinxAtStartPar
Computed x\sphinxhyphen{}grid.

\sphinxlineitem{Return type}
\sphinxAtStartPar
np.ndarray

\end{description}\end{quote}

\end{fulllineitems}

\index{interpolate\_otf() (OpticalSystems.OpticalSystem3D method)@\spxentry{interpolate\_otf()}\spxextra{OpticalSystems.OpticalSystem3D method}}

\begin{fulllineitems}
\phantomsection\label{\detokenize{source/OpticalSystems:OpticalSystems.OpticalSystem3D.interpolate_otf}}
\pysigstartsignatures
\pysiglinewithargsret
{\sphinxbfcode{\sphinxupquote{interpolate\_otf}}}
{\sphinxparam{\DUrole{n}{k\_shift}\DUrole{p}{:}\DUrole{w}{ }\DUrole{n}{ndarray\DUrole{p}{{[}}\DUrole{m}{3}\DUrole{p}{,}\DUrole{w}{ }float64\DUrole{p}{{]}}}}}
{{ $\rightarrow$ float64}}
\pysigstopsignatures
\sphinxAtStartPar
Interpolate the OTF with a given shift.
\begin{quote}\begin{description}
\sphinxlineitem{Parameters}
\sphinxAtStartPar
\sphinxstyleliteralstrong{\sphinxupquote{k\_shift}} (\sphinxstyleliteralemphasis{\sphinxupquote{np.ndarray}}) \textendash{} Shift vector for interpolation.

\sphinxlineitem{Returns}
\sphinxAtStartPar
Interpolated OTF.

\sphinxlineitem{Return type}
\sphinxAtStartPar
np.ndarray

\end{description}\end{quote}

\end{fulllineitems}

\index{psf\_coordinates (OpticalSystems.OpticalSystem3D property)@\spxentry{psf\_coordinates}\spxextra{OpticalSystems.OpticalSystem3D property}}

\begin{fulllineitems}
\phantomsection\label{\detokenize{source/OpticalSystems:OpticalSystems.OpticalSystem3D.psf_coordinates}}
\pysigstartsignatures
\pysigline
{\sphinxbfcode{\sphinxupquote{property\DUrole{w}{ }}}\sphinxbfcode{\sphinxupquote{psf\_coordinates}}}
\pysigstopsignatures
\end{fulllineitems}


\end{fulllineitems}

\index{System4f2D (class in OpticalSystems)@\spxentry{System4f2D}\spxextra{class in OpticalSystems}}

\begin{fulllineitems}
\phantomsection\label{\detokenize{source/OpticalSystems:OpticalSystems.System4f2D}}
\pysigstartsignatures
\pysiglinewithargsret
{\sphinxbfcode{\sphinxupquote{class\DUrole{w}{ }}}\sphinxcode{\sphinxupquote{OpticalSystems.}}\sphinxbfcode{\sphinxupquote{System4f2D}}}
{\sphinxparam{\DUrole{n}{alpha}\DUrole{o}{=}\DUrole{default_value}{0.7853981633974483}}\sphinxparamcomma \sphinxparam{\DUrole{n}{refractive\_index}\DUrole{o}{=}\DUrole{default_value}{1}}\sphinxparamcomma \sphinxparam{\DUrole{n}{interpolation\_method}\DUrole{o}{=}\DUrole{default_value}{\textquotesingle{}linear\textquotesingle{}}}}
{}
\pysigstopsignatures
\sphinxAtStartPar
Bases: {\hyperref[\detokenize{source/OpticalSystems:OpticalSystems.OpticalSystem2D}]{\sphinxcrossref{\sphinxcode{\sphinxupquote{OpticalSystem2D}}}}}

\sphinxincludegraphics[]{inheritance-fcf82fd85d9c96eb798f82c5a05e50f1ec77cbf3.pdf}
\index{compute\_psf\_and\_otf() (OpticalSystems.System4f2D method)@\spxentry{compute\_psf\_and\_otf()}\spxextra{OpticalSystems.System4f2D method}}

\begin{fulllineitems}
\phantomsection\label{\detokenize{source/OpticalSystems:OpticalSystems.System4f2D.compute_psf_and_otf}}
\pysigstartsignatures
\pysiglinewithargsret
{\sphinxbfcode{\sphinxupquote{compute\_psf\_and\_otf}}}
{\sphinxparam{\DUrole{n}{parameters}\DUrole{o}{=}\DUrole{default_value}{None}}\sphinxparamcomma \sphinxparam{\DUrole{n}{pupil\_function}\DUrole{o}{=}\DUrole{default_value}{None}}\sphinxparamcomma \sphinxparam{\DUrole{n}{mask}\DUrole{o}{=}\DUrole{default_value}{None}}}
{}
\pysigstopsignatures
\sphinxAtStartPar
Compute the PSF and OTF.

\end{fulllineitems}


\end{fulllineitems}

\index{System4f3D (class in OpticalSystems)@\spxentry{System4f3D}\spxextra{class in OpticalSystems}}

\begin{fulllineitems}
\phantomsection\label{\detokenize{source/OpticalSystems:OpticalSystems.System4f3D}}
\pysigstartsignatures
\pysiglinewithargsret
{\sphinxbfcode{\sphinxupquote{class\DUrole{w}{ }}}\sphinxcode{\sphinxupquote{OpticalSystems.}}\sphinxbfcode{\sphinxupquote{System4f3D}}}
{\sphinxparam{\DUrole{n}{alpha}\DUrole{o}{=}\DUrole{default_value}{0.7853981633974483}}\sphinxparamcomma \sphinxparam{\DUrole{n}{refractive\_index\_sample}\DUrole{o}{=}\DUrole{default_value}{1}}\sphinxparamcomma \sphinxparam{\DUrole{n}{refractive\_index\_medium}\DUrole{o}{=}\DUrole{default_value}{1}}\sphinxparamcomma \sphinxparam{\DUrole{n}{regularization\_parameter}\DUrole{o}{=}\DUrole{default_value}{0.01}}\sphinxparamcomma \sphinxparam{\DUrole{n}{interpolation\_method}\DUrole{o}{=}\DUrole{default_value}{\textquotesingle{}linear\textquotesingle{}}}}
{}
\pysigstopsignatures
\sphinxAtStartPar
Bases: {\hyperref[\detokenize{source/OpticalSystems:OpticalSystems.OpticalSystem3D}]{\sphinxcrossref{\sphinxcode{\sphinxupquote{OpticalSystem3D}}}}}

\sphinxincludegraphics[]{inheritance-275da061e59f597a4b5630a98be9a2443cbb1439.pdf}
\index{compute\_psf\_and\_otf() (OpticalSystems.System4f3D method)@\spxentry{compute\_psf\_and\_otf()}\spxextra{OpticalSystems.System4f3D method}}

\begin{fulllineitems}
\phantomsection\label{\detokenize{source/OpticalSystems:OpticalSystems.System4f3D.compute_psf_and_otf}}
\pysigstartsignatures
\pysiglinewithargsret
{\sphinxbfcode{\sphinxupquote{compute\_psf\_and\_otf}}}
{\sphinxparam{\DUrole{n}{parameters=None}}\sphinxparamcomma \sphinxparam{\DUrole{n}{high\_NA=False}}\sphinxparamcomma \sphinxparam{\DUrole{n}{apodization\_function=\textquotesingle{}Sine\textquotesingle{}}}\sphinxparamcomma \sphinxparam{\DUrole{n}{pupil\_function=\textless{}function System4f3D.\textless{}lambda\textgreater{}\textgreater{}}}\sphinxparamcomma \sphinxparam{\DUrole{n}{mask=None}}}
{}
\pysigstopsignatures
\sphinxAtStartPar
Compute the PSF and OTF.

\end{fulllineitems}


\end{fulllineitems}


\sphinxstepscope


\subsection{ProcessorSIM module}
\label{\detokenize{source/ProcessorSIM:module-ProcessorSIM}}\label{\detokenize{source/ProcessorSIM:processorsim-module}}\label{\detokenize{source/ProcessorSIM::doc}}\index{module@\spxentry{module}!ProcessorSIM@\spxentry{ProcessorSIM}}\index{ProcessorSIM@\spxentry{ProcessorSIM}!module@\spxentry{module}}
\sphinxAtStartPar
ProcessorSIM.py

\sphinxAtStartPar
When implemented, this class will be a top\sphinxhyphen{}level class, responsible for SIM reconstructions.
\begin{description}
\sphinxlineitem{Classes:}
\sphinxAtStartPar
ProcessorSIM: Base class for SIM processors.
ProcessorProjective3dSIM: Class for processing projective 3D SIM data.
ProcessorTrue3dSIM: Class for processing true 3D SIM data.

\end{description}

\sphinxincludegraphics[]{inheritance-d50435ccad36d578a942b25091bd6046b7515ce7.pdf}
\index{ProcessorProjective3dSIM (class in ProcessorSIM)@\spxentry{ProcessorProjective3dSIM}\spxextra{class in ProcessorSIM}}

\begin{fulllineitems}
\phantomsection\label{\detokenize{source/ProcessorSIM:ProcessorSIM.ProcessorProjective3dSIM}}
\pysigstartsignatures
\pysiglinewithargsret
{\sphinxbfcode{\sphinxupquote{class\DUrole{w}{ }}}\sphinxcode{\sphinxupquote{ProcessorSIM.}}\sphinxbfcode{\sphinxupquote{ProcessorProjective3dSIM}}}
{\sphinxparam{\DUrole{n}{illumination}}\sphinxparamcomma \sphinxparam{\DUrole{n}{optical\_system}}}
{}
\pysigstopsignatures
\sphinxAtStartPar
Bases: {\hyperref[\detokenize{source/ProcessorSIM:ProcessorSIM.ProcessorSIM}]{\sphinxcrossref{\sphinxcode{\sphinxupquote{ProcessorSIM}}}}}

\sphinxincludegraphics[]{inheritance-337ae16f16e55a826da8e60999ba456f28d74512.pdf}

\end{fulllineitems}

\index{ProcessorSIM (class in ProcessorSIM)@\spxentry{ProcessorSIM}\spxextra{class in ProcessorSIM}}

\begin{fulllineitems}
\phantomsection\label{\detokenize{source/ProcessorSIM:ProcessorSIM.ProcessorSIM}}
\pysigstartsignatures
\pysiglinewithargsret
{\sphinxbfcode{\sphinxupquote{class\DUrole{w}{ }}}\sphinxcode{\sphinxupquote{ProcessorSIM.}}\sphinxbfcode{\sphinxupquote{ProcessorSIM}}}
{\sphinxparam{\DUrole{n}{illumination}}\sphinxparamcomma \sphinxparam{\DUrole{n}{optical\_system}}}
{}
\pysigstopsignatures
\sphinxAtStartPar
Bases: \sphinxcode{\sphinxupquote{object}}

\sphinxincludegraphics[]{inheritance-62addb99a06bba6e6cb3d6c8c8d3b46b324fb96c.pdf}
\index{compute\_apodization\_filter\_autoconvolution() (ProcessorSIM.ProcessorSIM method)@\spxentry{compute\_apodization\_filter\_autoconvolution()}\spxextra{ProcessorSIM.ProcessorSIM method}}

\begin{fulllineitems}
\phantomsection\label{\detokenize{source/ProcessorSIM:ProcessorSIM.ProcessorSIM.compute_apodization_filter_autoconvolution}}
\pysigstartsignatures
\pysiglinewithargsret
{\sphinxbfcode{\sphinxupquote{compute\_apodization\_filter\_autoconvolution}}}
{}
{}
\pysigstopsignatures
\end{fulllineitems}

\index{compute\_apodization\_filter\_lukosz() (ProcessorSIM.ProcessorSIM method)@\spxentry{compute\_apodization\_filter\_lukosz()}\spxextra{ProcessorSIM.ProcessorSIM method}}

\begin{fulllineitems}
\phantomsection\label{\detokenize{source/ProcessorSIM:ProcessorSIM.ProcessorSIM.compute_apodization_filter_lukosz}}
\pysigstartsignatures
\pysiglinewithargsret
{\sphinxbfcode{\sphinxupquote{compute\_apodization\_filter\_lukosz}}}
{}
{}
\pysigstopsignatures
\end{fulllineitems}

\index{compute\_effective\_psfs\_and\_otfs() (ProcessorSIM.ProcessorSIM static method)@\spxentry{compute\_effective\_psfs\_and\_otfs()}\spxextra{ProcessorSIM.ProcessorSIM static method}}

\begin{fulllineitems}
\phantomsection\label{\detokenize{source/ProcessorSIM:ProcessorSIM.ProcessorSIM.compute_effective_psfs_and_otfs}}
\pysigstartsignatures
\pysiglinewithargsret
{\sphinxbfcode{\sphinxupquote{abstract\DUrole{w}{ }static\DUrole{w}{ }}}\sphinxbfcode{\sphinxupquote{compute\_effective\_psfs\_and\_otfs}}}
{\sphinxparam{\DUrole{n}{illumination}}\sphinxparamcomma \sphinxparam{\DUrole{n}{optical\_system}}}
{}
\pysigstopsignatures
\end{fulllineitems}

\index{compute\_sim\_support() (ProcessorSIM.ProcessorSIM method)@\spxentry{compute\_sim\_support()}\spxextra{ProcessorSIM.ProcessorSIM method}}

\begin{fulllineitems}
\phantomsection\label{\detokenize{source/ProcessorSIM:ProcessorSIM.ProcessorSIM.compute_sim_support}}
\pysigstartsignatures
\pysiglinewithargsret
{\sphinxbfcode{\sphinxupquote{compute\_sim\_support}}}
{}
{}
\pysigstopsignatures
\end{fulllineitems}


\end{fulllineitems}

\index{ProcessorTrue3dSIM (class in ProcessorSIM)@\spxentry{ProcessorTrue3dSIM}\spxextra{class in ProcessorSIM}}

\begin{fulllineitems}
\phantomsection\label{\detokenize{source/ProcessorSIM:ProcessorSIM.ProcessorTrue3dSIM}}
\pysigstartsignatures
\pysiglinewithargsret
{\sphinxbfcode{\sphinxupquote{class\DUrole{w}{ }}}\sphinxcode{\sphinxupquote{ProcessorSIM.}}\sphinxbfcode{\sphinxupquote{ProcessorTrue3dSIM}}}
{\sphinxparam{\DUrole{n}{illumination}}\sphinxparamcomma \sphinxparam{\DUrole{n}{optical\_system}}}
{}
\pysigstopsignatures
\sphinxAtStartPar
Bases: {\hyperref[\detokenize{source/ProcessorSIM:ProcessorSIM.ProcessorSIM}]{\sphinxcrossref{\sphinxcode{\sphinxupquote{ProcessorSIM}}}}}

\sphinxincludegraphics[]{inheritance-612b9f9641cfb9ac61a0c6ecd34b181500598119.pdf}

\end{fulllineitems}


\sphinxstepscope


\subsection{SIMulator module}
\label{\detokenize{source/SIMulator:module-SIMulator}}\label{\detokenize{source/SIMulator:simulator-module}}\label{\detokenize{source/SIMulator::doc}}\index{module@\spxentry{module}!SIMulator@\spxentry{SIMulator}}\index{SIMulator@\spxentry{SIMulator}!module@\spxentry{module}}
\sphinxAtStartPar
SIMulator.py

\sphinxAtStartPar
This module contains the SIMulator class for simulating raw
structured illumination microscopy (SIM) images and/or reconstructing
the super resolution images from the raw SIM images.

\sphinxAtStartPar
This class will be probably split into two classes in the future. The detailed documentation will be provided in the further release.

\sphinxincludegraphics[]{inheritance-e6a38cee447bc4dc01b7b65961f96bb5ac43c98e.pdf}
\index{SIMulator (class in SIMulator)@\spxentry{SIMulator}\spxextra{class in SIMulator}}

\begin{fulllineitems}
\phantomsection\label{\detokenize{source/SIMulator:SIMulator.SIMulator}}
\pysigstartsignatures
\pysiglinewithargsret
{\sphinxbfcode{\sphinxupquote{class\DUrole{w}{ }}}\sphinxcode{\sphinxupquote{SIMulator.}}\sphinxbfcode{\sphinxupquote{SIMulator}}}
{\sphinxparam{\DUrole{n}{illumination}}\sphinxparamcomma \sphinxparam{\DUrole{n}{optical\_system}}\sphinxparamcomma \sphinxparam{\DUrole{n}{box\_size}\DUrole{o}{=}\DUrole{default_value}{10}}\sphinxparamcomma \sphinxparam{\DUrole{n}{point\_number}\DUrole{o}{=}\DUrole{default_value}{100}}\sphinxparamcomma \sphinxparam{\DUrole{n}{readout\_noise\_variance}\DUrole{o}{=}\DUrole{default_value}{0}}\sphinxparamcomma \sphinxparam{\DUrole{n}{additional\_info}\DUrole{o}{=}\DUrole{default_value}{None}}}
{}
\pysigstopsignatures
\sphinxAtStartPar
Bases: {\hyperref[\detokenize{source/Box:Box.BoxSIM}]{\sphinxcrossref{\sphinxcode{\sphinxupquote{BoxSIM}}}}}

\sphinxincludegraphics[]{inheritance-9eeda497e809151230d032c0114f6d7169fbae0a.pdf}
\index{generate\_sim\_images() (SIMulator.SIMulator method)@\spxentry{generate\_sim\_images()}\spxextra{SIMulator.SIMulator method}}

\begin{fulllineitems}
\phantomsection\label{\detokenize{source/SIMulator:SIMulator.SIMulator.generate_sim_images}}
\pysigstartsignatures
\pysiglinewithargsret
{\sphinxbfcode{\sphinxupquote{generate\_sim\_images}}}
{\sphinxparam{\DUrole{n}{object}}}
{}
\pysigstopsignatures
\end{fulllineitems}

\index{generate\_sim\_images2d() (SIMulator.SIMulator method)@\spxentry{generate\_sim\_images2d()}\spxextra{SIMulator.SIMulator method}}

\begin{fulllineitems}
\phantomsection\label{\detokenize{source/SIMulator:SIMulator.SIMulator.generate_sim_images2d}}
\pysigstartsignatures
\pysiglinewithargsret
{\sphinxbfcode{\sphinxupquote{generate\_sim\_images2d}}}
{\sphinxparam{\DUrole{n}{object}}}
{}
\pysigstopsignatures
\end{fulllineitems}

\index{generate\_widefield() (SIMulator.SIMulator method)@\spxentry{generate\_widefield()}\spxextra{SIMulator.SIMulator method}}

\begin{fulllineitems}
\phantomsection\label{\detokenize{source/SIMulator:SIMulator.SIMulator.generate_widefield}}
\pysigstartsignatures
\pysiglinewithargsret
{\sphinxbfcode{\sphinxupquote{generate\_widefield}}}
{\sphinxparam{\DUrole{n}{sim\_images}}}
{}
\pysigstopsignatures
\end{fulllineitems}

\index{reconstruct\_Fourier2d\_finite\_kernel() (SIMulator.SIMulator method)@\spxentry{reconstruct\_Fourier2d\_finite\_kernel()}\spxextra{SIMulator.SIMulator method}}

\begin{fulllineitems}
\phantomsection\label{\detokenize{source/SIMulator:SIMulator.SIMulator.reconstruct_Fourier2d_finite_kernel}}
\pysigstartsignatures
\pysiglinewithargsret
{\sphinxbfcode{\sphinxupquote{reconstruct\_Fourier2d\_finite\_kernel}}}
{\sphinxparam{\DUrole{n}{sim\_images}}\sphinxparamcomma \sphinxparam{\DUrole{n}{shifted\_kernels}}}
{}
\pysigstopsignatures
\end{fulllineitems}

\index{reconstruct\_Fourier\_space() (SIMulator.SIMulator method)@\spxentry{reconstruct\_Fourier\_space()}\spxextra{SIMulator.SIMulator method}}

\begin{fulllineitems}
\phantomsection\label{\detokenize{source/SIMulator:SIMulator.SIMulator.reconstruct_Fourier_space}}
\pysigstartsignatures
\pysiglinewithargsret
{\sphinxbfcode{\sphinxupquote{reconstruct\_Fourier\_space}}}
{\sphinxparam{\DUrole{n}{sim\_images}}}
{}
\pysigstopsignatures
\end{fulllineitems}

\index{reconstruct\_real2d\_finite\_kernel() (SIMulator.SIMulator method)@\spxentry{reconstruct\_real2d\_finite\_kernel()}\spxextra{SIMulator.SIMulator method}}

\begin{fulllineitems}
\phantomsection\label{\detokenize{source/SIMulator:SIMulator.SIMulator.reconstruct_real2d_finite_kernel}}
\pysigstartsignatures
\pysiglinewithargsret
{\sphinxbfcode{\sphinxupquote{reconstruct\_real2d\_finite\_kernel}}}
{\sphinxparam{\DUrole{n}{sim\_images}}\sphinxparamcomma \sphinxparam{\DUrole{n}{kernel}}\sphinxparamcomma \sphinxparam{\DUrole{n}{mode}\DUrole{o}{=}\DUrole{default_value}{\textquotesingle{}same\textquotesingle{}}}}
{}
\pysigstopsignatures
\end{fulllineitems}

\index{reconstruct\_real\_space() (SIMulator.SIMulator method)@\spxentry{reconstruct\_real\_space()}\spxextra{SIMulator.SIMulator method}}

\begin{fulllineitems}
\phantomsection\label{\detokenize{source/SIMulator:SIMulator.SIMulator.reconstruct_real_space}}
\pysigstartsignatures
\pysiglinewithargsret
{\sphinxbfcode{\sphinxupquote{reconstruct\_real\_space}}}
{\sphinxparam{\DUrole{n}{sim\_images}}\sphinxparamcomma \sphinxparam{\DUrole{n}{mode}\DUrole{o}{=}\DUrole{default_value}{\textquotesingle{}same\textquotesingle{}}}}
{}
\pysigstopsignatures
\end{fulllineitems}


\end{fulllineitems}


\sphinxstepscope


\subsection{SSNRBasedFiltering module}
\label{\detokenize{source/SSNRBasedFiltering:module-SSNRBasedFiltering}}\label{\detokenize{source/SSNRBasedFiltering:ssnrbasedfiltering-module}}\label{\detokenize{source/SSNRBasedFiltering::doc}}\index{module@\spxentry{module}!SSNRBasedFiltering@\spxentry{SSNRBasedFiltering}}\index{SSNRBasedFiltering@\spxentry{SSNRBasedFiltering}!module@\spxentry{module}}
\sphinxAtStartPar
SSNRBasedFiltering.py

\sphinxAtStartPar
This module contains classes for filtering images based on their total SSNR.

\sphinxAtStartPar
The detailed documentation for this class will be provided in the further release.

\sphinxincludegraphics[]{inheritance-74f08172f840e3f52607028348826ab4e0d89744.pdf}
\index{FlatNoiseFilter3d (class in SSNRBasedFiltering)@\spxentry{FlatNoiseFilter3d}\spxextra{class in SSNRBasedFiltering}}

\begin{fulllineitems}
\phantomsection\label{\detokenize{source/SSNRBasedFiltering:SSNRBasedFiltering.FlatNoiseFilter3d}}
\pysigstartsignatures
\pysiglinewithargsret
{\sphinxbfcode{\sphinxupquote{class\DUrole{w}{ }}}\sphinxcode{\sphinxupquote{SSNRBasedFiltering.}}\sphinxbfcode{\sphinxupquote{FlatNoiseFilter3d}}}
{\sphinxparam{\DUrole{n}{ssnr\_calculator}}\sphinxparamcomma \sphinxparam{\DUrole{n}{apodization\_filter}\DUrole{o}{=}\DUrole{default_value}{1}}}
{}
\pysigstopsignatures
\sphinxAtStartPar
Bases: \sphinxcode{\sphinxupquote{object}}

\sphinxincludegraphics[]{inheritance-12cadffb2a7b738baf209ef1e7128c2be9e829cb.pdf}
\index{filter\_object() (SSNRBasedFiltering.FlatNoiseFilter3d method)@\spxentry{filter\_object()}\spxextra{SSNRBasedFiltering.FlatNoiseFilter3d method}}

\begin{fulllineitems}
\phantomsection\label{\detokenize{source/SSNRBasedFiltering:SSNRBasedFiltering.FlatNoiseFilter3d.filter_object}}
\pysigstartsignatures
\pysiglinewithargsret
{\sphinxbfcode{\sphinxupquote{filter\_object}}}
{\sphinxparam{\DUrole{n}{object}}\sphinxparamcomma \sphinxparam{\DUrole{n}{real\_space}\DUrole{o}{=}\DUrole{default_value}{True}}}
{}
\pysigstopsignatures
\end{fulllineitems}


\end{fulllineitems}

\index{FlatNoiseFilter3dModel (class in SSNRBasedFiltering)@\spxentry{FlatNoiseFilter3dModel}\spxextra{class in SSNRBasedFiltering}}

\begin{fulllineitems}
\phantomsection\label{\detokenize{source/SSNRBasedFiltering:SSNRBasedFiltering.FlatNoiseFilter3dModel}}
\pysigstartsignatures
\pysiglinewithargsret
{\sphinxbfcode{\sphinxupquote{class\DUrole{w}{ }}}\sphinxcode{\sphinxupquote{SSNRBasedFiltering.}}\sphinxbfcode{\sphinxupquote{FlatNoiseFilter3dModel}}}
{\sphinxparam{\DUrole{n}{ssnr\_calculator}}\sphinxparamcomma \sphinxparam{\DUrole{n}{apodization\_filter}\DUrole{o}{=}\DUrole{default_value}{1}}}
{}
\pysigstopsignatures
\sphinxAtStartPar
Bases: {\hyperref[\detokenize{source/SSNRBasedFiltering:SSNRBasedFiltering.FlatNoiseFilter3d}]{\sphinxcrossref{\sphinxcode{\sphinxupquote{FlatNoiseFilter3d}}}}}

\sphinxincludegraphics[]{inheritance-f733c003342b2ae9d655e08576941e69ff32feaa.pdf}
\index{filter\_object() (SSNRBasedFiltering.FlatNoiseFilter3dModel method)@\spxentry{filter\_object()}\spxextra{SSNRBasedFiltering.FlatNoiseFilter3dModel method}}

\begin{fulllineitems}
\phantomsection\label{\detokenize{source/SSNRBasedFiltering:SSNRBasedFiltering.FlatNoiseFilter3dModel.filter_object}}
\pysigstartsignatures
\pysiglinewithargsret
{\sphinxbfcode{\sphinxupquote{filter\_object}}}
{\sphinxparam{\DUrole{n}{model\_object}}\sphinxparamcomma \sphinxparam{\DUrole{n}{real\_space}\DUrole{o}{=}\DUrole{default_value}{True}}}
{}
\pysigstopsignatures
\end{fulllineitems}


\end{fulllineitems}

\index{WienerFilter3d (class in SSNRBasedFiltering)@\spxentry{WienerFilter3d}\spxextra{class in SSNRBasedFiltering}}

\begin{fulllineitems}
\phantomsection\label{\detokenize{source/SSNRBasedFiltering:SSNRBasedFiltering.WienerFilter3d}}
\pysigstartsignatures
\pysiglinewithargsret
{\sphinxbfcode{\sphinxupquote{class\DUrole{w}{ }}}\sphinxcode{\sphinxupquote{SSNRBasedFiltering.}}\sphinxbfcode{\sphinxupquote{WienerFilter3d}}}
{\sphinxparam{\DUrole{n}{ssnr\_calculator}}\sphinxparamcomma \sphinxparam{\DUrole{n}{apodization\_filter}\DUrole{o}{=}\DUrole{default_value}{1}}}
{}
\pysigstopsignatures
\sphinxAtStartPar
Bases: \sphinxcode{\sphinxupquote{object}}

\sphinxincludegraphics[]{inheritance-87483a5930bb904bfcfbbb317a02882419e00bef.pdf}
\index{filter\_object() (SSNRBasedFiltering.WienerFilter3d method)@\spxentry{filter\_object()}\spxextra{SSNRBasedFiltering.WienerFilter3d method}}

\begin{fulllineitems}
\phantomsection\label{\detokenize{source/SSNRBasedFiltering:SSNRBasedFiltering.WienerFilter3d.filter_object}}
\pysigstartsignatures
\pysiglinewithargsret
{\sphinxbfcode{\sphinxupquote{filter\_object}}}
{\sphinxparam{\DUrole{n}{object}}\sphinxparamcomma \sphinxparam{\DUrole{n}{real\_space}\DUrole{o}{=}\DUrole{default_value}{True}}}
{}
\pysigstopsignatures
\end{fulllineitems}


\end{fulllineitems}

\index{WienerFilter3dModel (class in SSNRBasedFiltering)@\spxentry{WienerFilter3dModel}\spxextra{class in SSNRBasedFiltering}}

\begin{fulllineitems}
\phantomsection\label{\detokenize{source/SSNRBasedFiltering:SSNRBasedFiltering.WienerFilter3dModel}}
\pysigstartsignatures
\pysiglinewithargsret
{\sphinxbfcode{\sphinxupquote{class\DUrole{w}{ }}}\sphinxcode{\sphinxupquote{SSNRBasedFiltering.}}\sphinxbfcode{\sphinxupquote{WienerFilter3dModel}}}
{\sphinxparam{\DUrole{n}{ssnr\_calculator}}\sphinxparamcomma \sphinxparam{\DUrole{n}{apodization\_filter}\DUrole{o}{=}\DUrole{default_value}{1}}}
{}
\pysigstopsignatures
\sphinxAtStartPar
Bases: {\hyperref[\detokenize{source/SSNRBasedFiltering:SSNRBasedFiltering.WienerFilter3d}]{\sphinxcrossref{\sphinxcode{\sphinxupquote{WienerFilter3d}}}}}

\sphinxincludegraphics[]{inheritance-32cd44f9c19aacca061c8c4e00255969f53d77eb.pdf}
\index{filter\_object() (SSNRBasedFiltering.WienerFilter3dModel method)@\spxentry{filter\_object()}\spxextra{SSNRBasedFiltering.WienerFilter3dModel method}}

\begin{fulllineitems}
\phantomsection\label{\detokenize{source/SSNRBasedFiltering:SSNRBasedFiltering.WienerFilter3dModel.filter_object}}
\pysigstartsignatures
\pysiglinewithargsret
{\sphinxbfcode{\sphinxupquote{filter\_object}}}
{\sphinxparam{\DUrole{n}{model\_object}}\sphinxparamcomma \sphinxparam{\DUrole{n}{real\_space}\DUrole{o}{=}\DUrole{default_value}{True}}}
{}
\pysigstopsignatures
\end{fulllineitems}


\end{fulllineitems}

\index{WienerFilter3dModelSDR (class in SSNRBasedFiltering)@\spxentry{WienerFilter3dModelSDR}\spxextra{class in SSNRBasedFiltering}}

\begin{fulllineitems}
\phantomsection\label{\detokenize{source/SSNRBasedFiltering:SSNRBasedFiltering.WienerFilter3dModelSDR}}
\pysigstartsignatures
\pysiglinewithargsret
{\sphinxbfcode{\sphinxupquote{class\DUrole{w}{ }}}\sphinxcode{\sphinxupquote{SSNRBasedFiltering.}}\sphinxbfcode{\sphinxupquote{WienerFilter3dModelSDR}}}
{\sphinxparam{\DUrole{n}{ssnr\_calculator}}\sphinxparamcomma \sphinxparam{\DUrole{n}{apodization\_filter}\DUrole{o}{=}\DUrole{default_value}{1}}}
{}
\pysigstopsignatures
\sphinxAtStartPar
Bases: {\hyperref[\detokenize{source/SSNRBasedFiltering:SSNRBasedFiltering.WienerFilter3dModel}]{\sphinxcrossref{\sphinxcode{\sphinxupquote{WienerFilter3dModel}}}}}

\sphinxincludegraphics[]{inheritance-670d444aa1b96474f54e32b1d1b95bb9449ef6ec.pdf}
\index{filter\_SDR\_reconstruction() (SSNRBasedFiltering.WienerFilter3dModelSDR method)@\spxentry{filter\_SDR\_reconstruction()}\spxextra{SSNRBasedFiltering.WienerFilter3dModelSDR method}}

\begin{fulllineitems}
\phantomsection\label{\detokenize{source/SSNRBasedFiltering:SSNRBasedFiltering.WienerFilter3dModelSDR.filter_SDR_reconstruction}}
\pysigstartsignatures
\pysiglinewithargsret
{\sphinxbfcode{\sphinxupquote{filter\_SDR\_reconstruction}}}
{\sphinxparam{\DUrole{n}{object}}\sphinxparamcomma \sphinxparam{\DUrole{n}{reconstruction}}}
{}
\pysigstopsignatures
\end{fulllineitems}


\end{fulllineitems}

\index{WienerFilter3dReconstruction (class in SSNRBasedFiltering)@\spxentry{WienerFilter3dReconstruction}\spxextra{class in SSNRBasedFiltering}}

\begin{fulllineitems}
\phantomsection\label{\detokenize{source/SSNRBasedFiltering:SSNRBasedFiltering.WienerFilter3dReconstruction}}
\pysigstartsignatures
\pysiglinewithargsret
{\sphinxbfcode{\sphinxupquote{class\DUrole{w}{ }}}\sphinxcode{\sphinxupquote{SSNRBasedFiltering.}}\sphinxbfcode{\sphinxupquote{WienerFilter3dReconstruction}}}
{\sphinxparam{\DUrole{n}{ssnr\_calculator}}\sphinxparamcomma \sphinxparam{\DUrole{n}{apodization\_filter}\DUrole{o}{=}\DUrole{default_value}{1}}}
{}
\pysigstopsignatures
\sphinxAtStartPar
Bases: {\hyperref[\detokenize{source/SSNRBasedFiltering:SSNRBasedFiltering.WienerFilter3d}]{\sphinxcrossref{\sphinxcode{\sphinxupquote{WienerFilter3d}}}}}

\sphinxincludegraphics[]{inheritance-ad4f8ddb138c96c0b910c2c22f9256775d94bd02.pdf}
\index{filter\_object() (SSNRBasedFiltering.WienerFilter3dReconstruction method)@\spxentry{filter\_object()}\spxextra{SSNRBasedFiltering.WienerFilter3dReconstruction method}}

\begin{fulllineitems}
\phantomsection\label{\detokenize{source/SSNRBasedFiltering:SSNRBasedFiltering.WienerFilter3dReconstruction.filter_object}}
\pysigstartsignatures
\pysiglinewithargsret
{\sphinxbfcode{\sphinxupquote{filter\_object}}}
{\sphinxparam{\DUrole{n}{reconstruction}}\sphinxparamcomma \sphinxparam{\DUrole{n}{real\_space}\DUrole{o}{=}\DUrole{default_value}{True}}\sphinxparamcomma \sphinxparam{\DUrole{n}{average}\DUrole{o}{=}\DUrole{default_value}{\textquotesingle{}surface\_levels\_3d\textquotesingle{}}}}
{}
\pysigstopsignatures
\end{fulllineitems}


\end{fulllineitems}


\sphinxstepscope


\subsection{SSNRCalculator module}
\label{\detokenize{source/SSNRCalculator:module-SSNRCalculator}}\label{\detokenize{source/SSNRCalculator:ssnrcalculator-module}}\label{\detokenize{source/SSNRCalculator::doc}}\index{module@\spxentry{module}!SSNRCalculator@\spxentry{SSNRCalculator}}\index{SSNRCalculator@\spxentry{SSNRCalculator}!module@\spxentry{module}}
\sphinxAtStartPar
SSNRCalculator.py

\sphinxAtStartPar
This module contains classes for calculating the (image\sphinxhyphen{}independent) spectral signal\sphinxhyphen{}to\sphinxhyphen{}noise ratio (SSNR)
for a given system optical system and illumination.

\sphinxAtStartPar
Mathematical details will be provided in the later documentation versions and in the corresponding papers.

\sphinxincludegraphics[]{inheritance-1105d05c6319398f64f40a6557198480a463fa14.pdf}
\index{SSNR2dSIM (class in SSNRCalculator)@\spxentry{SSNR2dSIM}\spxextra{class in SSNRCalculator}}

\begin{fulllineitems}
\phantomsection\label{\detokenize{source/SSNRCalculator:SSNRCalculator.SSNR2dSIM}}
\pysigstartsignatures
\pysiglinewithargsret
{\sphinxbfcode{\sphinxupquote{class\DUrole{w}{ }}}\sphinxcode{\sphinxupquote{SSNRCalculator.}}\sphinxbfcode{\sphinxupquote{SSNR2dSIM}}}
{\sphinxparam{\DUrole{n}{illumination}}\sphinxparamcomma \sphinxparam{\DUrole{n}{optical\_system}}\sphinxparamcomma \sphinxparam{\DUrole{n}{readout\_noise\_variance}\DUrole{o}{=}\DUrole{default_value}{0}}}
{}
\pysigstopsignatures
\sphinxAtStartPar
Bases: {\hyperref[\detokenize{source/SSNRCalculator:SSNRCalculator.SSNRCalculator}]{\sphinxcrossref{\sphinxcode{\sphinxupquote{SSNRCalculator}}}}}

\sphinxincludegraphics[]{inheritance-b09a0bbe97299e567bbf41dda8e03d38eda64ab0.pdf}
\index{ring\_average\_ssnr() (SSNRCalculator.SSNR2dSIM method)@\spxentry{ring\_average\_ssnr()}\spxextra{SSNRCalculator.SSNR2dSIM method}}

\begin{fulllineitems}
\phantomsection\label{\detokenize{source/SSNRCalculator:SSNRCalculator.SSNR2dSIM.ring_average_ssnr}}
\pysigstartsignatures
\pysiglinewithargsret
{\sphinxbfcode{\sphinxupquote{ring\_average\_ssnr}}}
{\sphinxparam{\DUrole{n}{number\_of\_samples}\DUrole{o}{=}\DUrole{default_value}{None}}}
{}
\pysigstopsignatures
\end{fulllineitems}


\end{fulllineitems}

\index{SSNR2dSIMFiniteKernel (class in SSNRCalculator)@\spxentry{SSNR2dSIMFiniteKernel}\spxextra{class in SSNRCalculator}}

\begin{fulllineitems}
\phantomsection\label{\detokenize{source/SSNRCalculator:SSNRCalculator.SSNR2dSIMFiniteKernel}}
\pysigstartsignatures
\pysiglinewithargsret
{\sphinxbfcode{\sphinxupquote{class\DUrole{w}{ }}}\sphinxcode{\sphinxupquote{SSNRCalculator.}}\sphinxbfcode{\sphinxupquote{SSNR2dSIMFiniteKernel}}}
{\sphinxparam{\DUrole{n}{illumination}}\sphinxparamcomma \sphinxparam{\DUrole{n}{optical\_system}}\sphinxparamcomma \sphinxparam{\DUrole{n}{kernel}}\sphinxparamcomma \sphinxparam{\DUrole{n}{readout\_noise\_variance}\DUrole{o}{=}\DUrole{default_value}{0}}}
{}
\pysigstopsignatures
\sphinxAtStartPar
Bases: {\hyperref[\detokenize{source/SSNRCalculator:SSNRCalculator.SSNR2dSIM}]{\sphinxcrossref{\sphinxcode{\sphinxupquote{SSNR2dSIM}}}}}

\sphinxincludegraphics[]{inheritance-83a28a6ee409219b15cd8575469157a8e99bca9e.pdf}
\index{illumination (SSNRCalculator.SSNR2dSIMFiniteKernel property)@\spxentry{illumination}\spxextra{SSNRCalculator.SSNR2dSIMFiniteKernel property}}

\begin{fulllineitems}
\phantomsection\label{\detokenize{source/SSNRCalculator:SSNRCalculator.SSNR2dSIMFiniteKernel.illumination}}
\pysigstartsignatures
\pysigline
{\sphinxbfcode{\sphinxupquote{property\DUrole{w}{ }}}\sphinxbfcode{\sphinxupquote{illumination}}}
\pysigstopsignatures
\end{fulllineitems}

\index{kernel (SSNRCalculator.SSNR2dSIMFiniteKernel property)@\spxentry{kernel}\spxextra{SSNRCalculator.SSNR2dSIMFiniteKernel property}}

\begin{fulllineitems}
\phantomsection\label{\detokenize{source/SSNRCalculator:SSNRCalculator.SSNR2dSIMFiniteKernel.kernel}}
\pysigstartsignatures
\pysigline
{\sphinxbfcode{\sphinxupquote{property\DUrole{w}{ }}}\sphinxbfcode{\sphinxupquote{kernel}}}
\pysigstopsignatures
\end{fulllineitems}

\index{plot\_effective\_kernel\_and\_otf() (SSNRCalculator.SSNR2dSIMFiniteKernel method)@\spxentry{plot\_effective\_kernel\_and\_otf()}\spxextra{SSNRCalculator.SSNR2dSIMFiniteKernel method}}

\begin{fulllineitems}
\phantomsection\label{\detokenize{source/SSNRCalculator:SSNRCalculator.SSNR2dSIMFiniteKernel.plot_effective_kernel_and_otf}}
\pysigstartsignatures
\pysiglinewithargsret
{\sphinxbfcode{\sphinxupquote{plot\_effective\_kernel\_and\_otf}}}
{}
{}
\pysigstopsignatures
\end{fulllineitems}


\end{fulllineitems}

\index{SSNR3dSIM2dShifts (class in SSNRCalculator)@\spxentry{SSNR3dSIM2dShifts}\spxextra{class in SSNRCalculator}}

\begin{fulllineitems}
\phantomsection\label{\detokenize{source/SSNRCalculator:SSNRCalculator.SSNR3dSIM2dShifts}}
\pysigstartsignatures
\pysiglinewithargsret
{\sphinxbfcode{\sphinxupquote{class\DUrole{w}{ }}}\sphinxcode{\sphinxupquote{SSNRCalculator.}}\sphinxbfcode{\sphinxupquote{SSNR3dSIM2dShifts}}}
{\sphinxparam{\DUrole{n}{illumination}}\sphinxparamcomma \sphinxparam{\DUrole{n}{optical\_system}}\sphinxparamcomma \sphinxparam{\DUrole{n}{readout\_noise\_variance}\DUrole{o}{=}\DUrole{default_value}{0}}}
{}
\pysigstopsignatures
\sphinxAtStartPar
Bases: {\hyperref[\detokenize{source/SSNRCalculator:SSNRCalculator.SSNR3dSIMBase}]{\sphinxcrossref{\sphinxcode{\sphinxupquote{SSNR3dSIMBase}}}}}

\sphinxincludegraphics[]{inheritance-e2a511edb94d321cd43c6cc485f8edbeec6d9f37.pdf}

\end{fulllineitems}

\index{SSNR3dSIM2dShiftsFiniteKernel (class in SSNRCalculator)@\spxentry{SSNR3dSIM2dShiftsFiniteKernel}\spxextra{class in SSNRCalculator}}

\begin{fulllineitems}
\phantomsection\label{\detokenize{source/SSNRCalculator:SSNRCalculator.SSNR3dSIM2dShiftsFiniteKernel}}
\pysigstartsignatures
\pysiglinewithargsret
{\sphinxbfcode{\sphinxupquote{class\DUrole{w}{ }}}\sphinxcode{\sphinxupquote{SSNRCalculator.}}\sphinxbfcode{\sphinxupquote{SSNR3dSIM2dShiftsFiniteKernel}}}
{\sphinxparam{\DUrole{n}{illumination}}\sphinxparamcomma \sphinxparam{\DUrole{n}{optical\_system}}\sphinxparamcomma \sphinxparam{\DUrole{n}{kernel}}\sphinxparamcomma \sphinxparam{\DUrole{n}{readout\_noise\_variance}\DUrole{o}{=}\DUrole{default_value}{0}}}
{}
\pysigstopsignatures
\sphinxAtStartPar
Bases: {\hyperref[\detokenize{source/SSNRCalculator:SSNRCalculator.SSNR3dSIM2dShifts}]{\sphinxcrossref{\sphinxcode{\sphinxupquote{SSNR3dSIM2dShifts}}}}}

\sphinxincludegraphics[]{inheritance-4e56612b5a826ef2fbee7369ef4d9ebbc115a57d.pdf}
\index{illumination (SSNRCalculator.SSNR3dSIM2dShiftsFiniteKernel property)@\spxentry{illumination}\spxextra{SSNRCalculator.SSNR3dSIM2dShiftsFiniteKernel property}}

\begin{fulllineitems}
\phantomsection\label{\detokenize{source/SSNRCalculator:SSNRCalculator.SSNR3dSIM2dShiftsFiniteKernel.illumination}}
\pysigstartsignatures
\pysigline
{\sphinxbfcode{\sphinxupquote{property\DUrole{w}{ }}}\sphinxbfcode{\sphinxupquote{illumination}}}
\pysigstopsignatures
\end{fulllineitems}

\index{kernel (SSNRCalculator.SSNR3dSIM2dShiftsFiniteKernel property)@\spxentry{kernel}\spxextra{SSNRCalculator.SSNR3dSIM2dShiftsFiniteKernel property}}

\begin{fulllineitems}
\phantomsection\label{\detokenize{source/SSNRCalculator:SSNRCalculator.SSNR3dSIM2dShiftsFiniteKernel.kernel}}
\pysigstartsignatures
\pysigline
{\sphinxbfcode{\sphinxupquote{property\DUrole{w}{ }}}\sphinxbfcode{\sphinxupquote{kernel}}}
\pysigstopsignatures
\end{fulllineitems}

\index{plot\_effective\_kernel\_and\_otf() (SSNRCalculator.SSNR3dSIM2dShiftsFiniteKernel method)@\spxentry{plot\_effective\_kernel\_and\_otf()}\spxextra{SSNRCalculator.SSNR3dSIM2dShiftsFiniteKernel method}}

\begin{fulllineitems}
\phantomsection\label{\detokenize{source/SSNRCalculator:SSNRCalculator.SSNR3dSIM2dShiftsFiniteKernel.plot_effective_kernel_and_otf}}
\pysigstartsignatures
\pysiglinewithargsret
{\sphinxbfcode{\sphinxupquote{plot\_effective\_kernel\_and\_otf}}}
{}
{}
\pysigstopsignatures
\end{fulllineitems}


\end{fulllineitems}

\index{SSNR3dSIM3dShifts (class in SSNRCalculator)@\spxentry{SSNR3dSIM3dShifts}\spxextra{class in SSNRCalculator}}

\begin{fulllineitems}
\phantomsection\label{\detokenize{source/SSNRCalculator:SSNRCalculator.SSNR3dSIM3dShifts}}
\pysigstartsignatures
\pysiglinewithargsret
{\sphinxbfcode{\sphinxupquote{class\DUrole{w}{ }}}\sphinxcode{\sphinxupquote{SSNRCalculator.}}\sphinxbfcode{\sphinxupquote{SSNR3dSIM3dShifts}}}
{\sphinxparam{\DUrole{n}{illumination}}\sphinxparamcomma \sphinxparam{\DUrole{n}{optical\_system}}\sphinxparamcomma \sphinxparam{\DUrole{n}{readout\_noise\_variance}\DUrole{o}{=}\DUrole{default_value}{0}}}
{}
\pysigstopsignatures
\sphinxAtStartPar
Bases: {\hyperref[\detokenize{source/SSNRCalculator:SSNRCalculator.SSNR3dSIMBase}]{\sphinxcrossref{\sphinxcode{\sphinxupquote{SSNR3dSIMBase}}}}}

\sphinxincludegraphics[]{inheritance-ca185644f9ca65eaffc7ba48a93bf6a78b7a9218.pdf}

\end{fulllineitems}

\index{SSNRCalculator (class in SSNRCalculator)@\spxentry{SSNRCalculator}\spxextra{class in SSNRCalculator}}

\begin{fulllineitems}
\phantomsection\label{\detokenize{source/SSNRCalculator:SSNRCalculator.SSNRCalculator}}
\pysigstartsignatures
\pysiglinewithargsret
{\sphinxbfcode{\sphinxupquote{class\DUrole{w}{ }}}\sphinxcode{\sphinxupquote{SSNRCalculator.}}\sphinxbfcode{\sphinxupquote{SSNRCalculator}}}
{\sphinxparam{\DUrole{n}{illumination}}\sphinxparamcomma \sphinxparam{\DUrole{n}{optical\_system}}\sphinxparamcomma \sphinxparam{\DUrole{n}{readout\_noise\_variance}\DUrole{o}{=}\DUrole{default_value}{0}}}
{}
\pysigstopsignatures
\sphinxAtStartPar
Bases: {\hyperref[\detokenize{source/SSNRCalculator:SSNRCalculator.SSNRBase}]{\sphinxcrossref{\sphinxcode{\sphinxupquote{SSNRBase}}}}}

\sphinxincludegraphics[]{inheritance-43033ce8baf98921d187eb37f0404312ac7b864e.pdf}
\index{compute\_analytic\_ssnr\_volume() (SSNRCalculator.SSNRCalculator method)@\spxentry{compute\_analytic\_ssnr\_volume()}\spxextra{SSNRCalculator.SSNRCalculator method}}

\begin{fulllineitems}
\phantomsection\label{\detokenize{source/SSNRCalculator:SSNRCalculator.SSNRCalculator.compute_analytic_ssnr_volume}}
\pysigstartsignatures
\pysiglinewithargsret
{\sphinxbfcode{\sphinxupquote{compute\_analytic\_ssnr\_volume}}}
{\sphinxparam{\DUrole{n}{factor}\DUrole{o}{=}\DUrole{default_value}{10}}\sphinxparamcomma \sphinxparam{\DUrole{n}{volume\_element}\DUrole{o}{=}\DUrole{default_value}{1}}}
{}
\pysigstopsignatures
\end{fulllineitems}

\index{compute\_analytic\_total\_ssnr() (SSNRCalculator.SSNRCalculator method)@\spxentry{compute\_analytic\_total\_ssnr()}\spxextra{SSNRCalculator.SSNRCalculator method}}

\begin{fulllineitems}
\phantomsection\label{\detokenize{source/SSNRCalculator:SSNRCalculator.SSNRCalculator.compute_analytic_total_ssnr}}
\pysigstartsignatures
\pysiglinewithargsret
{\sphinxbfcode{\sphinxupquote{compute\_analytic\_total\_ssnr}}}
{\sphinxparam{\DUrole{n}{factor}\DUrole{o}{=}\DUrole{default_value}{10}}\sphinxparamcomma \sphinxparam{\DUrole{n}{volume\_element}\DUrole{o}{=}\DUrole{default_value}{1}}}
{}
\pysigstopsignatures
\end{fulllineitems}

\index{compute\_maximum\_resolved\_lateral() (SSNRCalculator.SSNRCalculator method)@\spxentry{compute\_maximum\_resolved\_lateral()}\spxextra{SSNRCalculator.SSNRCalculator method}}

\begin{fulllineitems}
\phantomsection\label{\detokenize{source/SSNRCalculator:SSNRCalculator.SSNRCalculator.compute_maximum_resolved_lateral}}
\pysigstartsignatures
\pysiglinewithargsret
{\sphinxbfcode{\sphinxupquote{compute\_maximum\_resolved\_lateral}}}
{}
{}
\pysigstopsignatures
\end{fulllineitems}

\index{compute\_ssnr() (SSNRCalculator.SSNRCalculator method)@\spxentry{compute\_ssnr()}\spxextra{SSNRCalculator.SSNRCalculator method}}

\begin{fulllineitems}
\phantomsection\label{\detokenize{source/SSNRCalculator:SSNRCalculator.SSNRCalculator.compute_ssnr}}
\pysigstartsignatures
\pysiglinewithargsret
{\sphinxbfcode{\sphinxupquote{compute\_ssnr}}}
{}
{}
\pysigstopsignatures
\end{fulllineitems}

\index{compute\_ssnr\_waterline\_measure() (SSNRCalculator.SSNRCalculator method)@\spxentry{compute\_ssnr\_waterline\_measure()}\spxextra{SSNRCalculator.SSNRCalculator method}}

\begin{fulllineitems}
\phantomsection\label{\detokenize{source/SSNRCalculator:SSNRCalculator.SSNRCalculator.compute_ssnr_waterline_measure}}
\pysigstartsignatures
\pysiglinewithargsret
{\sphinxbfcode{\sphinxupquote{compute\_ssnr\_waterline\_measure}}}
{\sphinxparam{\DUrole{n}{factor}\DUrole{o}{=}\DUrole{default_value}{10}}}
{}
\pysigstopsignatures
\end{fulllineitems}

\index{compute\_total\_ssnr() (SSNRCalculator.SSNRCalculator method)@\spxentry{compute\_total\_ssnr()}\spxextra{SSNRCalculator.SSNRCalculator method}}

\begin{fulllineitems}
\phantomsection\label{\detokenize{source/SSNRCalculator:SSNRCalculator.SSNRCalculator.compute_total_ssnr}}
\pysigstartsignatures
\pysiglinewithargsret
{\sphinxbfcode{\sphinxupquote{compute\_total\_ssnr}}}
{\sphinxparam{\DUrole{n}{factor}\DUrole{o}{=}\DUrole{default_value}{10}}\sphinxparamcomma \sphinxparam{\DUrole{n}{volume\_element}\DUrole{o}{=}\DUrole{default_value}{1}}}
{}
\pysigstopsignatures
\end{fulllineitems}

\index{illumination (SSNRCalculator.SSNRCalculator property)@\spxentry{illumination}\spxextra{SSNRCalculator.SSNRCalculator property}}

\begin{fulllineitems}
\phantomsection\label{\detokenize{source/SSNRCalculator:SSNRCalculator.SSNRCalculator.illumination}}
\pysigstartsignatures
\pysigline
{\sphinxbfcode{\sphinxupquote{property\DUrole{w}{ }}}\sphinxbfcode{\sphinxupquote{illumination}}}
\pysigstopsignatures
\end{fulllineitems}

\index{optical\_system (SSNRCalculator.SSNRCalculator property)@\spxentry{optical\_system}\spxextra{SSNRCalculator.SSNRCalculator property}}

\begin{fulllineitems}
\phantomsection\label{\detokenize{source/SSNRCalculator:SSNRCalculator.SSNRCalculator.optical_system}}
\pysigstartsignatures
\pysigline
{\sphinxbfcode{\sphinxupquote{property\DUrole{w}{ }}}\sphinxbfcode{\sphinxupquote{optical\_system}}}
\pysigstopsignatures
\end{fulllineitems}


\end{fulllineitems}

\index{SSNR3dSIMBase (class in SSNRCalculator)@\spxentry{SSNR3dSIMBase}\spxextra{class in SSNRCalculator}}

\begin{fulllineitems}
\phantomsection\label{\detokenize{source/SSNRCalculator:SSNRCalculator.SSNR3dSIMBase}}
\pysigstartsignatures
\pysiglinewithargsret
{\sphinxbfcode{\sphinxupquote{class\DUrole{w}{ }}}\sphinxcode{\sphinxupquote{SSNRCalculator.}}\sphinxbfcode{\sphinxupquote{SSNR3dSIMBase}}}
{\sphinxparam{\DUrole{n}{illumination}}\sphinxparamcomma \sphinxparam{\DUrole{n}{optical\_system}}\sphinxparamcomma \sphinxparam{\DUrole{n}{readout\_noise\_variance}\DUrole{o}{=}\DUrole{default_value}{0}}}
{}
\pysigstopsignatures
\sphinxAtStartPar
Bases: {\hyperref[\detokenize{source/SSNRCalculator:SSNRCalculator.SSNRCalculator}]{\sphinxcrossref{\sphinxcode{\sphinxupquote{SSNRCalculator}}}}}

\sphinxincludegraphics[]{inheritance-25ee8d2008d5f44ea9379c578fbe7df5e03d2cf8.pdf}

\end{fulllineitems}

\index{SSNRConfocal (class in SSNRCalculator)@\spxentry{SSNRConfocal}\spxextra{class in SSNRCalculator}}

\begin{fulllineitems}
\phantomsection\label{\detokenize{source/SSNRCalculator:SSNRCalculator.SSNRConfocal}}
\pysigstartsignatures
\pysiglinewithargsret
{\sphinxbfcode{\sphinxupquote{class\DUrole{w}{ }}}\sphinxcode{\sphinxupquote{SSNRCalculator.}}\sphinxbfcode{\sphinxupquote{SSNRConfocal}}}
{\sphinxparam{\DUrole{n}{optical\_system}}}
{}
\pysigstopsignatures
\sphinxAtStartPar
Bases: {\hyperref[\detokenize{source/SSNRCalculator:SSNRCalculator.SSNRBase}]{\sphinxcrossref{\sphinxcode{\sphinxupquote{SSNRBase}}}}}

\sphinxincludegraphics[]{inheritance-1f59de299a176fa06ff5f8de6a3a77e5cdb4d4c0.pdf}
\index{compute\_ssnr() (SSNRCalculator.SSNRConfocal method)@\spxentry{compute\_ssnr()}\spxextra{SSNRCalculator.SSNRConfocal method}}

\begin{fulllineitems}
\phantomsection\label{\detokenize{source/SSNRCalculator:SSNRCalculator.SSNRConfocal.compute_ssnr}}
\pysigstartsignatures
\pysiglinewithargsret
{\sphinxbfcode{\sphinxupquote{compute\_ssnr}}}
{}
{}
\pysigstopsignatures
\end{fulllineitems}


\end{fulllineitems}

\index{SSNRBase (class in SSNRCalculator)@\spxentry{SSNRBase}\spxextra{class in SSNRCalculator}}

\begin{fulllineitems}
\phantomsection\label{\detokenize{source/SSNRCalculator:SSNRCalculator.SSNRBase}}
\pysigstartsignatures
\pysiglinewithargsret
{\sphinxbfcode{\sphinxupquote{class\DUrole{w}{ }}}\sphinxcode{\sphinxupquote{SSNRCalculator.}}\sphinxbfcode{\sphinxupquote{SSNRBase}}}
{\sphinxparam{\DUrole{n}{optical\_system}}}
{}
\pysigstopsignatures
\sphinxAtStartPar
Bases: \sphinxcode{\sphinxupquote{object}}

\sphinxincludegraphics[]{inheritance-77e7655ddb0281e7fe5fcedc5b72713f0d40e202.pdf}
\index{compute\_radial\_ssnr\_entropy() (SSNRCalculator.SSNRBase method)@\spxentry{compute\_radial\_ssnr\_entropy()}\spxextra{SSNRCalculator.SSNRBase method}}

\begin{fulllineitems}
\phantomsection\label{\detokenize{source/SSNRCalculator:SSNRCalculator.SSNRBase.compute_radial_ssnr_entropy}}
\pysigstartsignatures
\pysiglinewithargsret
{\sphinxbfcode{\sphinxupquote{compute\_radial\_ssnr\_entropy}}}
{\sphinxparam{\DUrole{n}{factor}\DUrole{o}{=}\DUrole{default_value}{100}}}
{}
\pysigstopsignatures
\end{fulllineitems}

\index{compute\_ssnr() (SSNRCalculator.SSNRBase method)@\spxentry{compute\_ssnr()}\spxextra{SSNRCalculator.SSNRBase method}}

\begin{fulllineitems}
\phantomsection\label{\detokenize{source/SSNRCalculator:SSNRCalculator.SSNRBase.compute_ssnr}}
\pysigstartsignatures
\pysiglinewithargsret
{\sphinxbfcode{\sphinxupquote{abstract\DUrole{w}{ }}}\sphinxbfcode{\sphinxupquote{compute\_ssnr}}}
{}
{}
\pysigstopsignatures
\end{fulllineitems}

\index{compute\_ssnr\_volume() (SSNRCalculator.SSNRBase method)@\spxentry{compute\_ssnr\_volume()}\spxextra{SSNRCalculator.SSNRBase method}}

\begin{fulllineitems}
\phantomsection\label{\detokenize{source/SSNRCalculator:SSNRCalculator.SSNRBase.compute_ssnr_volume}}
\pysigstartsignatures
\pysiglinewithargsret
{\sphinxbfcode{\sphinxupquote{compute\_ssnr\_volume}}}
{\sphinxparam{\DUrole{n}{factor}\DUrole{o}{=}\DUrole{default_value}{10}}\sphinxparamcomma \sphinxparam{\DUrole{n}{volume\_element}\DUrole{o}{=}\DUrole{default_value}{1}}}
{}
\pysigstopsignatures
\end{fulllineitems}

\index{compute\_true\_ssnr\_entropy() (SSNRCalculator.SSNRBase method)@\spxentry{compute\_true\_ssnr\_entropy()}\spxextra{SSNRCalculator.SSNRBase method}}

\begin{fulllineitems}
\phantomsection\label{\detokenize{source/SSNRCalculator:SSNRCalculator.SSNRBase.compute_true_ssnr_entropy}}
\pysigstartsignatures
\pysiglinewithargsret
{\sphinxbfcode{\sphinxupquote{compute\_true\_ssnr\_entropy}}}
{\sphinxparam{\DUrole{n}{factor}\DUrole{o}{=}\DUrole{default_value}{100}}}
{}
\pysigstopsignatures
\end{fulllineitems}

\index{optical\_system (SSNRCalculator.SSNRBase property)@\spxentry{optical\_system}\spxextra{SSNRCalculator.SSNRBase property}}

\begin{fulllineitems}
\phantomsection\label{\detokenize{source/SSNRCalculator:SSNRCalculator.SSNRBase.optical_system}}
\pysigstartsignatures
\pysigline
{\sphinxbfcode{\sphinxupquote{property\DUrole{w}{ }}}\sphinxbfcode{\sphinxupquote{optical\_system}}}
\pysigstopsignatures
\end{fulllineitems}

\index{ring\_average\_ssnr() (SSNRCalculator.SSNRBase method)@\spxentry{ring\_average\_ssnr()}\spxextra{SSNRCalculator.SSNRBase method}}

\begin{fulllineitems}
\phantomsection\label{\detokenize{source/SSNRCalculator:SSNRCalculator.SSNRBase.ring_average_ssnr}}
\pysigstartsignatures
\pysiglinewithargsret
{\sphinxbfcode{\sphinxupquote{abstract\DUrole{w}{ }}}\sphinxbfcode{\sphinxupquote{ring\_average\_ssnr}}}
{\sphinxparam{\DUrole{n}{number\_of\_samples}\DUrole{o}{=}\DUrole{default_value}{None}}}
{}
\pysigstopsignatures
\end{fulllineitems}


\end{fulllineitems}

\index{SSNRWidefield (class in SSNRCalculator)@\spxentry{SSNRWidefield}\spxextra{class in SSNRCalculator}}

\begin{fulllineitems}
\phantomsection\label{\detokenize{source/SSNRCalculator:SSNRCalculator.SSNRWidefield}}
\pysigstartsignatures
\pysiglinewithargsret
{\sphinxbfcode{\sphinxupquote{class\DUrole{w}{ }}}\sphinxcode{\sphinxupquote{SSNRCalculator.}}\sphinxbfcode{\sphinxupquote{SSNRWidefield}}}
{\sphinxparam{\DUrole{n}{optical\_system}}}
{}
\pysigstopsignatures
\sphinxAtStartPar
Bases: {\hyperref[\detokenize{source/SSNRCalculator:SSNRCalculator.SSNRBase}]{\sphinxcrossref{\sphinxcode{\sphinxupquote{SSNRBase}}}}}

\sphinxincludegraphics[]{inheritance-85a15641a4d04fd2c6e2ff6b400118211e61bd3e.pdf}
\index{compute\_ssnr() (SSNRCalculator.SSNRWidefield method)@\spxentry{compute\_ssnr()}\spxextra{SSNRCalculator.SSNRWidefield method}}

\begin{fulllineitems}
\phantomsection\label{\detokenize{source/SSNRCalculator:SSNRCalculator.SSNRWidefield.compute_ssnr}}
\pysigstartsignatures
\pysiglinewithargsret
{\sphinxbfcode{\sphinxupquote{compute\_ssnr}}}
{}
{}
\pysigstopsignatures
\end{fulllineitems}


\end{fulllineitems}


\sphinxstepscope


\subsection{ShapesGenerator module}
\label{\detokenize{source/ShapesGenerator:module-ShapesGenerator}}\label{\detokenize{source/ShapesGenerator:shapesgenerator-module}}\label{\detokenize{source/ShapesGenerator::doc}}\index{module@\spxentry{module}!ShapesGenerator@\spxentry{ShapesGenerator}}\index{ShapesGenerator@\spxentry{ShapesGenerator}!module@\spxentry{module}}
\sphinxAtStartPar
ShapesGenerator.py

\sphinxAtStartPar
This module contains functions for generating various simulated images used in simulations.
\index{generate\_random\_lines() (in module ShapesGenerator)@\spxentry{generate\_random\_lines()}\spxextra{in module ShapesGenerator}}

\begin{fulllineitems}
\phantomsection\label{\detokenize{source/ShapesGenerator:ShapesGenerator.generate_random_lines}}
\pysigstartsignatures
\pysiglinewithargsret
{\sphinxcode{\sphinxupquote{ShapesGenerator.}}\sphinxbfcode{\sphinxupquote{generate\_random\_lines}}}
{\sphinxparam{\DUrole{n}{image\_size}\DUrole{p}{:}\DUrole{w}{ }\DUrole{n}{tuple\DUrole{p}{{[}}int\DUrole{p}{,}\DUrole{w}{ }int\DUrole{p}{,}\DUrole{w}{ }int\DUrole{p}{{]}}}}\sphinxparamcomma \sphinxparam{\DUrole{n}{point\_number}\DUrole{p}{:}\DUrole{w}{ }\DUrole{n}{int}}\sphinxparamcomma \sphinxparam{\DUrole{n}{line\_width}\DUrole{p}{:}\DUrole{w}{ }\DUrole{n}{float}}\sphinxparamcomma \sphinxparam{\DUrole{n}{num\_lines}\DUrole{p}{:}\DUrole{w}{ }\DUrole{n}{int}}\sphinxparamcomma \sphinxparam{\DUrole{n}{intensity}\DUrole{p}{:}\DUrole{w}{ }\DUrole{n}{float}}}
{{ $\rightarrow$ ndarray}}
\pysigstopsignatures
\sphinxAtStartPar
Generate an image with randomly oriented lines.
\begin{quote}\begin{description}
\sphinxlineitem{Parameters}\begin{itemize}
\item {} 
\sphinxAtStartPar
\sphinxstyleliteralstrong{\sphinxupquote{point\_number}} \textendash{} Number of points defining the size of the image grid (image will be point\_number x point\_number).

\item {} 
\sphinxAtStartPar
\sphinxstyleliteralstrong{\sphinxupquote{image\_size}} \textendash{} Tuple of (psf\_x\_size, psf\_y\_size) defining scaling in x and y directions.

\item {} 
\sphinxAtStartPar
\sphinxstyleliteralstrong{\sphinxupquote{line\_width}} \textendash{} Width of the lines.

\item {} 
\sphinxAtStartPar
\sphinxstyleliteralstrong{\sphinxupquote{num\_lines}} \textendash{} Number of lines to generate.

\item {} 
\sphinxAtStartPar
\sphinxstyleliteralstrong{\sphinxupquote{intensity}} \textendash{} Total intensity of each line.

\end{itemize}

\sphinxlineitem{Returns}
\sphinxAtStartPar
Generated image with lines.

\end{description}\end{quote}

\end{fulllineitems}

\index{generate\_random\_spheres() (in module ShapesGenerator)@\spxentry{generate\_random\_spheres()}\spxextra{in module ShapesGenerator}}

\begin{fulllineitems}
\phantomsection\label{\detokenize{source/ShapesGenerator:ShapesGenerator.generate_random_spherical_particles}}
\pysigstartsignatures
\pysiglinewithargsret
{\sphinxcode{\sphinxupquote{ShapesGenerator.}}\sphinxbfcode{\sphinxupquote{generate\_random\_spheres}}}
{\sphinxparam{\DUrole{n}{image\_size}\DUrole{p}{:}\DUrole{w}{ }\DUrole{n}{tuple\DUrole{p}{{[}}int\DUrole{p}{,}\DUrole{w}{ }int\DUrole{p}{,}\DUrole{w}{ }int\DUrole{p}{{]}}}}\sphinxparamcomma \sphinxparam{\DUrole{n}{point\_number}\DUrole{p}{:}\DUrole{w}{ }\DUrole{n}{int}}\sphinxparamcomma \sphinxparam{\DUrole{n}{r}\DUrole{o}{=}\DUrole{default_value}{0.1}}\sphinxparamcomma \sphinxparam{\DUrole{n}{N}\DUrole{o}{=}\DUrole{default_value}{10}}\sphinxparamcomma \sphinxparam{\DUrole{n}{I}\DUrole{o}{=}\DUrole{default_value}{1000}}}
{{ $\rightarrow$ ndarray}}
\pysigstopsignatures
\sphinxAtStartPar
Generates an array with random spheres.
\begin{quote}\begin{description}
\sphinxlineitem{Parameters}\begin{itemize}
\item {} 
\sphinxAtStartPar
\sphinxstyleliteralstrong{\sphinxupquote{image\_size}} (\sphinxstyleliteralemphasis{\sphinxupquote{tuple}}\sphinxstyleliteralemphasis{\sphinxupquote{{[}}}\sphinxstyleliteralemphasis{\sphinxupquote{int}}\sphinxstyleliteralemphasis{\sphinxupquote{, }}\sphinxstyleliteralemphasis{\sphinxupquote{int}}\sphinxstyleliteralemphasis{\sphinxupquote{, }}\sphinxstyleliteralemphasis{\sphinxupquote{int}}\sphinxstyleliteralemphasis{\sphinxupquote{{]}}}) \textendash{} Size of the point spread function in each dimension.

\item {} 
\sphinxAtStartPar
\sphinxstyleliteralstrong{\sphinxupquote{point\_number}} (\sphinxstyleliteralemphasis{\sphinxupquote{int}}) \textendash{} Number of points in each dimension.

\item {} 
\sphinxAtStartPar
\sphinxstyleliteralstrong{\sphinxupquote{r}} (\sphinxstyleliteralemphasis{\sphinxupquote{float}}\sphinxstyleliteralemphasis{\sphinxupquote{, }}\sphinxstyleliteralemphasis{\sphinxupquote{optional}}) \textendash{} Radius of the spheres. Defaults to 0.1.

\item {} 
\sphinxAtStartPar
\sphinxstyleliteralstrong{\sphinxupquote{N}} (\sphinxstyleliteralemphasis{\sphinxupquote{int}}\sphinxstyleliteralemphasis{\sphinxupquote{, }}\sphinxstyleliteralemphasis{\sphinxupquote{optional}}) \textendash{} Number of spheres to generate. Defaults to 10.

\item {} 
\sphinxAtStartPar
\sphinxstyleliteralstrong{\sphinxupquote{I}} (\sphinxstyleliteralemphasis{\sphinxupquote{int}}\sphinxstyleliteralemphasis{\sphinxupquote{, }}\sphinxstyleliteralemphasis{\sphinxupquote{optional}}) \textendash{} Intensity of the spheres. Defaults to 1000.

\end{itemize}

\sphinxlineitem{Returns}
\sphinxAtStartPar
Array with random spheres.

\sphinxlineitem{Return type}
\sphinxAtStartPar
np.ndarray

\end{description}\end{quote}

\end{fulllineitems}

\index{generate\_sphere\_slices() (in module ShapesGenerator)@\spxentry{generate\_sphere\_slices()}\spxextra{in module ShapesGenerator}}

\begin{fulllineitems}
\phantomsection\label{\detokenize{source/ShapesGenerator:ShapesGenerator.generate_sphere_slices}}
\pysigstartsignatures
\pysiglinewithargsret
{\sphinxcode{\sphinxupquote{ShapesGenerator.}}\sphinxbfcode{\sphinxupquote{generate\_sphere\_slices}}}
{\sphinxparam{\DUrole{n}{image\_size}\DUrole{p}{:}\DUrole{w}{ }\DUrole{n}{tuple\DUrole{p}{{[}}int\DUrole{p}{,}\DUrole{w}{ }int\DUrole{p}{,}\DUrole{w}{ }int\DUrole{p}{{]}}}}\sphinxparamcomma \sphinxparam{\DUrole{n}{point\_number}\DUrole{p}{:}\DUrole{w}{ }\DUrole{n}{int}}\sphinxparamcomma \sphinxparam{\DUrole{n}{r}\DUrole{o}{=}\DUrole{default_value}{0.1}}\sphinxparamcomma \sphinxparam{\DUrole{n}{N}\DUrole{o}{=}\DUrole{default_value}{10}}\sphinxparamcomma \sphinxparam{\DUrole{n}{I}\DUrole{o}{=}\DUrole{default_value}{1000}}}
{{ $\rightarrow$ ndarray}}
\pysigstopsignatures
\sphinxAtStartPar
Generates a thin slice with random spheres.
\begin{quote}\begin{description}
\sphinxlineitem{Parameters}\begin{itemize}
\item {} 
\sphinxAtStartPar
\sphinxstyleliteralstrong{\sphinxupquote{image\_size}} (\sphinxstyleliteralemphasis{\sphinxupquote{tuple}}\sphinxstyleliteralemphasis{\sphinxupquote{{[}}}\sphinxstyleliteralemphasis{\sphinxupquote{int}}\sphinxstyleliteralemphasis{\sphinxupquote{, }}\sphinxstyleliteralemphasis{\sphinxupquote{int}}\sphinxstyleliteralemphasis{\sphinxupquote{, }}\sphinxstyleliteralemphasis{\sphinxupquote{int}}\sphinxstyleliteralemphasis{\sphinxupquote{{]}}}) \textendash{} Size of the point spread function in each dimension.

\item {} 
\sphinxAtStartPar
\sphinxstyleliteralstrong{\sphinxupquote{point\_number}} (\sphinxstyleliteralemphasis{\sphinxupquote{int}}) \textendash{} Number of points in each dimension.

\item {} 
\sphinxAtStartPar
\sphinxstyleliteralstrong{\sphinxupquote{r}} (\sphinxstyleliteralemphasis{\sphinxupquote{float}}\sphinxstyleliteralemphasis{\sphinxupquote{, }}\sphinxstyleliteralemphasis{\sphinxupquote{optional}}) \textendash{} Radius of the spheres. Defaults to 0.1.

\item {} 
\sphinxAtStartPar
\sphinxstyleliteralstrong{\sphinxupquote{N}} (\sphinxstyleliteralemphasis{\sphinxupquote{int}}\sphinxstyleliteralemphasis{\sphinxupquote{, }}\sphinxstyleliteralemphasis{\sphinxupquote{optional}}) \textendash{} Number of spheres to generate. Defaults to 10.

\item {} 
\sphinxAtStartPar
\sphinxstyleliteralstrong{\sphinxupquote{I}} (\sphinxstyleliteralemphasis{\sphinxupquote{int}}\sphinxstyleliteralemphasis{\sphinxupquote{, }}\sphinxstyleliteralemphasis{\sphinxupquote{optional}}) \textendash{} Intensity of the spheres. Defaults to 1000.

\end{itemize}

\sphinxlineitem{Returns}
\sphinxAtStartPar
A thin slice of random spheres.

\sphinxlineitem{Return type}
\sphinxAtStartPar
np.ndarray

\end{description}\end{quote}

\end{fulllineitems}


\sphinxstepscope


\subsection{Sources module}
\label{\detokenize{source/Sources:module-Sources}}\label{\detokenize{source/Sources:sources-module}}\label{\detokenize{source/Sources::doc}}\index{module@\spxentry{module}!Sources@\spxentry{Sources}}\index{Sources@\spxentry{Sources}!module@\spxentry{module}}
\sphinxAtStartPar
Sources.py

\sphinxAtStartPar
This module contains classes for different types of sources used in simulations.
The sources can provide either electric fields or intensity fields.

\sphinxincludegraphics[]{inheritance-2f53957d00ee2820bc298f81de6d8cab3aebfa21.pdf}
\index{ElectricFieldSource (class in Sources)@\spxentry{ElectricFieldSource}\spxextra{class in Sources}}

\begin{fulllineitems}
\phantomsection\label{\detokenize{source/Sources:Sources.ElectricFieldSource}}
\pysigstartsignatures
\pysigline
{\sphinxbfcode{\sphinxupquote{class\DUrole{w}{ }}}\sphinxcode{\sphinxupquote{Sources.}}\sphinxbfcode{\sphinxupquote{ElectricFieldSource}}}
\pysigstopsignatures
\sphinxAtStartPar
Bases: {\hyperref[\detokenize{source/Sources:Sources.Source}]{\sphinxcrossref{\sphinxcode{\sphinxupquote{Source}}}}}

\sphinxAtStartPar
Abstract base class for sources that provide an electric field.

\sphinxincludegraphics[]{inheritance-96ffbb5e743244384d146a29a0ff6b1f42227c6c.pdf}
\index{get\_electric\_field() (Sources.ElectricFieldSource method)@\spxentry{get\_electric\_field()}\spxextra{Sources.ElectricFieldSource method}}

\begin{fulllineitems}
\phantomsection\label{\detokenize{source/Sources:Sources.ElectricFieldSource.get_electric_field}}
\pysigstartsignatures
\pysiglinewithargsret
{\sphinxbfcode{\sphinxupquote{abstract\DUrole{w}{ }}}\sphinxbfcode{\sphinxupquote{get\_electric\_field}}}
{\sphinxparam{\DUrole{n}{coordinates}\DUrole{p}{:}\DUrole{w}{ }\DUrole{n}{float64}}}
{{ $\rightarrow$ complex128}}
\pysigstopsignatures
\sphinxAtStartPar
Gets the electric field at the given coordinates.
\begin{quote}\begin{description}
\sphinxlineitem{Parameters}
\sphinxAtStartPar
\sphinxstyleliteralstrong{\sphinxupquote{coordinates}} (\sphinxstyleliteralemphasis{\sphinxupquote{numpy.ndarray}}\sphinxstyleliteralemphasis{\sphinxupquote{{[}}}\sphinxstyleliteralemphasis{\sphinxupquote{np.float64}}\sphinxstyleliteralemphasis{\sphinxupquote{{]}}}) \textendash{} The coordinates at which to get the electric field.

\sphinxlineitem{Returns}
\sphinxAtStartPar
The electric field at the given coordinates.

\sphinxlineitem{Return type}
\sphinxAtStartPar
numpy.ndarray{[}np.complex128{]}

\end{description}\end{quote}

\end{fulllineitems}

\index{get\_source\_type() (Sources.ElectricFieldSource method)@\spxentry{get\_source\_type()}\spxextra{Sources.ElectricFieldSource method}}

\begin{fulllineitems}
\phantomsection\label{\detokenize{source/Sources:Sources.ElectricFieldSource.get_source_type}}
\pysigstartsignatures
\pysiglinewithargsret
{\sphinxbfcode{\sphinxupquote{get\_source\_type}}}
{}
{{ $\rightarrow$ str}}
\pysigstopsignatures\begin{description}
\sphinxlineitem{Returns a type of the source in a human\sphinxhyphen{}readable form.}
\sphinxAtStartPar
str: The type of the source.

\end{description}

\end{fulllineitems}


\end{fulllineitems}

\index{IntensityPlaneWave (class in Sources)@\spxentry{IntensityPlaneWave}\spxextra{class in Sources}}

\begin{fulllineitems}
\phantomsection\label{\detokenize{source/Sources:Sources.IntensityPlaneWave}}
\pysigstartsignatures
\pysiglinewithargsret
{\sphinxbfcode{\sphinxupquote{class\DUrole{w}{ }}}\sphinxcode{\sphinxupquote{Sources.}}\sphinxbfcode{\sphinxupquote{IntensityPlaneWave}}}
{\sphinxparam{\DUrole{n}{amplitude}\DUrole{o}{=}\DUrole{default_value}{0.0}}\sphinxparamcomma \sphinxparam{\DUrole{n}{phase}\DUrole{o}{=}\DUrole{default_value}{0.0}}\sphinxparamcomma \sphinxparam{\DUrole{n}{wavevector}\DUrole{o}{=}\DUrole{default_value}{array({[}0., 0., 0.{]})}}}
{}
\pysigstopsignatures
\sphinxAtStartPar
Bases: {\hyperref[\detokenize{source/Sources:Sources.IntensitySource}]{\sphinxcrossref{\sphinxcode{\sphinxupquote{IntensitySource}}}}}

\sphinxAtStartPar
Intensity plane wave is a component of the Fourier
transform of the energy density distribution in a given volume
(e.g., standing waves)

\sphinxincludegraphics[]{inheritance-febfdabe5c9e2daa1937733ef98118c5cb3b3650.pdf}
\index{get\_intensity() (Sources.IntensityPlaneWave method)@\spxentry{get\_intensity()}\spxextra{Sources.IntensityPlaneWave method}}

\begin{fulllineitems}
\phantomsection\label{\detokenize{source/Sources:Sources.IntensityPlaneWave.get_intensity}}
\pysigstartsignatures
\pysiglinewithargsret
{\sphinxbfcode{\sphinxupquote{get\_intensity}}}
{\sphinxparam{\DUrole{n}{coordinates}\DUrole{p}{:}\DUrole{w}{ }\DUrole{n}{float64}}}
{}
\pysigstopsignatures
\sphinxAtStartPar
Gets the intensity at the given coordinates.
\begin{quote}\begin{description}
\sphinxlineitem{Parameters}
\sphinxAtStartPar
\sphinxstyleliteralstrong{\sphinxupquote{coordinates}} (\sphinxstyleliteralemphasis{\sphinxupquote{numpy.ndarray}}\sphinxstyleliteralemphasis{\sphinxupquote{{[}}}\sphinxstyleliteralemphasis{\sphinxupquote{np.float64}}\sphinxstyleliteralemphasis{\sphinxupquote{{]}}}) \textendash{} The coordinates at which to get the intensity.

\sphinxlineitem{Returns}
\sphinxAtStartPar
The intensity at the given coordinates.

\sphinxlineitem{Return type}
\sphinxAtStartPar
numpy.ndarray{[}np.float64{]}

\end{description}\end{quote}

\end{fulllineitems}


\end{fulllineitems}

\index{IntensitySource (class in Sources)@\spxentry{IntensitySource}\spxextra{class in Sources}}

\begin{fulllineitems}
\phantomsection\label{\detokenize{source/Sources:Sources.IntensitySource}}
\pysigstartsignatures
\pysigline
{\sphinxbfcode{\sphinxupquote{class\DUrole{w}{ }}}\sphinxcode{\sphinxupquote{Sources.}}\sphinxbfcode{\sphinxupquote{IntensitySource}}}
\pysigstopsignatures
\sphinxAtStartPar
Bases: {\hyperref[\detokenize{source/Sources:Sources.Source}]{\sphinxcrossref{\sphinxcode{\sphinxupquote{Source}}}}}

\sphinxAtStartPar
Abstract base class for sources that provide intensity.

\sphinxincludegraphics[]{inheritance-ac6c5d8b894685e8d480f902d382d989027dbac5.pdf}
\index{get\_intensity() (Sources.IntensitySource method)@\spxentry{get\_intensity()}\spxextra{Sources.IntensitySource method}}

\begin{fulllineitems}
\phantomsection\label{\detokenize{source/Sources:Sources.IntensitySource.get_intensity}}
\pysigstartsignatures
\pysiglinewithargsret
{\sphinxbfcode{\sphinxupquote{abstract\DUrole{w}{ }}}\sphinxbfcode{\sphinxupquote{get\_intensity}}}
{\sphinxparam{\DUrole{n}{coordinates}\DUrole{p}{:}\DUrole{w}{ }\DUrole{n}{float64}}}
{{ $\rightarrow$ int64}}
\pysigstopsignatures
\sphinxAtStartPar
Gets the intensity at the given coordinates.
\begin{quote}\begin{description}
\sphinxlineitem{Parameters}
\sphinxAtStartPar
\sphinxstyleliteralstrong{\sphinxupquote{coordinates}} (\sphinxstyleliteralemphasis{\sphinxupquote{numpy.ndarray}}\sphinxstyleliteralemphasis{\sphinxupquote{{[}}}\sphinxstyleliteralemphasis{\sphinxupquote{np.float64}}\sphinxstyleliteralemphasis{\sphinxupquote{{]}}}) \textendash{} The coordinates at which to get the intensity.

\sphinxlineitem{Returns}
\sphinxAtStartPar
The intensity at the given coordinates.

\sphinxlineitem{Return type}
\sphinxAtStartPar
numpy.ndarray{[}np.float64{]}

\end{description}\end{quote}

\end{fulllineitems}

\index{get\_source\_type() (Sources.IntensitySource method)@\spxentry{get\_source\_type()}\spxextra{Sources.IntensitySource method}}

\begin{fulllineitems}
\phantomsection\label{\detokenize{source/Sources:Sources.IntensitySource.get_source_type}}
\pysigstartsignatures
\pysiglinewithargsret
{\sphinxbfcode{\sphinxupquote{get\_source\_type}}}
{}
{{ $\rightarrow$ str}}
\pysigstopsignatures\begin{description}
\sphinxlineitem{Returns a type of the source in a human\sphinxhyphen{}readable form.}
\sphinxAtStartPar
str: The type of the source.

\end{description}

\end{fulllineitems}


\end{fulllineitems}

\index{PlaneWave (class in Sources)@\spxentry{PlaneWave}\spxextra{class in Sources}}

\begin{fulllineitems}
\phantomsection\label{\detokenize{source/Sources:Sources.PlaneWave}}
\pysigstartsignatures
\pysiglinewithargsret
{\sphinxbfcode{\sphinxupquote{class\DUrole{w}{ }}}\sphinxcode{\sphinxupquote{Sources.}}\sphinxbfcode{\sphinxupquote{PlaneWave}}}
{\sphinxparam{\DUrole{n}{electric\_field\_p}\DUrole{p}{:}\DUrole{w}{ }\DUrole{n}{complex}}\sphinxparamcomma \sphinxparam{\DUrole{n}{electric\_field\_s}\DUrole{p}{:}\DUrole{w}{ }\DUrole{n}{complex}}\sphinxparamcomma \sphinxparam{\DUrole{n}{phase1}\DUrole{p}{:}\DUrole{w}{ }\DUrole{n}{float}}\sphinxparamcomma \sphinxparam{\DUrole{n}{phase2}\DUrole{p}{:}\DUrole{w}{ }\DUrole{n}{float}}\sphinxparamcomma \sphinxparam{\DUrole{n}{wavevector}\DUrole{p}{:}\DUrole{w}{ }\DUrole{n}{float64}}}
{}
\pysigstopsignatures
\sphinxAtStartPar
Bases: {\hyperref[\detokenize{source/Sources:Sources.ElectricFieldSource}]{\sphinxcrossref{\sphinxcode{\sphinxupquote{ElectricFieldSource}}}}}

\sphinxAtStartPar
Electric field of a plane wave

\sphinxincludegraphics[]{inheritance-7f819b8d403e12e979810885727f71739442b5cc.pdf}
\index{get\_electric\_field() (Sources.PlaneWave method)@\spxentry{get\_electric\_field()}\spxextra{Sources.PlaneWave method}}

\begin{fulllineitems}
\phantomsection\label{\detokenize{source/Sources:Sources.PlaneWave.get_electric_field}}
\pysigstartsignatures
\pysiglinewithargsret
{\sphinxbfcode{\sphinxupquote{get\_electric\_field}}}
{\sphinxparam{\DUrole{n}{coordinates}}}
{}
\pysigstopsignatures
\sphinxAtStartPar
Gets the electric field at the given coordinates.
\begin{quote}\begin{description}
\sphinxlineitem{Parameters}
\sphinxAtStartPar
\sphinxstyleliteralstrong{\sphinxupquote{coordinates}} (\sphinxstyleliteralemphasis{\sphinxupquote{numpy.ndarray}}\sphinxstyleliteralemphasis{\sphinxupquote{{[}}}\sphinxstyleliteralemphasis{\sphinxupquote{np.float64}}\sphinxstyleliteralemphasis{\sphinxupquote{{]}}}) \textendash{} The coordinates at which to get the electric field.

\sphinxlineitem{Returns}
\sphinxAtStartPar
The electric field at the given coordinates.

\sphinxlineitem{Return type}
\sphinxAtStartPar
numpy.ndarray{[}np.complex128{]}

\end{description}\end{quote}

\end{fulllineitems}


\end{fulllineitems}

\index{PointSource (class in Sources)@\spxentry{PointSource}\spxextra{class in Sources}}

\begin{fulllineitems}
\phantomsection\label{\detokenize{source/Sources:Sources.PointSource}}
\pysigstartsignatures
\pysiglinewithargsret
{\sphinxbfcode{\sphinxupquote{class\DUrole{w}{ }}}\sphinxcode{\sphinxupquote{Sources.}}\sphinxbfcode{\sphinxupquote{PointSource}}}
{\sphinxparam{\DUrole{n}{coordinates}\DUrole{p}{:}\DUrole{w}{ }\DUrole{n}{float64}}\sphinxparamcomma \sphinxparam{\DUrole{n}{brightness}\DUrole{p}{:}\DUrole{w}{ }\DUrole{n}{float}}}
{}
\pysigstopsignatures
\sphinxAtStartPar
Bases: {\hyperref[\detokenize{source/Sources:Sources.ElectricFieldSource}]{\sphinxcrossref{\sphinxcode{\sphinxupquote{ElectricFieldSource}}}}}

\sphinxAtStartPar
Electric field of a point source

\sphinxincludegraphics[]{inheritance-f54774aa0124f350d39c526611c02cd317b67f97.pdf}
\index{get\_electric\_field() (Sources.PointSource method)@\spxentry{get\_electric\_field()}\spxextra{Sources.PointSource method}}

\begin{fulllineitems}
\phantomsection\label{\detokenize{source/Sources:Sources.PointSource.get_electric_field}}
\pysigstartsignatures
\pysiglinewithargsret
{\sphinxbfcode{\sphinxupquote{get\_electric\_field}}}
{\sphinxparam{\DUrole{n}{coordinates}\DUrole{p}{:}\DUrole{w}{ }\DUrole{n}{float64}}}
{}
\pysigstopsignatures
\sphinxAtStartPar
Gets the electric field at the given coordinates.
\begin{quote}\begin{description}
\sphinxlineitem{Parameters}
\sphinxAtStartPar
\sphinxstyleliteralstrong{\sphinxupquote{coordinates}} (\sphinxstyleliteralemphasis{\sphinxupquote{numpy.ndarray}}\sphinxstyleliteralemphasis{\sphinxupquote{{[}}}\sphinxstyleliteralemphasis{\sphinxupquote{np.float64}}\sphinxstyleliteralemphasis{\sphinxupquote{{]}}}) \textendash{} The coordinates at which to get the electric field.

\sphinxlineitem{Returns}
\sphinxAtStartPar
The electric field at the given coordinates.

\sphinxlineitem{Return type}
\sphinxAtStartPar
numpy.ndarray{[}np.complex128{]}

\end{description}\end{quote}

\end{fulllineitems}


\end{fulllineitems}

\index{Source (class in Sources)@\spxentry{Source}\spxextra{class in Sources}}

\begin{fulllineitems}
\phantomsection\label{\detokenize{source/Sources:Sources.Source}}
\pysigstartsignatures
\pysigline
{\sphinxbfcode{\sphinxupquote{class\DUrole{w}{ }}}\sphinxcode{\sphinxupquote{Sources.}}\sphinxbfcode{\sphinxupquote{Source}}}
\pysigstopsignatures
\sphinxAtStartPar
Bases: \sphinxcode{\sphinxupquote{object}}

\sphinxAtStartPar
Abstract base class for sources of electric or intensity fields
in our simulations.

\sphinxincludegraphics[]{inheritance-ca48498ca76b130af71b0a40f435b1352ff929c8.pdf}
\index{get\_source\_type() (Sources.Source method)@\spxentry{get\_source\_type()}\spxextra{Sources.Source method}}

\begin{fulllineitems}
\phantomsection\label{\detokenize{source/Sources:Sources.Source.get_source_type}}
\pysigstartsignatures
\pysiglinewithargsret
{\sphinxbfcode{\sphinxupquote{abstract\DUrole{w}{ }}}\sphinxbfcode{\sphinxupquote{get\_source\_type}}}
{}
{{ $\rightarrow$ str}}
\pysigstopsignatures\begin{description}
\sphinxlineitem{Returns a type of the source in a human\sphinxhyphen{}readable form.}
\sphinxAtStartPar
str: The type of the source.

\end{description}

\end{fulllineitems}


\end{fulllineitems}


\sphinxstepscope


\subsection{VectorOperations module}
\label{\detokenize{source/VectorOperations:module-VectorOperations}}\label{\detokenize{source/VectorOperations:vectoroperations-module}}\label{\detokenize{source/VectorOperations::doc}}\index{module@\spxentry{module}!VectorOperations@\spxentry{VectorOperations}}\index{VectorOperations@\spxentry{VectorOperations}!module@\spxentry{module}}
\sphinxAtStartPar
VectorOperations.py

\sphinxAtStartPar
This module contains utility functions for vector operations.
\begin{description}
\sphinxlineitem{Classes:}
\sphinxAtStartPar
VectorOperations: Class containing static methods for various vector operations.

\end{description}

\sphinxincludegraphics[]{inheritance-f33274d68009d6e33f7f3433903496050e0e855b.pdf}
\index{VectorOperations (class in VectorOperations)@\spxentry{VectorOperations}\spxextra{class in VectorOperations}}

\begin{fulllineitems}
\phantomsection\label{\detokenize{source/VectorOperations:VectorOperations.VectorOperations}}
\pysigstartsignatures
\pysigline
{\sphinxbfcode{\sphinxupquote{class\DUrole{w}{ }}}\sphinxcode{\sphinxupquote{VectorOperations.}}\sphinxbfcode{\sphinxupquote{VectorOperations}}}
\pysigstopsignatures
\sphinxAtStartPar
Bases: \sphinxcode{\sphinxupquote{object}}

\sphinxincludegraphics[]{inheritance-06810f63033806977641213ee19704e78c8daa75.pdf}
\index{rotate\_vector2d() (VectorOperations.VectorOperations static method)@\spxentry{rotate\_vector2d()}\spxextra{VectorOperations.VectorOperations static method}}

\begin{fulllineitems}
\phantomsection\label{\detokenize{source/VectorOperations:VectorOperations.VectorOperations.rotate_vector2d}}
\pysigstartsignatures
\pysiglinewithargsret
{\sphinxbfcode{\sphinxupquote{static\DUrole{w}{ }}}\sphinxbfcode{\sphinxupquote{rotate\_vector2d}}}
{\sphinxparam{\DUrole{n}{vector2d}}\sphinxparamcomma \sphinxparam{\DUrole{n}{angle}}}
{}
\pysigstopsignatures
\end{fulllineitems}

\index{rotate\_vector3d() (VectorOperations.VectorOperations static method)@\spxentry{rotate\_vector3d()}\spxextra{VectorOperations.VectorOperations static method}}

\begin{fulllineitems}
\phantomsection\label{\detokenize{source/VectorOperations:VectorOperations.VectorOperations.rotate_vector3d}}
\pysigstartsignatures
\pysiglinewithargsret
{\sphinxbfcode{\sphinxupquote{static\DUrole{w}{ }}}\sphinxbfcode{\sphinxupquote{rotate\_vector3d}}}
{\sphinxparam{\DUrole{n}{vector3d}}\sphinxparamcomma \sphinxparam{\DUrole{n}{rot\_ax\_vector}}\sphinxparamcomma \sphinxparam{\DUrole{n}{rot\_angle}}}
{}
\pysigstopsignatures
\end{fulllineitems}

\index{rotation\_matrix() (VectorOperations.VectorOperations static method)@\spxentry{rotation\_matrix()}\spxextra{VectorOperations.VectorOperations static method}}

\begin{fulllineitems}
\phantomsection\label{\detokenize{source/VectorOperations:VectorOperations.VectorOperations.rotation_matrix}}
\pysigstartsignatures
\pysiglinewithargsret
{\sphinxbfcode{\sphinxupquote{static\DUrole{w}{ }}}\sphinxbfcode{\sphinxupquote{rotation\_matrix}}}
{\sphinxparam{\DUrole{n}{angle}}}
{}
\pysigstopsignatures
\end{fulllineitems}


\end{fulllineitems}


\sphinxstepscope


\subsection{Windowing module}
\label{\detokenize{source/Windowing:module-Windowing}}\label{\detokenize{source/Windowing:windowing-module}}\label{\detokenize{source/Windowing::doc}}\index{module@\spxentry{module}!Windowing@\spxentry{Windowing}}\index{Windowing@\spxentry{Windowing}!module@\spxentry{module}}
\sphinxAtStartPar
This module provides functions to modify the image near the edges for different purposes.
\index{make\_mask\_cosine\_edge2d() (in module Windowing)@\spxentry{make\_mask\_cosine\_edge2d()}\spxextra{in module Windowing}}

\begin{fulllineitems}
\phantomsection\label{\detokenize{source/Windowing:Windowing.make_mask_cosine_edge2d}}
\pysigstartsignatures
\pysiglinewithargsret
{\sphinxcode{\sphinxupquote{Windowing.}}\sphinxbfcode{\sphinxupquote{make\_mask\_cosine\_edge2d}}}
{\sphinxparam{\DUrole{n}{shape}\DUrole{p}{:}\DUrole{w}{ }\DUrole{n}{tuple\DUrole{p}{{[}}int\DUrole{p}{,}\DUrole{w}{ }int\DUrole{p}{{]}}}}\sphinxparamcomma \sphinxparam{\DUrole{n}{edge}\DUrole{p}{:}\DUrole{w}{ }\DUrole{n}{int}}}
{{ $\rightarrow$ ndarray}}
\pysigstopsignatures
\sphinxAtStartPar
2D Weight mask that vanishes with the cosine distance to the edge.
\begin{quote}\begin{description}
\sphinxlineitem{Parameters}\begin{itemize}
\item {} 
\sphinxAtStartPar
\sphinxstyleliteralstrong{\sphinxupquote{shape}} (\sphinxstyleliteralemphasis{\sphinxupquote{tuple}}\sphinxstyleliteralemphasis{\sphinxupquote{{[}}}\sphinxstyleliteralemphasis{\sphinxupquote{int}}\sphinxstyleliteralemphasis{\sphinxupquote{, }}\sphinxstyleliteralemphasis{\sphinxupquote{int}}\sphinxstyleliteralemphasis{\sphinxupquote{{]}}}) \textendash{} Shape of the mask.

\item {} 
\sphinxAtStartPar
\sphinxstyleliteralstrong{\sphinxupquote{edge}} (\sphinxstyleliteralemphasis{\sphinxupquote{int}}) \textendash{} Width of the edge.

\end{itemize}

\sphinxlineitem{Returns}
\sphinxAtStartPar
The mask.

\sphinxlineitem{Return type}
\sphinxAtStartPar
np.ndarray

\end{description}\end{quote}

\end{fulllineitems}

\index{make\_mask\_cosine\_edge3d() (in module Windowing)@\spxentry{make\_mask\_cosine\_edge3d()}\spxextra{in module Windowing}}

\begin{fulllineitems}
\phantomsection\label{\detokenize{source/Windowing:Windowing.make_mask_cosine_edge3d}}
\pysigstartsignatures
\pysiglinewithargsret
{\sphinxcode{\sphinxupquote{Windowing.}}\sphinxbfcode{\sphinxupquote{make\_mask\_cosine\_edge3d}}}
{\sphinxparam{\DUrole{n}{shape}\DUrole{p}{:}\DUrole{w}{ }\DUrole{n}{tuple\DUrole{p}{{[}}int\DUrole{p}{,}\DUrole{w}{ }int\DUrole{p}{,}\DUrole{w}{ }int\DUrole{p}{{]}}}}\sphinxparamcomma \sphinxparam{\DUrole{n}{edge}\DUrole{p}{:}\DUrole{w}{ }\DUrole{n}{int}}}
{{ $\rightarrow$ ndarray}}
\pysigstopsignatures
\sphinxAtStartPar
3D Weight mask that vanishes with the cosine distance to the edges.
\begin{quote}\begin{description}
\sphinxlineitem{Parameters}\begin{itemize}
\item {} 
\sphinxAtStartPar
\sphinxstyleliteralstrong{\sphinxupquote{shape}} (\sphinxstyleliteralemphasis{\sphinxupquote{tuple}}\sphinxstyleliteralemphasis{\sphinxupquote{{[}}}\sphinxstyleliteralemphasis{\sphinxupquote{int}}\sphinxstyleliteralemphasis{\sphinxupquote{, }}\sphinxstyleliteralemphasis{\sphinxupquote{int}}\sphinxstyleliteralemphasis{\sphinxupquote{, }}\sphinxstyleliteralemphasis{\sphinxupquote{int}}\sphinxstyleliteralemphasis{\sphinxupquote{{]}}}) \textendash{} Shape of the mask.

\item {} 
\sphinxAtStartPar
\sphinxstyleliteralstrong{\sphinxupquote{edge}} (\sphinxstyleliteralemphasis{\sphinxupquote{int}}) \textendash{} Width of the edge.

\end{itemize}

\sphinxlineitem{Returns}
\sphinxAtStartPar
The mask.

\sphinxlineitem{Return type}
\sphinxAtStartPar
np.ndarray

\end{description}\end{quote}

\end{fulllineitems}


\sphinxstepscope


\subsection{compute\_optimal\_lattices module}
\label{\detokenize{source/compute_optimal_lattices:module-compute_optimal_lattices}}\label{\detokenize{source/compute_optimal_lattices:compute-optimal-lattices-module}}\label{\detokenize{source/compute_optimal_lattices::doc}}\index{module@\spxentry{module}!compute\_optimal\_lattices@\spxentry{compute\_optimal\_lattices}}\index{compute\_optimal\_lattices@\spxentry{compute\_optimal\_lattices}!module@\spxentry{module}}
\sphinxAtStartPar
Yet not finalized module for computing one\sphinxhyphen{}dimension spatial shifts, satisfying the orthogonality condition.
Implemented for 2D and 3D lattices. The design is to be changed, thus no detailed documentation is provided.
\index{check\_peaks2d() (in module compute\_optimal\_lattices)@\spxentry{check\_peaks2d()}\spxextra{in module compute\_optimal\_lattices}}

\begin{fulllineitems}
\phantomsection\label{\detokenize{source/compute_optimal_lattices:compute_optimal_lattices.check_peaks2d}}
\pysigstartsignatures
\pysiglinewithargsret
{\sphinxcode{\sphinxupquote{compute\_optimal\_lattices.}}\sphinxbfcode{\sphinxupquote{check\_peaks2d}}}
{\sphinxparam{\DUrole{n}{matrix}}\sphinxparamcomma \sphinxparam{\DUrole{n}{peaks}}}
{}
\pysigstopsignatures
\end{fulllineitems}

\index{check\_peaks3d() (in module compute\_optimal\_lattices)@\spxentry{check\_peaks3d()}\spxextra{in module compute\_optimal\_lattices}}

\begin{fulllineitems}
\phantomsection\label{\detokenize{source/compute_optimal_lattices:compute_optimal_lattices.check_peaks3d}}
\pysigstartsignatures
\pysiglinewithargsret
{\sphinxcode{\sphinxupquote{compute\_optimal\_lattices.}}\sphinxbfcode{\sphinxupquote{check\_peaks3d}}}
{\sphinxparam{\DUrole{n}{matrix}}\sphinxparamcomma \sphinxparam{\DUrole{n}{peaks}}}
{}
\pysigstopsignatures
\end{fulllineitems}

\index{combine\_dict() (in module compute\_optimal\_lattices)@\spxentry{combine\_dict()}\spxextra{in module compute\_optimal\_lattices}}

\begin{fulllineitems}
\phantomsection\label{\detokenize{source/compute_optimal_lattices:compute_optimal_lattices.combine_dict}}
\pysigstartsignatures
\pysiglinewithargsret
{\sphinxcode{\sphinxupquote{compute\_optimal\_lattices.}}\sphinxbfcode{\sphinxupquote{combine\_dict}}}
{\sphinxparam{\DUrole{n}{d1}}\sphinxparamcomma \sphinxparam{\DUrole{n}{d2}}}
{}
\pysigstopsignatures
\end{fulllineitems}

\index{exponent\_sum2d() (in module compute\_optimal\_lattices)@\spxentry{exponent\_sum2d()}\spxextra{in module compute\_optimal\_lattices}}

\begin{fulllineitems}
\phantomsection\label{\detokenize{source/compute_optimal_lattices:compute_optimal_lattices.exponent_sum2d}}
\pysigstartsignatures
\pysiglinewithargsret
{\sphinxcode{\sphinxupquote{compute\_optimal\_lattices.}}\sphinxbfcode{\sphinxupquote{exponent\_sum2d}}}
{\sphinxparam{\DUrole{n}{matrix}}\sphinxparamcomma \sphinxparam{\DUrole{n}{Mx}}\sphinxparamcomma \sphinxparam{\DUrole{n}{My}}}
{}
\pysigstopsignatures
\end{fulllineitems}

\index{exponent\_sum3d() (in module compute\_optimal\_lattices)@\spxentry{exponent\_sum3d()}\spxextra{in module compute\_optimal\_lattices}}

\begin{fulllineitems}
\phantomsection\label{\detokenize{source/compute_optimal_lattices:compute_optimal_lattices.exponent_sum3d}}
\pysigstartsignatures
\pysiglinewithargsret
{\sphinxcode{\sphinxupquote{compute\_optimal\_lattices.}}\sphinxbfcode{\sphinxupquote{exponent\_sum3d}}}
{\sphinxparam{\DUrole{n}{matrix}}\sphinxparamcomma \sphinxparam{\DUrole{n}{Mx}}\sphinxparamcomma \sphinxparam{\DUrole{n}{My}}\sphinxparamcomma \sphinxparam{\DUrole{n}{Mz}}}
{}
\pysigstopsignatures
\end{fulllineitems}

\index{find\_pairs2d() (in module compute\_optimal\_lattices)@\spxentry{find\_pairs2d()}\spxextra{in module compute\_optimal\_lattices}}

\begin{fulllineitems}
\phantomsection\label{\detokenize{source/compute_optimal_lattices:compute_optimal_lattices.find_pairs2d}}
\pysigstartsignatures
\pysiglinewithargsret
{\sphinxcode{\sphinxupquote{compute\_optimal\_lattices.}}\sphinxbfcode{\sphinxupquote{find\_pairs2d}}}
{\sphinxparam{\DUrole{n}{table2d}}\sphinxparamcomma \sphinxparam{\DUrole{n}{modulos}}\sphinxparamcomma \sphinxparam{\DUrole{n}{power1}\DUrole{o}{=}\DUrole{default_value}{1}}}
{}
\pysigstopsignatures
\end{fulllineitems}

\index{find\_pairs3d() (in module compute\_optimal\_lattices)@\spxentry{find\_pairs3d()}\spxextra{in module compute\_optimal\_lattices}}

\begin{fulllineitems}
\phantomsection\label{\detokenize{source/compute_optimal_lattices:compute_optimal_lattices.find_pairs3d}}
\pysigstartsignatures
\pysiglinewithargsret
{\sphinxcode{\sphinxupquote{compute\_optimal\_lattices.}}\sphinxbfcode{\sphinxupquote{find\_pairs3d}}}
{\sphinxparam{\DUrole{n}{table3d}}\sphinxparamcomma \sphinxparam{\DUrole{n}{modulos}}\sphinxparamcomma \sphinxparam{\DUrole{n}{p1}\DUrole{o}{=}\DUrole{default_value}{1}}}
{}
\pysigstopsignatures
\end{fulllineitems}

\index{find\_pairs\_extended() (in module compute\_optimal\_lattices)@\spxentry{find\_pairs\_extended()}\spxextra{in module compute\_optimal\_lattices}}

\begin{fulllineitems}
\phantomsection\label{\detokenize{source/compute_optimal_lattices:compute_optimal_lattices.find_pairs_extended}}
\pysigstartsignatures
\pysiglinewithargsret
{\sphinxcode{\sphinxupquote{compute\_optimal\_lattices.}}\sphinxbfcode{\sphinxupquote{find\_pairs\_extended}}}
{\sphinxparam{\DUrole{n}{tables}}\sphinxparamcomma \sphinxparam{\DUrole{n}{modulos}}}
{}
\pysigstopsignatures
\end{fulllineitems}

\index{generate\_conditions2d() (in module compute\_optimal\_lattices)@\spxentry{generate\_conditions2d()}\spxextra{in module compute\_optimal\_lattices}}

\begin{fulllineitems}
\phantomsection\label{\detokenize{source/compute_optimal_lattices:compute_optimal_lattices.generate_conditions2d}}
\pysigstartsignatures
\pysiglinewithargsret
{\sphinxcode{\sphinxupquote{compute\_optimal\_lattices.}}\sphinxbfcode{\sphinxupquote{generate\_conditions2d}}}
{\sphinxparam{\DUrole{n}{peaks2d}}}
{}
\pysigstopsignatures
\end{fulllineitems}

\index{generate\_conditions3d() (in module compute\_optimal\_lattices)@\spxentry{generate\_conditions3d()}\spxextra{in module compute\_optimal\_lattices}}

\begin{fulllineitems}
\phantomsection\label{\detokenize{source/compute_optimal_lattices:compute_optimal_lattices.generate_conditions3d}}
\pysigstartsignatures
\pysiglinewithargsret
{\sphinxcode{\sphinxupquote{compute\_optimal\_lattices.}}\sphinxbfcode{\sphinxupquote{generate\_conditions3d}}}
{\sphinxparam{\DUrole{n}{peaks3d}}}
{}
\pysigstopsignatures
\end{fulllineitems}

\index{generate\_table2d() (in module compute\_optimal\_lattices)@\spxentry{generate\_table2d()}\spxextra{in module compute\_optimal\_lattices}}

\begin{fulllineitems}
\phantomsection\label{\detokenize{source/compute_optimal_lattices:compute_optimal_lattices.generate_table2d}}
\pysigstartsignatures
\pysiglinewithargsret
{\sphinxcode{\sphinxupquote{compute\_optimal\_lattices.}}\sphinxbfcode{\sphinxupquote{generate\_table2d}}}
{\sphinxparam{\DUrole{n}{funcs}}\sphinxparamcomma \sphinxparam{\DUrole{n}{bases}}\sphinxparamcomma \sphinxparam{\DUrole{n}{p1}\DUrole{o}{=}\DUrole{default_value}{1}}}
{}
\pysigstopsignatures
\end{fulllineitems}

\index{generate\_table3d() (in module compute\_optimal\_lattices)@\spxentry{generate\_table3d()}\spxextra{in module compute\_optimal\_lattices}}

\begin{fulllineitems}
\phantomsection\label{\detokenize{source/compute_optimal_lattices:compute_optimal_lattices.generate_table3d}}
\pysigstartsignatures
\pysiglinewithargsret
{\sphinxcode{\sphinxupquote{compute\_optimal\_lattices.}}\sphinxbfcode{\sphinxupquote{generate\_table3d}}}
{\sphinxparam{\DUrole{n}{funcs}}\sphinxparamcomma \sphinxparam{\DUrole{n}{bases}}\sphinxparamcomma \sphinxparam{\DUrole{n}{p1}\DUrole{o}{=}\DUrole{default_value}{1}}}
{}
\pysigstopsignatures
\end{fulllineitems}

\index{generate\_tables2d() (in module compute\_optimal\_lattices)@\spxentry{generate\_tables2d()}\spxextra{in module compute\_optimal\_lattices}}

\begin{fulllineitems}
\phantomsection\label{\detokenize{source/compute_optimal_lattices:compute_optimal_lattices.generate_tables2d}}
\pysigstartsignatures
\pysiglinewithargsret
{\sphinxcode{\sphinxupquote{compute\_optimal\_lattices.}}\sphinxbfcode{\sphinxupquote{generate\_tables2d}}}
{\sphinxparam{\DUrole{n}{funcs}}\sphinxparamcomma \sphinxparam{\DUrole{n}{max\_power}}}
{}
\pysigstopsignatures
\end{fulllineitems}

\index{get\_matrix2d() (in module compute\_optimal\_lattices)@\spxentry{get\_matrix2d()}\spxextra{in module compute\_optimal\_lattices}}

\begin{fulllineitems}
\phantomsection\label{\detokenize{source/compute_optimal_lattices:compute_optimal_lattices.get_matrix2d}}
\pysigstartsignatures
\pysiglinewithargsret
{\sphinxcode{\sphinxupquote{compute\_optimal\_lattices.}}\sphinxbfcode{\sphinxupquote{get\_matrix2d}}}
{\sphinxparam{\DUrole{n}{base}}\sphinxparamcomma \sphinxparam{\DUrole{n}{powers}}}
{}
\pysigstopsignatures
\end{fulllineitems}

\index{get\_matrix3d() (in module compute\_optimal\_lattices)@\spxentry{get\_matrix3d()}\spxextra{in module compute\_optimal\_lattices}}

\begin{fulllineitems}
\phantomsection\label{\detokenize{source/compute_optimal_lattices:compute_optimal_lattices.get_matrix3d}}
\pysigstartsignatures
\pysiglinewithargsret
{\sphinxcode{\sphinxupquote{compute\_optimal\_lattices.}}\sphinxbfcode{\sphinxupquote{get\_matrix3d}}}
{\sphinxparam{\DUrole{n}{base}}\sphinxparamcomma \sphinxparam{\DUrole{n}{powers}}}
{}
\pysigstopsignatures
\end{fulllineitems}


\sphinxstepscope


\subsection{confocal\_ssnr module}
\label{\detokenize{source/confocal_ssnr:module-confocal_ssnr}}\label{\detokenize{source/confocal_ssnr:confocal-ssnr-module}}\label{\detokenize{source/confocal_ssnr::doc}}\index{module@\spxentry{module}!confocal\_ssnr@\spxentry{confocal\_ssnr}}\index{confocal\_ssnr@\spxentry{confocal\_ssnr}!module@\spxentry{module}}
\sphinxAtStartPar
confocal\_ssnr.py

\sphinxAtStartPar
This script contains test computations of the SSNR in confocal microscopy, ISM and Rescan.

\sphinxincludegraphics[]{inheritance-6e5b1c20ff8ddc9946f50e2353d7f556ef3baa50.pdf}
\index{TestConfocalSSNR (class in confocal\_ssnr)@\spxentry{TestConfocalSSNR}\spxextra{class in confocal\_ssnr}}

\begin{fulllineitems}
\phantomsection\label{\detokenize{source/confocal_ssnr:confocal_ssnr.TestConfocalSSNR}}
\pysigstartsignatures
\pysiglinewithargsret
{\sphinxbfcode{\sphinxupquote{class\DUrole{w}{ }}}\sphinxcode{\sphinxupquote{confocal\_ssnr.}}\sphinxbfcode{\sphinxupquote{TestConfocalSSNR}}}
{\sphinxparam{\DUrole{n}{methodName}\DUrole{o}{=}\DUrole{default_value}{\textquotesingle{}runTest\textquotesingle{}}}}
{}
\pysigstopsignatures
\sphinxAtStartPar
Bases: \sphinxcode{\sphinxupquote{TestCase}}

\sphinxincludegraphics[]{inheritance-e72a08b398f8e3bae8b5a180c388d291d8ed8ca7.pdf}
\index{test\_SSNR2D() (confocal\_ssnr.TestConfocalSSNR method)@\spxentry{test\_SSNR2D()}\spxextra{confocal\_ssnr.TestConfocalSSNR method}}

\begin{fulllineitems}
\phantomsection\label{\detokenize{source/confocal_ssnr:confocal_ssnr.TestConfocalSSNR.test_SSNR2D}}
\pysigstartsignatures
\pysiglinewithargsret
{\sphinxbfcode{\sphinxupquote{test\_SSNR2D}}}
{}
{}
\pysigstopsignatures
\end{fulllineitems}

\index{test\_SSNR3D() (confocal\_ssnr.TestConfocalSSNR method)@\spxentry{test\_SSNR3D()}\spxextra{confocal\_ssnr.TestConfocalSSNR method}}

\begin{fulllineitems}
\phantomsection\label{\detokenize{source/confocal_ssnr:confocal_ssnr.TestConfocalSSNR.test_SSNR3D}}
\pysigstartsignatures
\pysiglinewithargsret
{\sphinxbfcode{\sphinxupquote{test\_SSNR3D}}}
{}
{}
\pysigstopsignatures
\end{fulllineitems}


\end{fulllineitems}


\sphinxstepscope


\subsection{globvar module}
\label{\detokenize{source/globvar:module-globvar}}\label{\detokenize{source/globvar:globvar-module}}\label{\detokenize{source/globvar::doc}}\index{module@\spxentry{module}!globvar@\spxentry{globvar}}\index{globvar@\spxentry{globvar}!module@\spxentry{module}}
\sphinxAtStartPar
globvar.py

\sphinxAtStartPar
This module contains global variables and constants used throughout the project.
It also contains physical constants and units for the case when calculations must be performed in SI units.
\begin{description}
\sphinxlineitem{Classes:}
\sphinxAtStartPar
Pauli: Class containing Pauli matrices.
SI: Class containing various physical constants and units.

\end{description}

\sphinxincludegraphics[]{inheritance-bc7b4bd71fb0fd72bfe8acb2943a197d8bc15539.pdf}
\index{Pauli (class in globvar)@\spxentry{Pauli}\spxextra{class in globvar}}

\begin{fulllineitems}
\phantomsection\label{\detokenize{source/globvar:globvar.Pauli}}
\pysigstartsignatures
\pysigline
{\sphinxbfcode{\sphinxupquote{class\DUrole{w}{ }}}\sphinxcode{\sphinxupquote{globvar.}}\sphinxbfcode{\sphinxupquote{Pauli}}}
\pysigstopsignatures
\sphinxAtStartPar
Bases: \sphinxcode{\sphinxupquote{object}}

\sphinxincludegraphics[]{inheritance-96cc10e3f8bc28dcdd4d4e78711b6d0bf760b14f.pdf}
\index{I (globvar.Pauli attribute)@\spxentry{I}\spxextra{globvar.Pauli attribute}}

\begin{fulllineitems}
\phantomsection\label{\detokenize{source/globvar:globvar.Pauli.I}}
\pysigstartsignatures
\pysigline
{\sphinxbfcode{\sphinxupquote{I}}\sphinxbfcode{\sphinxupquote{\DUrole{w}{ }\DUrole{p}{=}\DUrole{w}{ }array({[}{[}1, 0{]},        {[}0, 1{]}{]})}}}
\pysigstopsignatures
\end{fulllineitems}

\index{X (globvar.Pauli attribute)@\spxentry{X}\spxextra{globvar.Pauli attribute}}

\begin{fulllineitems}
\phantomsection\label{\detokenize{source/globvar:globvar.Pauli.X}}
\pysigstartsignatures
\pysigline
{\sphinxbfcode{\sphinxupquote{X}}\sphinxbfcode{\sphinxupquote{\DUrole{w}{ }\DUrole{p}{=}\DUrole{w}{ }array({[}{[}0, 1{]},        {[}1, 0{]}{]})}}}
\pysigstopsignatures
\end{fulllineitems}

\index{Y (globvar.Pauli attribute)@\spxentry{Y}\spxextra{globvar.Pauli attribute}}

\begin{fulllineitems}
\phantomsection\label{\detokenize{source/globvar:globvar.Pauli.Y}}
\pysigstartsignatures
\pysigline
{\sphinxbfcode{\sphinxupquote{Y}}\sphinxbfcode{\sphinxupquote{\DUrole{w}{ }\DUrole{p}{=}\DUrole{w}{ }array({[}{[} 0.+0.j, \sphinxhyphen{}0.\sphinxhyphen{}1.j{]},        {[} 0.+1.j,  0.+0.j{]}{]})}}}
\pysigstopsignatures
\end{fulllineitems}

\index{Z (globvar.Pauli attribute)@\spxentry{Z}\spxextra{globvar.Pauli attribute}}

\begin{fulllineitems}
\phantomsection\label{\detokenize{source/globvar:globvar.Pauli.Z}}
\pysigstartsignatures
\pysigline
{\sphinxbfcode{\sphinxupquote{Z}}\sphinxbfcode{\sphinxupquote{\DUrole{w}{ }\DUrole{p}{=}\DUrole{w}{ }array({[}{[} 1,  0{]},        {[} 0, \sphinxhyphen{}1{]}{]})}}}
\pysigstopsignatures
\end{fulllineitems}


\end{fulllineitems}

\index{SI (class in globvar)@\spxentry{SI}\spxextra{class in globvar}}

\begin{fulllineitems}
\phantomsection\label{\detokenize{source/globvar:globvar.SI}}
\pysigstartsignatures
\pysigline
{\sphinxbfcode{\sphinxupquote{class\DUrole{w}{ }}}\sphinxcode{\sphinxupquote{globvar.}}\sphinxbfcode{\sphinxupquote{SI}}}
\pysigstopsignatures
\sphinxAtStartPar
Bases: \sphinxcode{\sphinxupquote{object}}

\sphinxincludegraphics[]{inheritance-5902b66117c6fa3cc60a5a6069eb85956e1413d9.pdf}
\index{SI.Constants (class in globvar)@\spxentry{SI.Constants}\spxextra{class in globvar}}

\begin{fulllineitems}
\phantomsection\label{\detokenize{source/globvar:globvar.SI.Constants}}
\pysigstartsignatures
\pysigline
{\sphinxbfcode{\sphinxupquote{class\DUrole{w}{ }}}\sphinxbfcode{\sphinxupquote{Constants}}}
\pysigstopsignatures
\sphinxAtStartPar
Bases: \sphinxcode{\sphinxupquote{object}}
\index{Kcd (globvar.SI.Constants attribute)@\spxentry{Kcd}\spxextra{globvar.SI.Constants attribute}}

\begin{fulllineitems}
\phantomsection\label{\detokenize{source/globvar:globvar.SI.Constants.Kcd}}
\pysigstartsignatures
\pysigline
{\sphinxbfcode{\sphinxupquote{Kcd}}\sphinxbfcode{\sphinxupquote{\DUrole{w}{ }\DUrole{p}{=}\DUrole{w}{ }683}}}
\pysigstopsignatures
\end{fulllineitems}

\index{NAvogadro (globvar.SI.Constants attribute)@\spxentry{NAvogadro}\spxextra{globvar.SI.Constants attribute}}

\begin{fulllineitems}
\phantomsection\label{\detokenize{source/globvar:globvar.SI.Constants.NAvogadro}}
\pysigstartsignatures
\pysigline
{\sphinxbfcode{\sphinxupquote{NAvogadro}}\sphinxbfcode{\sphinxupquote{\DUrole{w}{ }\DUrole{p}{=}\DUrole{w}{ }6.02214076e\sphinxhyphen{}23}}}
\pysigstopsignatures
\end{fulllineitems}

\index{c (globvar.SI.Constants attribute)@\spxentry{c}\spxextra{globvar.SI.Constants attribute}}

\begin{fulllineitems}
\phantomsection\label{\detokenize{source/globvar:globvar.SI.Constants.c}}
\pysigstartsignatures
\pysigline
{\sphinxbfcode{\sphinxupquote{c}}\sphinxbfcode{\sphinxupquote{\DUrole{w}{ }\DUrole{p}{=}\DUrole{w}{ }299792458}}}
\pysigstopsignatures
\end{fulllineitems}

\index{dnuCs (globvar.SI.Constants attribute)@\spxentry{dnuCs}\spxextra{globvar.SI.Constants attribute}}

\begin{fulllineitems}
\phantomsection\label{\detokenize{source/globvar:globvar.SI.Constants.dnuCs}}
\pysigstartsignatures
\pysigline
{\sphinxbfcode{\sphinxupquote{dnuCs}}\sphinxbfcode{\sphinxupquote{\DUrole{w}{ }\DUrole{p}{=}\DUrole{w}{ }9192631770}}}
\pysigstopsignatures
\end{fulllineitems}

\index{e (globvar.SI.Constants attribute)@\spxentry{e}\spxextra{globvar.SI.Constants attribute}}

\begin{fulllineitems}
\phantomsection\label{\detokenize{source/globvar:globvar.SI.Constants.e}}
\pysigstartsignatures
\pysigline
{\sphinxbfcode{\sphinxupquote{e}}\sphinxbfcode{\sphinxupquote{\DUrole{w}{ }\DUrole{p}{=}\DUrole{w}{ }1.6021766340000001e\sphinxhyphen{}19}}}
\pysigstopsignatures
\end{fulllineitems}

\index{h (globvar.SI.Constants attribute)@\spxentry{h}\spxextra{globvar.SI.Constants attribute}}

\begin{fulllineitems}
\phantomsection\label{\detokenize{source/globvar:globvar.SI.Constants.h}}
\pysigstartsignatures
\pysigline
{\sphinxbfcode{\sphinxupquote{h}}\sphinxbfcode{\sphinxupquote{\DUrole{w}{ }\DUrole{p}{=}\DUrole{w}{ }6.62607015e\sphinxhyphen{}34}}}
\pysigstopsignatures
\end{fulllineitems}

\index{k (globvar.SI.Constants attribute)@\spxentry{k}\spxextra{globvar.SI.Constants attribute}}

\begin{fulllineitems}
\phantomsection\label{\detokenize{source/globvar:globvar.SI.Constants.k}}
\pysigstartsignatures
\pysigline
{\sphinxbfcode{\sphinxupquote{k}}\sphinxbfcode{\sphinxupquote{\DUrole{w}{ }\DUrole{p}{=}\DUrole{w}{ }1.380649e\sphinxhyphen{}23}}}
\pysigstopsignatures
\end{fulllineitems}


\end{fulllineitems}

\index{SI.Energy (class in globvar)@\spxentry{SI.Energy}\spxextra{class in globvar}}

\begin{fulllineitems}
\phantomsection\label{\detokenize{source/globvar:globvar.SI.Energy}}
\pysigstartsignatures
\pysigline
{\sphinxbfcode{\sphinxupquote{class\DUrole{w}{ }}}\sphinxbfcode{\sphinxupquote{Energy}}}
\pysigstopsignatures
\sphinxAtStartPar
Bases: \sphinxcode{\sphinxupquote{object}}
\index{GJ (globvar.SI.Energy attribute)@\spxentry{GJ}\spxextra{globvar.SI.Energy attribute}}

\begin{fulllineitems}
\phantomsection\label{\detokenize{source/globvar:globvar.SI.Energy.GJ}}
\pysigstartsignatures
\pysigline
{\sphinxbfcode{\sphinxupquote{GJ}}\sphinxbfcode{\sphinxupquote{\DUrole{w}{ }\DUrole{p}{=}\DUrole{w}{ }1000000000}}}
\pysigstopsignatures
\end{fulllineitems}

\index{J (globvar.SI.Energy attribute)@\spxentry{J}\spxextra{globvar.SI.Energy attribute}}

\begin{fulllineitems}
\phantomsection\label{\detokenize{source/globvar:globvar.SI.Energy.J}}
\pysigstartsignatures
\pysigline
{\sphinxbfcode{\sphinxupquote{J}}\sphinxbfcode{\sphinxupquote{\DUrole{w}{ }\DUrole{p}{=}\DUrole{w}{ }1}}}
\pysigstopsignatures
\end{fulllineitems}

\index{MJ (globvar.SI.Energy attribute)@\spxentry{MJ}\spxextra{globvar.SI.Energy attribute}}

\begin{fulllineitems}
\phantomsection\label{\detokenize{source/globvar:globvar.SI.Energy.MJ}}
\pysigstartsignatures
\pysigline
{\sphinxbfcode{\sphinxupquote{MJ}}\sphinxbfcode{\sphinxupquote{\DUrole{w}{ }\DUrole{p}{=}\DUrole{w}{ }1000000}}}
\pysigstopsignatures
\end{fulllineitems}

\index{aJ (globvar.SI.Energy attribute)@\spxentry{aJ}\spxextra{globvar.SI.Energy attribute}}

\begin{fulllineitems}
\phantomsection\label{\detokenize{source/globvar:globvar.SI.Energy.aJ}}
\pysigstartsignatures
\pysigline
{\sphinxbfcode{\sphinxupquote{aJ}}\sphinxbfcode{\sphinxupquote{\DUrole{w}{ }\DUrole{p}{=}\DUrole{w}{ }1e\sphinxhyphen{}18}}}
\pysigstopsignatures
\end{fulllineitems}

\index{eV (globvar.SI.Energy attribute)@\spxentry{eV}\spxextra{globvar.SI.Energy attribute}}

\begin{fulllineitems}
\phantomsection\label{\detokenize{source/globvar:globvar.SI.Energy.eV}}
\pysigstartsignatures
\pysigline
{\sphinxbfcode{\sphinxupquote{eV}}\sphinxbfcode{\sphinxupquote{\DUrole{w}{ }\DUrole{p}{=}\DUrole{w}{ }0.00012897410387530461}}}
\pysigstopsignatures
\end{fulllineitems}

\index{fJ (globvar.SI.Energy attribute)@\spxentry{fJ}\spxextra{globvar.SI.Energy attribute}}

\begin{fulllineitems}
\phantomsection\label{\detokenize{source/globvar:globvar.SI.Energy.fJ}}
\pysigstartsignatures
\pysigline
{\sphinxbfcode{\sphinxupquote{fJ}}\sphinxbfcode{\sphinxupquote{\DUrole{w}{ }\DUrole{p}{=}\DUrole{w}{ }1e\sphinxhyphen{}15}}}
\pysigstopsignatures
\end{fulllineitems}

\index{kJ (globvar.SI.Energy attribute)@\spxentry{kJ}\spxextra{globvar.SI.Energy attribute}}

\begin{fulllineitems}
\phantomsection\label{\detokenize{source/globvar:globvar.SI.Energy.kJ}}
\pysigstartsignatures
\pysigline
{\sphinxbfcode{\sphinxupquote{kJ}}\sphinxbfcode{\sphinxupquote{\DUrole{w}{ }\DUrole{p}{=}\DUrole{w}{ }1000}}}
\pysigstopsignatures
\end{fulllineitems}

\index{mJ (globvar.SI.Energy attribute)@\spxentry{mJ}\spxextra{globvar.SI.Energy attribute}}

\begin{fulllineitems}
\phantomsection\label{\detokenize{source/globvar:globvar.SI.Energy.mJ}}
\pysigstartsignatures
\pysigline
{\sphinxbfcode{\sphinxupquote{mJ}}\sphinxbfcode{\sphinxupquote{\DUrole{w}{ }\DUrole{p}{=}\DUrole{w}{ }0.001}}}
\pysigstopsignatures
\end{fulllineitems}

\index{mcJ (globvar.SI.Energy attribute)@\spxentry{mcJ}\spxextra{globvar.SI.Energy attribute}}

\begin{fulllineitems}
\phantomsection\label{\detokenize{source/globvar:globvar.SI.Energy.mcJ}}
\pysigstartsignatures
\pysigline
{\sphinxbfcode{\sphinxupquote{mcJ}}\sphinxbfcode{\sphinxupquote{\DUrole{w}{ }\DUrole{p}{=}\DUrole{w}{ }1e\sphinxhyphen{}06}}}
\pysigstopsignatures
\end{fulllineitems}

\index{nJ (globvar.SI.Energy attribute)@\spxentry{nJ}\spxextra{globvar.SI.Energy attribute}}

\begin{fulllineitems}
\phantomsection\label{\detokenize{source/globvar:globvar.SI.Energy.nJ}}
\pysigstartsignatures
\pysigline
{\sphinxbfcode{\sphinxupquote{nJ}}\sphinxbfcode{\sphinxupquote{\DUrole{w}{ }\DUrole{p}{=}\DUrole{w}{ }1e\sphinxhyphen{}09}}}
\pysigstopsignatures
\end{fulllineitems}

\index{pJ (globvar.SI.Energy attribute)@\spxentry{pJ}\spxextra{globvar.SI.Energy attribute}}

\begin{fulllineitems}
\phantomsection\label{\detokenize{source/globvar:globvar.SI.Energy.pJ}}
\pysigstartsignatures
\pysigline
{\sphinxbfcode{\sphinxupquote{pJ}}\sphinxbfcode{\sphinxupquote{\DUrole{w}{ }\DUrole{p}{=}\DUrole{w}{ }1e\sphinxhyphen{}12}}}
\pysigstopsignatures
\end{fulllineitems}


\end{fulllineitems}

\index{SI.Force (class in globvar)@\spxentry{SI.Force}\spxextra{class in globvar}}

\begin{fulllineitems}
\phantomsection\label{\detokenize{source/globvar:globvar.SI.Force}}
\pysigstartsignatures
\pysigline
{\sphinxbfcode{\sphinxupquote{class\DUrole{w}{ }}}\sphinxbfcode{\sphinxupquote{Force}}}
\pysigstopsignatures
\sphinxAtStartPar
Bases: \sphinxcode{\sphinxupquote{object}}
\index{GN (globvar.SI.Force attribute)@\spxentry{GN}\spxextra{globvar.SI.Force attribute}}

\begin{fulllineitems}
\phantomsection\label{\detokenize{source/globvar:globvar.SI.Force.GN}}
\pysigstartsignatures
\pysigline
{\sphinxbfcode{\sphinxupquote{GN}}\sphinxbfcode{\sphinxupquote{\DUrole{w}{ }\DUrole{p}{=}\DUrole{w}{ }1000000000}}}
\pysigstopsignatures
\end{fulllineitems}

\index{MN (globvar.SI.Force attribute)@\spxentry{MN}\spxextra{globvar.SI.Force attribute}}

\begin{fulllineitems}
\phantomsection\label{\detokenize{source/globvar:globvar.SI.Force.MN}}
\pysigstartsignatures
\pysigline
{\sphinxbfcode{\sphinxupquote{MN}}\sphinxbfcode{\sphinxupquote{\DUrole{w}{ }\DUrole{p}{=}\DUrole{w}{ }1000000}}}
\pysigstopsignatures
\end{fulllineitems}

\index{N (globvar.SI.Force attribute)@\spxentry{N}\spxextra{globvar.SI.Force attribute}}

\begin{fulllineitems}
\phantomsection\label{\detokenize{source/globvar:globvar.SI.Force.N}}
\pysigstartsignatures
\pysigline
{\sphinxbfcode{\sphinxupquote{N}}\sphinxbfcode{\sphinxupquote{\DUrole{w}{ }\DUrole{p}{=}\DUrole{w}{ }1}}}
\pysigstopsignatures
\end{fulllineitems}

\index{fN (globvar.SI.Force attribute)@\spxentry{fN}\spxextra{globvar.SI.Force attribute}}

\begin{fulllineitems}
\phantomsection\label{\detokenize{source/globvar:globvar.SI.Force.fN}}
\pysigstartsignatures
\pysigline
{\sphinxbfcode{\sphinxupquote{fN}}\sphinxbfcode{\sphinxupquote{\DUrole{w}{ }\DUrole{p}{=}\DUrole{w}{ }1e\sphinxhyphen{}15}}}
\pysigstopsignatures
\end{fulllineitems}

\index{kN (globvar.SI.Force attribute)@\spxentry{kN}\spxextra{globvar.SI.Force attribute}}

\begin{fulllineitems}
\phantomsection\label{\detokenize{source/globvar:globvar.SI.Force.kN}}
\pysigstartsignatures
\pysigline
{\sphinxbfcode{\sphinxupquote{kN}}\sphinxbfcode{\sphinxupquote{\DUrole{w}{ }\DUrole{p}{=}\DUrole{w}{ }1000}}}
\pysigstopsignatures
\end{fulllineitems}

\index{mN (globvar.SI.Force attribute)@\spxentry{mN}\spxextra{globvar.SI.Force attribute}}

\begin{fulllineitems}
\phantomsection\label{\detokenize{source/globvar:globvar.SI.Force.mN}}
\pysigstartsignatures
\pysigline
{\sphinxbfcode{\sphinxupquote{mN}}\sphinxbfcode{\sphinxupquote{\DUrole{w}{ }\DUrole{p}{=}\DUrole{w}{ }0.001}}}
\pysigstopsignatures
\end{fulllineitems}

\index{mcN (globvar.SI.Force attribute)@\spxentry{mcN}\spxextra{globvar.SI.Force attribute}}

\begin{fulllineitems}
\phantomsection\label{\detokenize{source/globvar:globvar.SI.Force.mcN}}
\pysigstartsignatures
\pysigline
{\sphinxbfcode{\sphinxupquote{mcN}}\sphinxbfcode{\sphinxupquote{\DUrole{w}{ }\DUrole{p}{=}\DUrole{w}{ }1e\sphinxhyphen{}06}}}
\pysigstopsignatures
\end{fulllineitems}

\index{nN (globvar.SI.Force attribute)@\spxentry{nN}\spxextra{globvar.SI.Force attribute}}

\begin{fulllineitems}
\phantomsection\label{\detokenize{source/globvar:globvar.SI.Force.nN}}
\pysigstartsignatures
\pysigline
{\sphinxbfcode{\sphinxupquote{nN}}\sphinxbfcode{\sphinxupquote{\DUrole{w}{ }\DUrole{p}{=}\DUrole{w}{ }1e\sphinxhyphen{}09}}}
\pysigstopsignatures
\end{fulllineitems}

\index{pN (globvar.SI.Force attribute)@\spxentry{pN}\spxextra{globvar.SI.Force attribute}}

\begin{fulllineitems}
\phantomsection\label{\detokenize{source/globvar:globvar.SI.Force.pN}}
\pysigstartsignatures
\pysigline
{\sphinxbfcode{\sphinxupquote{pN}}\sphinxbfcode{\sphinxupquote{\DUrole{w}{ }\DUrole{p}{=}\DUrole{w}{ }1e\sphinxhyphen{}12}}}
\pysigstopsignatures
\end{fulllineitems}


\end{fulllineitems}

\index{SI.Frequency (class in globvar)@\spxentry{SI.Frequency}\spxextra{class in globvar}}

\begin{fulllineitems}
\phantomsection\label{\detokenize{source/globvar:globvar.SI.Frequency}}
\pysigstartsignatures
\pysigline
{\sphinxbfcode{\sphinxupquote{class\DUrole{w}{ }}}\sphinxbfcode{\sphinxupquote{Frequency}}}
\pysigstopsignatures
\sphinxAtStartPar
Bases: \sphinxcode{\sphinxupquote{object}}
\index{EHz (globvar.SI.Frequency attribute)@\spxentry{EHz}\spxextra{globvar.SI.Frequency attribute}}

\begin{fulllineitems}
\phantomsection\label{\detokenize{source/globvar:globvar.SI.Frequency.EHz}}
\pysigstartsignatures
\pysigline
{\sphinxbfcode{\sphinxupquote{EHz}}\sphinxbfcode{\sphinxupquote{\DUrole{w}{ }\DUrole{p}{=}\DUrole{w}{ }1000000000000000000}}}
\pysigstopsignatures
\end{fulllineitems}

\index{GHz (globvar.SI.Frequency attribute)@\spxentry{GHz}\spxextra{globvar.SI.Frequency attribute}}

\begin{fulllineitems}
\phantomsection\label{\detokenize{source/globvar:globvar.SI.Frequency.GHz}}
\pysigstartsignatures
\pysigline
{\sphinxbfcode{\sphinxupquote{GHz}}\sphinxbfcode{\sphinxupquote{\DUrole{w}{ }\DUrole{p}{=}\DUrole{w}{ }1000000000}}}
\pysigstopsignatures
\end{fulllineitems}

\index{Hz (globvar.SI.Frequency attribute)@\spxentry{Hz}\spxextra{globvar.SI.Frequency attribute}}

\begin{fulllineitems}
\phantomsection\label{\detokenize{source/globvar:globvar.SI.Frequency.Hz}}
\pysigstartsignatures
\pysigline
{\sphinxbfcode{\sphinxupquote{Hz}}\sphinxbfcode{\sphinxupquote{\DUrole{w}{ }\DUrole{p}{=}\DUrole{w}{ }1}}}
\pysigstopsignatures
\end{fulllineitems}

\index{MHz (globvar.SI.Frequency attribute)@\spxentry{MHz}\spxextra{globvar.SI.Frequency attribute}}

\begin{fulllineitems}
\phantomsection\label{\detokenize{source/globvar:globvar.SI.Frequency.MHz}}
\pysigstartsignatures
\pysigline
{\sphinxbfcode{\sphinxupquote{MHz}}\sphinxbfcode{\sphinxupquote{\DUrole{w}{ }\DUrole{p}{=}\DUrole{w}{ }1000000}}}
\pysigstopsignatures
\end{fulllineitems}

\index{PHz (globvar.SI.Frequency attribute)@\spxentry{PHz}\spxextra{globvar.SI.Frequency attribute}}

\begin{fulllineitems}
\phantomsection\label{\detokenize{source/globvar:globvar.SI.Frequency.PHz}}
\pysigstartsignatures
\pysigline
{\sphinxbfcode{\sphinxupquote{PHz}}\sphinxbfcode{\sphinxupquote{\DUrole{w}{ }\DUrole{p}{=}\DUrole{w}{ }1000000000000000}}}
\pysigstopsignatures
\end{fulllineitems}

\index{THz (globvar.SI.Frequency attribute)@\spxentry{THz}\spxextra{globvar.SI.Frequency attribute}}

\begin{fulllineitems}
\phantomsection\label{\detokenize{source/globvar:globvar.SI.Frequency.THz}}
\pysigstartsignatures
\pysigline
{\sphinxbfcode{\sphinxupquote{THz}}\sphinxbfcode{\sphinxupquote{\DUrole{w}{ }\DUrole{p}{=}\DUrole{w}{ }1000000000000}}}
\pysigstopsignatures
\end{fulllineitems}

\index{kHz (globvar.SI.Frequency attribute)@\spxentry{kHz}\spxextra{globvar.SI.Frequency attribute}}

\begin{fulllineitems}
\phantomsection\label{\detokenize{source/globvar:globvar.SI.Frequency.kHz}}
\pysigstartsignatures
\pysigline
{\sphinxbfcode{\sphinxupquote{kHz}}\sphinxbfcode{\sphinxupquote{\DUrole{w}{ }\DUrole{p}{=}\DUrole{w}{ }1000}}}
\pysigstopsignatures
\end{fulllineitems}


\end{fulllineitems}

\index{SI.Length (class in globvar)@\spxentry{SI.Length}\spxextra{class in globvar}}

\begin{fulllineitems}
\phantomsection\label{\detokenize{source/globvar:globvar.SI.Length}}
\pysigstartsignatures
\pysigline
{\sphinxbfcode{\sphinxupquote{class\DUrole{w}{ }}}\sphinxbfcode{\sphinxupquote{Length}}}
\pysigstopsignatures
\sphinxAtStartPar
Bases: \sphinxcode{\sphinxupquote{object}}
\index{fm (globvar.SI.Length attribute)@\spxentry{fm}\spxextra{globvar.SI.Length attribute}}

\begin{fulllineitems}
\phantomsection\label{\detokenize{source/globvar:globvar.SI.Length.fm}}
\pysigstartsignatures
\pysigline
{\sphinxbfcode{\sphinxupquote{fm}}\sphinxbfcode{\sphinxupquote{\DUrole{w}{ }\DUrole{p}{=}\DUrole{w}{ }1e\sphinxhyphen{}15}}}
\pysigstopsignatures
\end{fulllineitems}

\index{km (globvar.SI.Length attribute)@\spxentry{km}\spxextra{globvar.SI.Length attribute}}

\begin{fulllineitems}
\phantomsection\label{\detokenize{source/globvar:globvar.SI.Length.km}}
\pysigstartsignatures
\pysigline
{\sphinxbfcode{\sphinxupquote{km}}\sphinxbfcode{\sphinxupquote{\DUrole{w}{ }\DUrole{p}{=}\DUrole{w}{ }1000}}}
\pysigstopsignatures
\end{fulllineitems}

\index{m (globvar.SI.Length attribute)@\spxentry{m}\spxextra{globvar.SI.Length attribute}}

\begin{fulllineitems}
\phantomsection\label{\detokenize{source/globvar:globvar.SI.Length.m}}
\pysigstartsignatures
\pysigline
{\sphinxbfcode{\sphinxupquote{m}}\sphinxbfcode{\sphinxupquote{\DUrole{w}{ }\DUrole{p}{=}\DUrole{w}{ }1}}}
\pysigstopsignatures
\end{fulllineitems}

\index{mcm (globvar.SI.Length attribute)@\spxentry{mcm}\spxextra{globvar.SI.Length attribute}}

\begin{fulllineitems}
\phantomsection\label{\detokenize{source/globvar:globvar.SI.Length.mcm}}
\pysigstartsignatures
\pysigline
{\sphinxbfcode{\sphinxupquote{mcm}}\sphinxbfcode{\sphinxupquote{\DUrole{w}{ }\DUrole{p}{=}\DUrole{w}{ }1e\sphinxhyphen{}06}}}
\pysigstopsignatures
\end{fulllineitems}

\index{mm (globvar.SI.Length attribute)@\spxentry{mm}\spxextra{globvar.SI.Length attribute}}

\begin{fulllineitems}
\phantomsection\label{\detokenize{source/globvar:globvar.SI.Length.mm}}
\pysigstartsignatures
\pysigline
{\sphinxbfcode{\sphinxupquote{mm}}\sphinxbfcode{\sphinxupquote{\DUrole{w}{ }\DUrole{p}{=}\DUrole{w}{ }0.001}}}
\pysigstopsignatures
\end{fulllineitems}

\index{nm (globvar.SI.Length attribute)@\spxentry{nm}\spxextra{globvar.SI.Length attribute}}

\begin{fulllineitems}
\phantomsection\label{\detokenize{source/globvar:globvar.SI.Length.nm}}
\pysigstartsignatures
\pysigline
{\sphinxbfcode{\sphinxupquote{nm}}\sphinxbfcode{\sphinxupquote{\DUrole{w}{ }\DUrole{p}{=}\DUrole{w}{ }1e\sphinxhyphen{}09}}}
\pysigstopsignatures
\end{fulllineitems}

\index{pm (globvar.SI.Length attribute)@\spxentry{pm}\spxextra{globvar.SI.Length attribute}}

\begin{fulllineitems}
\phantomsection\label{\detokenize{source/globvar:globvar.SI.Length.pm}}
\pysigstartsignatures
\pysigline
{\sphinxbfcode{\sphinxupquote{pm}}\sphinxbfcode{\sphinxupquote{\DUrole{w}{ }\DUrole{p}{=}\DUrole{w}{ }1e\sphinxhyphen{}12}}}
\pysigstopsignatures
\end{fulllineitems}


\end{fulllineitems}

\index{SI.Time (class in globvar)@\spxentry{SI.Time}\spxextra{class in globvar}}

\begin{fulllineitems}
\phantomsection\label{\detokenize{source/globvar:globvar.SI.Time}}
\pysigstartsignatures
\pysigline
{\sphinxbfcode{\sphinxupquote{class\DUrole{w}{ }}}\sphinxbfcode{\sphinxupquote{Time}}}
\pysigstopsignatures
\sphinxAtStartPar
Bases: \sphinxcode{\sphinxupquote{object}}
\index{ats (globvar.SI.Time attribute)@\spxentry{ats}\spxextra{globvar.SI.Time attribute}}

\begin{fulllineitems}
\phantomsection\label{\detokenize{source/globvar:globvar.SI.Time.ats}}
\pysigstartsignatures
\pysigline
{\sphinxbfcode{\sphinxupquote{ats}}\sphinxbfcode{\sphinxupquote{\DUrole{w}{ }\DUrole{p}{=}\DUrole{w}{ }1e\sphinxhyphen{}18}}}
\pysigstopsignatures
\end{fulllineitems}

\index{fs (globvar.SI.Time attribute)@\spxentry{fs}\spxextra{globvar.SI.Time attribute}}

\begin{fulllineitems}
\phantomsection\label{\detokenize{source/globvar:globvar.SI.Time.fs}}
\pysigstartsignatures
\pysigline
{\sphinxbfcode{\sphinxupquote{fs}}\sphinxbfcode{\sphinxupquote{\DUrole{w}{ }\DUrole{p}{=}\DUrole{w}{ }1e\sphinxhyphen{}15}}}
\pysigstopsignatures
\end{fulllineitems}

\index{mcs (globvar.SI.Time attribute)@\spxentry{mcs}\spxextra{globvar.SI.Time attribute}}

\begin{fulllineitems}
\phantomsection\label{\detokenize{source/globvar:globvar.SI.Time.mcs}}
\pysigstartsignatures
\pysigline
{\sphinxbfcode{\sphinxupquote{mcs}}\sphinxbfcode{\sphinxupquote{\DUrole{w}{ }\DUrole{p}{=}\DUrole{w}{ }1e\sphinxhyphen{}06}}}
\pysigstopsignatures
\end{fulllineitems}

\index{ms (globvar.SI.Time attribute)@\spxentry{ms}\spxextra{globvar.SI.Time attribute}}

\begin{fulllineitems}
\phantomsection\label{\detokenize{source/globvar:globvar.SI.Time.ms}}
\pysigstartsignatures
\pysigline
{\sphinxbfcode{\sphinxupquote{ms}}\sphinxbfcode{\sphinxupquote{\DUrole{w}{ }\DUrole{p}{=}\DUrole{w}{ }0.001}}}
\pysigstopsignatures
\end{fulllineitems}

\index{ns (globvar.SI.Time attribute)@\spxentry{ns}\spxextra{globvar.SI.Time attribute}}

\begin{fulllineitems}
\phantomsection\label{\detokenize{source/globvar:globvar.SI.Time.ns}}
\pysigstartsignatures
\pysigline
{\sphinxbfcode{\sphinxupquote{ns}}\sphinxbfcode{\sphinxupquote{\DUrole{w}{ }\DUrole{p}{=}\DUrole{w}{ }1e\sphinxhyphen{}09}}}
\pysigstopsignatures
\end{fulllineitems}

\index{ps (globvar.SI.Time attribute)@\spxentry{ps}\spxextra{globvar.SI.Time attribute}}

\begin{fulllineitems}
\phantomsection\label{\detokenize{source/globvar:globvar.SI.Time.ps}}
\pysigstartsignatures
\pysigline
{\sphinxbfcode{\sphinxupquote{ps}}\sphinxbfcode{\sphinxupquote{\DUrole{w}{ }\DUrole{p}{=}\DUrole{w}{ }1e\sphinxhyphen{}12}}}
\pysigstopsignatures
\end{fulllineitems}

\index{s (globvar.SI.Time attribute)@\spxentry{s}\spxextra{globvar.SI.Time attribute}}

\begin{fulllineitems}
\phantomsection\label{\detokenize{source/globvar:globvar.SI.Time.s}}
\pysigstartsignatures
\pysigline
{\sphinxbfcode{\sphinxupquote{s}}\sphinxbfcode{\sphinxupquote{\DUrole{w}{ }\DUrole{p}{=}\DUrole{w}{ }1}}}
\pysigstopsignatures
\end{fulllineitems}


\end{fulllineitems}


\end{fulllineitems}


\sphinxstepscope


\subsection{input\_parser module}
\label{\detokenize{source/input_parser:module-input_parser}}\label{\detokenize{source/input_parser:input-parser-module}}\label{\detokenize{source/input_parser::doc}}\index{module@\spxentry{module}!input\_parser@\spxentry{input\_parser}}\index{input\_parser@\spxentry{input\_parser}!module@\spxentry{module}}
\sphinxAtStartPar
This module contains a class for parsing command line arguments for the initialization of GUI

\sphinxincludegraphics[]{inheritance-3ec93793914a377fc6d97a7d860549fc35859d45.pdf}
\index{ConfigParser (class in input\_parser)@\spxentry{ConfigParser}\spxextra{class in input\_parser}}

\begin{fulllineitems}
\phantomsection\label{\detokenize{source/input_parser:input_parser.ConfigParser}}
\pysigstartsignatures
\pysigline
{\sphinxbfcode{\sphinxupquote{class\DUrole{w}{ }}}\sphinxcode{\sphinxupquote{input\_parser.}}\sphinxbfcode{\sphinxupquote{ConfigParser}}}
\pysigstopsignatures
\sphinxAtStartPar
Bases: \sphinxcode{\sphinxupquote{object}}

\sphinxincludegraphics[]{inheritance-7846408d3445afcf16d4e263a8720c67581c31de.pdf}
\index{read\_configuration() (input\_parser.ConfigParser static method)@\spxentry{read\_configuration()}\spxextra{input\_parser.ConfigParser static method}}

\begin{fulllineitems}
\phantomsection\label{\detokenize{source/input_parser:input_parser.ConfigParser.read_configuration}}
\pysigstartsignatures
\pysiglinewithargsret
{\sphinxbfcode{\sphinxupquote{static\DUrole{w}{ }}}\sphinxbfcode{\sphinxupquote{read\_configuration}}}
{\sphinxparam{\DUrole{n}{file}}}
{}
\pysigstopsignatures
\end{fulllineitems}


\end{fulllineitems}


\sphinxstepscope


\subsection{kernels module}
\label{\detokenize{source/kernels:module-kernels}}\label{\detokenize{source/kernels:kernels-module}}\label{\detokenize{source/kernels::doc}}\index{module@\spxentry{module}!kernels@\spxentry{kernels}}\index{kernels@\spxentry{kernels}!module@\spxentry{module}}
\sphinxAtStartPar
kernels.py

\sphinxAtStartPar
This module contains functions for generating finite size real space kernels for the SSNR calculations.
\begin{description}
\sphinxlineitem{Functions}
\sphinxAtStartPar
sinc\_kernel: Generate a 2D/3D triangular kernel, resulting in :math: \sphinxtitleref{sinc\textasciicircum{}2} in Fourier space.
psf\_kernel2d: Generate a 2D kernel that has the shape of PSF in the Fourier domain (and hence the shape of OTF in the real space).

\end{description}
\index{psf\_kernel2d() (in module kernels)@\spxentry{psf\_kernel2d()}\spxextra{in module kernels}}

\begin{fulllineitems}
\phantomsection\label{\detokenize{source/kernels:kernels.psf_kernel2d}}
\pysigstartsignatures
\pysiglinewithargsret
{\sphinxcode{\sphinxupquote{kernels.}}\sphinxbfcode{\sphinxupquote{psf\_kernel2d}}}
{\sphinxparam{\DUrole{n}{kernel\_size}\DUrole{p}{:}\DUrole{w}{ }\DUrole{n}{int}}\sphinxparamcomma \sphinxparam{\DUrole{n}{pixel\_size}\DUrole{p}{:}\DUrole{w}{ }\DUrole{n}{float}}\sphinxparamcomma \sphinxparam{\DUrole{n}{dense\_kernel\_size}\DUrole{o}{=}\DUrole{default_value}{50}}}
{{ $\rightarrow$ ndarray}}
\pysigstopsignatures
\sphinxAtStartPar
Generate a 2D kernel that has the shape of PSF in the Fourier domain (and hence the shape of OTF in the real space).
\begin{quote}\begin{description}
\sphinxlineitem{Parameters}\begin{itemize}
\item {} 
\sphinxAtStartPar
\sphinxstyleliteralstrong{\sphinxupquote{kernel\_size}} \textendash{} The size of the kernel.

\item {} 
\sphinxAtStartPar
\sphinxstyleliteralstrong{\sphinxupquote{pixel\_size}} \textendash{} The pixel size in the real space.

\item {} 
\sphinxAtStartPar
\sphinxstyleliteralstrong{\sphinxupquote{dense\_kernel\_size}} \textendash{} The size of the dense kernel. Default is 50. This parameter is used for better interpolation of the PSF values on a small grid.

\end{itemize}

\sphinxlineitem{Returns}
\sphinxAtStartPar
A 2D kernel.

\end{description}\end{quote}

\end{fulllineitems}

\index{sinc\_kernel() (in module kernels)@\spxentry{sinc\_kernel()}\spxextra{in module kernels}}

\begin{fulllineitems}
\phantomsection\label{\detokenize{source/kernels:kernels.sinc_kernel}}
\pysigstartsignatures
\pysiglinewithargsret
{\sphinxcode{\sphinxupquote{kernels.}}\sphinxbfcode{\sphinxupquote{sinc\_kernel}}}
{\sphinxparam{\DUrole{n}{kernel\_r\_size}\DUrole{p}{:}\DUrole{w}{ }\DUrole{n}{int}}\sphinxparamcomma \sphinxparam{\DUrole{n}{kernel\_z\_size}\DUrole{o}{=}\DUrole{default_value}{1}}}
{{ $\rightarrow$ ndarray}}
\pysigstopsignatures
\sphinxAtStartPar
Generate a 2D/3D triangular kernel, resulting in :math: \sphinxtitleref{sinc\textasciicircum{}2} in Fourier space.
\begin{quote}\begin{description}
\sphinxlineitem{Parameters}\begin{itemize}
\item {} 
\sphinxAtStartPar
\sphinxstyleliteralstrong{\sphinxupquote{kernel\_r\_size}} \textendash{} The size of the kernel in the radial direction.

\item {} 
\sphinxAtStartPar
\sphinxstyleliteralstrong{\sphinxupquote{kernel\_z\_size}} \textendash{} The size of the kernel in the axial direction. Default is 1.

\end{itemize}

\sphinxlineitem{Returns}
\sphinxAtStartPar
A 2D/3D triangular kernel.

\end{description}\end{quote}

\end{fulllineitems}


\sphinxstepscope


\subsection{stattools module}
\label{\detokenize{source/stattools:module-stattools}}\label{\detokenize{source/stattools:stattools-module}}\label{\detokenize{source/stattools::doc}}\index{module@\spxentry{module}!stattools@\spxentry{stattools}}\index{stattools@\spxentry{stattools}!module@\spxentry{module}}
\sphinxAtStartPar
stattools.py

\sphinxAtStartPar
This module contains commonly used operations on arrays, required in the context of our work.
\index{average\_mask() (in module stattools)@\spxentry{average\_mask()}\spxextra{in module stattools}}

\begin{fulllineitems}
\phantomsection\label{\detokenize{source/stattools:stattools.average_mask}}
\pysigstartsignatures
\pysiglinewithargsret
{\sphinxcode{\sphinxupquote{stattools.}}\sphinxbfcode{\sphinxupquote{average\_mask}}}
{\sphinxparam{\DUrole{n}{array}\DUrole{p}{:}\DUrole{w}{ }\DUrole{n}{ndarray\DUrole{p}{{[}}float64\DUrole{p}{{]}}}}\sphinxparamcomma \sphinxparam{\DUrole{n}{mask}\DUrole{p}{:}\DUrole{w}{ }\DUrole{n}{ndarray\DUrole{p}{{[}}int32\DUrole{p}{{]}}}}\sphinxparamcomma \sphinxparam{\DUrole{n}{shape}\DUrole{o}{=}\DUrole{default_value}{\textquotesingle{}same\textquotesingle{}}}}
{{ $\rightarrow$ ndarray\DUrole{p}{{[}}float64\DUrole{p}{{]}}}}
\pysigstopsignatures
\sphinxAtStartPar
Averages an array along the surface levels of the mask.
\begin{quote}\begin{description}
\sphinxlineitem{Parameters}\begin{itemize}
\item {} 
\sphinxAtStartPar
\sphinxstyleliteralstrong{\sphinxupquote{array}} (\sphinxstyleliteralemphasis{\sphinxupquote{np.ndarray}}) \textendash{} Array to average.

\item {} 
\sphinxAtStartPar
\sphinxstyleliteralstrong{\sphinxupquote{mask}} (\sphinxstyleliteralemphasis{\sphinxupquote{np.ndarray}}\sphinxstyleliteralemphasis{\sphinxupquote{{[}}}\sphinxstyleliteralemphasis{\sphinxupquote{np.int32}}\sphinxstyleliteralemphasis{\sphinxupquote{{]}}}) \textendash{} Mask indicating regions to average.

\item {} 
\sphinxAtStartPar
\sphinxstyleliteralstrong{\sphinxupquote{shape}} (\sphinxstyleliteralemphasis{\sphinxupquote{str}}\sphinxstyleliteralemphasis{\sphinxupquote{, }}\sphinxstyleliteralemphasis{\sphinxupquote{optional}}) \textendash{} Shape of the output array. Defaults to ‘same’.

\end{itemize}

\sphinxlineitem{Returns}
\sphinxAtStartPar
Averaged array.

\sphinxlineitem{Return type}
\sphinxAtStartPar
np.ndarray

\end{description}\end{quote}

\end{fulllineitems}

\index{average\_rings2d() (in module stattools)@\spxentry{average\_rings2d()}\spxextra{in module stattools}}

\begin{fulllineitems}
\phantomsection\label{\detokenize{source/stattools:stattools.average_rings2d}}
\pysigstartsignatures
\pysiglinewithargsret
{\sphinxcode{\sphinxupquote{stattools.}}\sphinxbfcode{\sphinxupquote{average\_rings2d}}}
{\sphinxparam{\DUrole{n}{array}\DUrole{p}{:}\DUrole{w}{ }\DUrole{n}{ndarray}}\sphinxparamcomma \sphinxparam{\DUrole{n}{axes}\DUrole{p}{:}\DUrole{w}{ }\DUrole{n}{tuple\DUrole{p}{{[}}ndarray\DUrole{p}{{]}}}\DUrole{w}{ }\DUrole{o}{=}\DUrole{w}{ }\DUrole{default_value}{None}}\sphinxparamcomma \sphinxparam{\DUrole{n}{num\_angles}\DUrole{o}{=}\DUrole{default_value}{360}}\sphinxparamcomma \sphinxparam{\DUrole{n}{number\_of\_samples}\DUrole{p}{:}\DUrole{w}{ }\DUrole{n}{int}\DUrole{w}{ }\DUrole{o}{=}\DUrole{w}{ }\DUrole{default_value}{None}}}
{}
\pysigstopsignatures
\sphinxAtStartPar
Averages the 2D array radially using bilinear interpolation in polar coordinates.
\begin{quote}\begin{description}
\sphinxlineitem{Parameters}\begin{itemize}
\item {} 
\sphinxAtStartPar
\sphinxstyleliteralstrong{\sphinxupquote{array}} \textendash{} 2D numpy array to average radially.

\item {} 
\sphinxAtStartPar
\sphinxstyleliteralstrong{\sphinxupquote{axes}} \textendash{} Tuple of arrays representing the grid axes (ax1, ax2).

\item {} 
\sphinxAtStartPar
\sphinxstyleliteralstrong{\sphinxupquote{num\_samples}} \textendash{} Number of radial samples (r) to take.

\item {} 
\sphinxAtStartPar
\sphinxstyleliteralstrong{\sphinxupquote{num\_angles}} \textendash{} Number of angular samples (theta).

\end{itemize}

\sphinxlineitem{Returns}
\sphinxAtStartPar
Radial distances at which the interpolation is performed.
averaged: Radially averaged values.

\sphinxlineitem{Return type}
\sphinxAtStartPar
radii

\end{description}\end{quote}

\end{fulllineitems}

\index{average\_rings3d() (in module stattools)@\spxentry{average\_rings3d()}\spxextra{in module stattools}}

\begin{fulllineitems}
\phantomsection\label{\detokenize{source/stattools:stattools.average_rings3d}}
\pysigstartsignatures
\pysiglinewithargsret
{\sphinxcode{\sphinxupquote{stattools.}}\sphinxbfcode{\sphinxupquote{average\_rings3d}}}
{\sphinxparam{\DUrole{n}{array}\DUrole{p}{:}\DUrole{w}{ }\DUrole{n}{ndarray\DUrole{p}{{[}}tuple\DUrole{p}{{[}}int\DUrole{p}{,}\DUrole{w}{ }int\DUrole{p}{,}\DUrole{w}{ }int\DUrole{p}{{]}}\DUrole{p}{,}\DUrole{w}{ }\DUrole{p}{...}\DUrole{p}{{]}}}}\sphinxparamcomma \sphinxparam{\DUrole{n}{axes}\DUrole{p}{:}\DUrole{w}{ }\DUrole{n}{tuple\DUrole{p}{{[}}ndarray\DUrole{p}{,}\DUrole{w}{ }ndarray\DUrole{p}{,}\DUrole{w}{ }ndarray\DUrole{p}{{]}}}\DUrole{w}{ }\DUrole{o}{=}\DUrole{w}{ }\DUrole{default_value}{None}}}
{{ $\rightarrow$ ndarray\DUrole{p}{{[}}tuple\DUrole{p}{{[}}int\DUrole{p}{,}\DUrole{w}{ }int\DUrole{p}{{]}}\DUrole{p}{,}\DUrole{w}{ }\DUrole{p}{...}\DUrole{p}{{]}}}}
\pysigstopsignatures
\sphinxAtStartPar
Averages the 3D array radially by averaging each 2D slice.
\begin{quote}\begin{description}
\sphinxlineitem{Parameters}\begin{itemize}
\item {} 
\sphinxAtStartPar
\sphinxstyleliteralstrong{\sphinxupquote{array}} (\sphinxstyleliteralemphasis{\sphinxupquote{np.ndarray}}) \textendash{} 3D array to average.

\item {} 
\sphinxAtStartPar
\sphinxstyleliteralstrong{\sphinxupquote{axes}} (\sphinxstyleliteralemphasis{\sphinxupquote{tuple}}\sphinxstyleliteralemphasis{\sphinxupquote{, }}\sphinxstyleliteralemphasis{\sphinxupquote{optional}}) \textendash{} Axes for the array. Defaults to None.

\end{itemize}

\sphinxlineitem{Returns}
\sphinxAtStartPar
Radially averaged values.

\sphinxlineitem{Return type}
\sphinxAtStartPar
np.ndarray

\end{description}\end{quote}

\end{fulllineitems}

\index{downsample\_circular\_function\_vectorized() (in module stattools)@\spxentry{downsample\_circular\_function\_vectorized()}\spxextra{in module stattools}}

\begin{fulllineitems}
\phantomsection\label{\detokenize{source/stattools:stattools.downsample_circular_function}}
\pysigstartsignatures
\pysiglinewithargsret
{\sphinxcode{\sphinxupquote{stattools.}}\sphinxbfcode{\sphinxupquote{downsample\_circular\_function\_vectorized}}}
{\sphinxparam{\DUrole{n}{dense\_function}}\sphinxparamcomma \sphinxparam{\DUrole{n}{small\_size}}}
{}
\pysigstopsignatures
\sphinxAtStartPar
Downsample a circularly symmetric function from a large grid to a smaller grid using a vectorized approach.
\begin{quote}\begin{description}
\sphinxlineitem{Parameters}\begin{itemize}
\item {} 
\sphinxAtStartPar
\sphinxstyleliteralstrong{\sphinxupquote{dense\_function}} \textendash{} 2D NumPy array representing the function values on the large grid (e.g., 51 x 51).

\item {} 
\sphinxAtStartPar
\sphinxstyleliteralstrong{\sphinxupquote{small\_size}} \textendash{} Tuple (m, n) representing the size of the small grid (e.g., (5, 5)).

\end{itemize}

\sphinxlineitem{Returns}
\sphinxAtStartPar
2D NumPy array representing the downsampled function on the smaller grid.

\sphinxlineitem{Return type}
\sphinxAtStartPar
small\_grid

\end{description}\end{quote}

\end{fulllineitems}

\index{estimate\_localized\_peaks() (in module stattools)@\spxentry{estimate\_localized\_peaks()}\spxextra{in module stattools}}

\begin{fulllineitems}
\phantomsection\label{\detokenize{source/stattools:stattools.estimate_localized_peaks}}
\pysigstartsignatures
\pysiglinewithargsret
{\sphinxcode{\sphinxupquote{stattools.}}\sphinxbfcode{\sphinxupquote{estimate\_localized\_peaks}}}
{\sphinxparam{\DUrole{n}{array}}\sphinxparamcomma \sphinxparam{\DUrole{n}{axes}}}
{}
\pysigstopsignatures
\sphinxAtStartPar
Estimates localized peaks in a 3D array.
Current implementation is inefficient and will be replaced.
\begin{quote}\begin{description}
\sphinxlineitem{Parameters}\begin{itemize}
\item {} 
\sphinxAtStartPar
\sphinxstyleliteralstrong{\sphinxupquote{array}} (\sphinxstyleliteralemphasis{\sphinxupquote{np.ndarray}}) \textendash{} 3D array to analyze.

\item {} 
\sphinxAtStartPar
\sphinxstyleliteralstrong{\sphinxupquote{axes}} (\sphinxstyleliteralemphasis{\sphinxupquote{tuple}}) \textendash{} Axes for the array.

\end{itemize}

\sphinxlineitem{Returns}
\sphinxAtStartPar
Localized peaks and their amplitudes.

\sphinxlineitem{Return type}
\sphinxAtStartPar
tuple

\end{description}\end{quote}

\end{fulllineitems}

\index{expand\_ring\_averages2d() (in module stattools)@\spxentry{expand\_ring\_averages2d()}\spxextra{in module stattools}}

\begin{fulllineitems}
\phantomsection\label{\detokenize{source/stattools:stattools.expand_ring_averages2d}}
\pysigstartsignatures
\pysiglinewithargsret
{\sphinxcode{\sphinxupquote{stattools.}}\sphinxbfcode{\sphinxupquote{expand\_ring\_averages2d}}}
{\sphinxparam{\DUrole{n}{averaged}\DUrole{p}{:}\DUrole{w}{ }\DUrole{n}{ndarray\DUrole{p}{{[}}int\DUrole{p}{,}\DUrole{w}{ }\DUrole{p}{...}\DUrole{p}{{]}}}}\sphinxparamcomma \sphinxparam{\DUrole{n}{axes}\DUrole{p}{:}\DUrole{w}{ }\DUrole{n}{tuple\DUrole{p}{{[}}ndarray\DUrole{p}{,}\DUrole{w}{ }ndarray\DUrole{p}{{]}}}\DUrole{w}{ }\DUrole{o}{=}\DUrole{w}{ }\DUrole{default_value}{None}}}
{{ $\rightarrow$ ndarray\DUrole{p}{{[}}tuple\DUrole{p}{{[}}int\DUrole{p}{,}\DUrole{w}{ }int\DUrole{p}{{]}}\DUrole{p}{,}\DUrole{w}{ }\DUrole{p}{...}\DUrole{p}{{]}}}}
\pysigstopsignatures
\sphinxAtStartPar
Expands the radially averaged 2D array back to its original shape.
\begin{quote}\begin{description}
\sphinxlineitem{Parameters}\begin{itemize}
\item {} 
\sphinxAtStartPar
\sphinxstyleliteralstrong{\sphinxupquote{averaged}} (\sphinxstyleliteralemphasis{\sphinxupquote{np.ndarray}}) \textendash{} Radially averaged values.

\item {} 
\sphinxAtStartPar
\sphinxstyleliteralstrong{\sphinxupquote{axes}} (\sphinxstyleliteralemphasis{\sphinxupquote{tuple}}\sphinxstyleliteralemphasis{\sphinxupquote{, }}\sphinxstyleliteralemphasis{\sphinxupquote{optional}}) \textendash{} Axes for the array. Defaults to None.

\end{itemize}

\sphinxlineitem{Returns}
\sphinxAtStartPar
Expanded array.

\sphinxlineitem{Return type}
\sphinxAtStartPar
np.ndarray

\end{description}\end{quote}

\end{fulllineitems}

\index{expand\_ring\_averages3d() (in module stattools)@\spxentry{expand\_ring\_averages3d()}\spxextra{in module stattools}}

\begin{fulllineitems}
\phantomsection\label{\detokenize{source/stattools:stattools.expand_ring_averages3d}}
\pysigstartsignatures
\pysiglinewithargsret
{\sphinxcode{\sphinxupquote{stattools.}}\sphinxbfcode{\sphinxupquote{expand\_ring\_averages3d}}}
{\sphinxparam{\DUrole{n}{averaged}\DUrole{p}{:}\DUrole{w}{ }\DUrole{n}{ndarray\DUrole{p}{{[}}tuple\DUrole{p}{{[}}int\DUrole{p}{,}\DUrole{w}{ }int\DUrole{p}{{]}}\DUrole{p}{,}\DUrole{w}{ }\DUrole{p}{...}\DUrole{p}{{]}}}}\sphinxparamcomma \sphinxparam{\DUrole{n}{axes}\DUrole{p}{:}\DUrole{w}{ }\DUrole{n}{tuple\DUrole{p}{{[}}ndarray\DUrole{p}{,}\DUrole{w}{ }ndarray\DUrole{p}{,}\DUrole{w}{ }ndarray\DUrole{p}{{]}}}\DUrole{w}{ }\DUrole{o}{=}\DUrole{w}{ }\DUrole{default_value}{None}}}
{{ $\rightarrow$ ndarray\DUrole{p}{{[}}tuple\DUrole{p}{{[}}int\DUrole{p}{,}\DUrole{w}{ }int\DUrole{p}{,}\DUrole{w}{ }int\DUrole{p}{{]}}\DUrole{p}{,}\DUrole{w}{ }\DUrole{p}{...}\DUrole{p}{{]}}}}
\pysigstopsignatures
\sphinxAtStartPar
Expands the radially averaged 3D array back to its original shape.
\begin{quote}\begin{description}
\sphinxlineitem{Parameters}\begin{itemize}
\item {} 
\sphinxAtStartPar
\sphinxstyleliteralstrong{\sphinxupquote{averaged}} (\sphinxstyleliteralemphasis{\sphinxupquote{np.ndarray}}) \textendash{} Radially averaged values.

\item {} 
\sphinxAtStartPar
\sphinxstyleliteralstrong{\sphinxupquote{axes}} (\sphinxstyleliteralemphasis{\sphinxupquote{tuple}}\sphinxstyleliteralemphasis{\sphinxupquote{, }}\sphinxstyleliteralemphasis{\sphinxupquote{optional}}) \textendash{} Axes for the array. Defaults to None.

\end{itemize}

\sphinxlineitem{Returns}
\sphinxAtStartPar
Expanded array.

\sphinxlineitem{Return type}
\sphinxAtStartPar
np.ndarray

\end{description}\end{quote}

\end{fulllineitems}

\index{find\_decreasing\_radial\_surface\_levels() (in module stattools)@\spxentry{find\_decreasing\_radial\_surface\_levels()}\spxextra{in module stattools}}

\begin{fulllineitems}
\phantomsection\label{\detokenize{source/stattools:stattools.find_decreasing_radial_surface_levels}}
\pysigstartsignatures
\pysiglinewithargsret
{\sphinxcode{\sphinxupquote{stattools.}}\sphinxbfcode{\sphinxupquote{find\_decreasing\_radial\_surface\_levels}}}
{\sphinxparam{\DUrole{n}{array}}\sphinxparamcomma \sphinxparam{\DUrole{n}{axes}\DUrole{o}{=}\DUrole{default_value}{None}}}
{}
\pysigstopsignatures
\sphinxAtStartPar
Not implemented yet

\end{fulllineitems}

\index{find\_decreasing\_surface\_levels2d() (in module stattools)@\spxentry{find\_decreasing\_surface\_levels2d()}\spxextra{in module stattools}}

\begin{fulllineitems}
\phantomsection\label{\detokenize{source/stattools:stattools.find_decreasing_surface_levels2d}}
\pysigstartsignatures
\pysiglinewithargsret
{\sphinxcode{\sphinxupquote{stattools.}}\sphinxbfcode{\sphinxupquote{find\_decreasing\_surface\_levels2d}}}
{\sphinxparam{\DUrole{n}{array}\DUrole{p}{:}\DUrole{w}{ }\DUrole{n}{ndarray\DUrole{p}{{[}}tuple\DUrole{p}{{[}}int\DUrole{p}{,}\DUrole{w}{ }int\DUrole{p}{{]}}\DUrole{p}{,}\DUrole{w}{ }float64\DUrole{p}{{]}}}}\sphinxparamcomma \sphinxparam{\DUrole{n}{axes}\DUrole{o}{=}\DUrole{default_value}{None}}\sphinxparamcomma \sphinxparam{\DUrole{n}{direction}\DUrole{o}{=}\DUrole{default_value}{None}}}
{{ $\rightarrow$ ndarray\DUrole{p}{{[}}tuple\DUrole{p}{{[}}int\DUrole{p}{,}\DUrole{w}{ }int\DUrole{p}{{]}}\DUrole{p}{,}\DUrole{w}{ }int32\DUrole{p}{{]}}}}
\pysigstopsignatures
\sphinxAtStartPar
Assuming function is monotonically decaying around some point, finds surface levels of this function.
No interpolation is used.
\begin{quote}\begin{description}
\sphinxlineitem{Parameters}\begin{itemize}
\item {} 
\sphinxAtStartPar
\sphinxstyleliteralstrong{\sphinxupquote{array}} (\sphinxstyleliteralemphasis{\sphinxupquote{np.ndarray}}) \textendash{} 2D array to analyze.

\item {} 
\sphinxAtStartPar
\sphinxstyleliteralstrong{\sphinxupquote{axes}} (\sphinxstyleliteralemphasis{\sphinxupquote{tuple}}\sphinxstyleliteralemphasis{\sphinxupquote{, }}\sphinxstyleliteralemphasis{\sphinxupquote{optional}}) \textendash{} Axes for the array. Defaults to None.

\item {} 
\sphinxAtStartPar
\sphinxstyleliteralstrong{\sphinxupquote{direction}} (\sphinxstyleliteralemphasis{\sphinxupquote{int}}\sphinxstyleliteralemphasis{\sphinxupquote{, }}\sphinxstyleliteralemphasis{\sphinxupquote{optional}}) \textendash{} Direction to analyze. Defaults to None.

\end{itemize}

\sphinxlineitem{Returns}
\sphinxAtStartPar
Mask indicating the surface levels.

\sphinxlineitem{Return type}
\sphinxAtStartPar
np.ndarray

\end{description}\end{quote}

\end{fulllineitems}

\index{find\_decreasing\_surface\_levels3d() (in module stattools)@\spxentry{find\_decreasing\_surface\_levels3d()}\spxextra{in module stattools}}

\begin{fulllineitems}
\phantomsection\label{\detokenize{source/stattools:stattools.find_decreasing_surface_levels3d}}
\pysigstartsignatures
\pysiglinewithargsret
{\sphinxcode{\sphinxupquote{stattools.}}\sphinxbfcode{\sphinxupquote{find\_decreasing\_surface\_levels3d}}}
{\sphinxparam{\DUrole{n}{array}\DUrole{p}{:}\DUrole{w}{ }\DUrole{n}{ndarray\DUrole{p}{{[}}tuple\DUrole{p}{{[}}int\DUrole{p}{,}\DUrole{w}{ }int\DUrole{p}{,}\DUrole{w}{ }int\DUrole{p}{{]}}\DUrole{p}{,}\DUrole{w}{ }float64\DUrole{p}{{]}}}}\sphinxparamcomma \sphinxparam{\DUrole{n}{axes}\DUrole{o}{=}\DUrole{default_value}{None}}\sphinxparamcomma \sphinxparam{\DUrole{n}{direction}\DUrole{o}{=}\DUrole{default_value}{None}}}
{{ $\rightarrow$ ndarray\DUrole{p}{{[}}tuple\DUrole{p}{{[}}int\DUrole{p}{,}\DUrole{w}{ }int\DUrole{p}{,}\DUrole{w}{ }int\DUrole{p}{{]}}\DUrole{p}{,}\DUrole{w}{ }int32\DUrole{p}{{]}}}}
\pysigstopsignatures
\sphinxAtStartPar
Assuming function is monotonically decaying around some point, finds surface levels of this function.
No interpolation is used.
\begin{quote}\begin{description}
\sphinxlineitem{Parameters}\begin{itemize}
\item {} 
\sphinxAtStartPar
\sphinxstyleliteralstrong{\sphinxupquote{array}} (\sphinxstyleliteralemphasis{\sphinxupquote{np.ndarray}}) \textendash{} 3D array to analyze.

\item {} 
\sphinxAtStartPar
\sphinxstyleliteralstrong{\sphinxupquote{axes}} (\sphinxstyleliteralemphasis{\sphinxupquote{tuple}}\sphinxstyleliteralemphasis{\sphinxupquote{, }}\sphinxstyleliteralemphasis{\sphinxupquote{optional}}) \textendash{} Axes for the array. Defaults to None.

\item {} 
\sphinxAtStartPar
\sphinxstyleliteralstrong{\sphinxupquote{direction}} (\sphinxstyleliteralemphasis{\sphinxupquote{int}}\sphinxstyleliteralemphasis{\sphinxupquote{, }}\sphinxstyleliteralemphasis{\sphinxupquote{optional}}) \textendash{} Direction to analyze. Defaults to None.

\end{itemize}

\sphinxlineitem{Returns}
\sphinxAtStartPar
Mask indicating the surface levels.

\sphinxlineitem{Return type}
\sphinxAtStartPar
np.ndarray

\end{description}\end{quote}

\end{fulllineitems}

\index{gaussian\_maxima\_fitting() (in module stattools)@\spxentry{gaussian\_maxima\_fitting()}\spxextra{in module stattools}}

\begin{fulllineitems}
\phantomsection\label{\detokenize{source/stattools:stattools.gaussian_maxima_fitting}}
\pysigstartsignatures
\pysiglinewithargsret
{\sphinxcode{\sphinxupquote{stattools.}}\sphinxbfcode{\sphinxupquote{gaussian\_maxima\_fitting}}}
{\sphinxparam{\DUrole{n}{array}}\sphinxparamcomma \sphinxparam{\DUrole{n}{axes}}\sphinxparamcomma \sphinxparam{\DUrole{n}{maxima\_indices}}\sphinxparamcomma \sphinxparam{\DUrole{n}{size}\DUrole{o}{=}\DUrole{default_value}{5}}}
{}
\pysigstopsignatures
\sphinxAtStartPar
Fits Gaussian functions to the maxima in a 3D array.
\begin{quote}\begin{description}
\sphinxlineitem{Parameters}\begin{itemize}
\item {} 
\sphinxAtStartPar
\sphinxstyleliteralstrong{\sphinxupquote{array}} (\sphinxstyleliteralemphasis{\sphinxupquote{np.ndarray}}) \textendash{} 3D array to analyze.

\item {} 
\sphinxAtStartPar
\sphinxstyleliteralstrong{\sphinxupquote{axes}} (\sphinxstyleliteralemphasis{\sphinxupquote{tuple}}) \textendash{} Axes for the array.

\item {} 
\sphinxAtStartPar
\sphinxstyleliteralstrong{\sphinxupquote{maxima\_indices}} (\sphinxstyleliteralemphasis{\sphinxupquote{list}}) \textendash{} Indices of the maxima.

\item {} 
\sphinxAtStartPar
\sphinxstyleliteralstrong{\sphinxupquote{size}} (\sphinxstyleliteralemphasis{\sphinxupquote{int}}\sphinxstyleliteralemphasis{\sphinxupquote{, }}\sphinxstyleliteralemphasis{\sphinxupquote{optional}}) \textendash{} Size of the fitting window. Defaults to 5.

\end{itemize}

\sphinxlineitem{Returns}
\sphinxAtStartPar
Fitted maxima and their standard deviations.

\sphinxlineitem{Return type}
\sphinxAtStartPar
tuple

\end{description}\end{quote}

\end{fulllineitems}

\index{reverse\_interpolation\_nearest() (in module stattools)@\spxentry{reverse\_interpolation\_nearest()}\spxextra{in module stattools}}

\begin{fulllineitems}
\phantomsection\label{\detokenize{source/stattools:stattools.reverse_interpolation_nearest}}
\pysigstartsignatures
\pysiglinewithargsret
{\sphinxcode{\sphinxupquote{stattools.}}\sphinxbfcode{\sphinxupquote{reverse\_interpolation\_nearest}}}
{\sphinxparam{\DUrole{n}{x\_axis}}\sphinxparamcomma \sphinxparam{\DUrole{n}{y\_axis}}\sphinxparamcomma \sphinxparam{\DUrole{n}{points}}\sphinxparamcomma \sphinxparam{\DUrole{n}{values}}}
{}
\pysigstopsignatures
\sphinxAtStartPar
Interpolate values from known points to a grid, affecting only the nearest grid cells.
\begin{quote}\begin{description}
\sphinxlineitem{Parameters}\begin{itemize}
\item {} 
\sphinxAtStartPar
\sphinxstyleliteralstrong{\sphinxupquote{x\_axis}} \textendash{} 1D array representing the x\sphinxhyphen{}coordinates of the grid.

\item {} 
\sphinxAtStartPar
\sphinxstyleliteralstrong{\sphinxupquote{y\_axis}} \textendash{} 1D array representing the y\sphinxhyphen{}coordinates of the grid.

\item {} 
\sphinxAtStartPar
\sphinxstyleliteralstrong{\sphinxupquote{points}} \textendash{} Array of known points’ coordinates, shape (N, 2).

\item {} 
\sphinxAtStartPar
\sphinxstyleliteralstrong{\sphinxupquote{values}} \textendash{} Array of known values at the points, shape (N,).

\end{itemize}

\sphinxlineitem{Returns}
\sphinxAtStartPar
2D array of interpolated values on the grid.

\sphinxlineitem{Return type}
\sphinxAtStartPar
interpolated\_grid

\end{description}\end{quote}

\end{fulllineitems}


\sphinxstepscope


\subsection{web\_interface module}
\label{\detokenize{source/web_interface:module-web_interface}}\label{\detokenize{source/web_interface:web-interface-module}}\label{\detokenize{source/web_interface::doc}}\index{module@\spxentry{module}!web\_interface@\spxentry{web\_interface}}\index{web\_interface@\spxentry{web\_interface}!module@\spxentry{module}}
\sphinxAtStartPar
Zeroth iteration on AI generated web interface.
\index{index() (in module web\_interface)@\spxentry{index()}\spxextra{in module web\_interface}}

\begin{fulllineitems}
\phantomsection\label{\detokenize{source/web_interface:web_interface.index}}
\pysigstartsignatures
\pysiglinewithargsret
{\sphinxcode{\sphinxupquote{web\_interface.}}\sphinxbfcode{\sphinxupquote{index}}}
{}
{}
\pysigstopsignatures
\end{fulllineitems}

\index{plot() (in module web\_interface)@\spxentry{plot()}\spxextra{in module web\_interface}}

\begin{fulllineitems}
\phantomsection\label{\detokenize{source/web_interface:web_interface.plot}}
\pysigstartsignatures
\pysiglinewithargsret
{\sphinxcode{\sphinxupquote{web\_interface.}}\sphinxbfcode{\sphinxupquote{plot}}}
{}
{}
\pysigstopsignatures
\end{fulllineitems}


\sphinxstepscope


\subsection{wrappers module}
\label{\detokenize{source/wrappers:module-wrappers}}\label{\detokenize{source/wrappers:wrappers-module}}\label{\detokenize{source/wrappers::doc}}\index{module@\spxentry{module}!wrappers@\spxentry{wrappers}}\index{wrappers@\spxentry{wrappers}!module@\spxentry{module}}
\sphinxAtStartPar
wrappers.py

\sphinxAtStartPar
This module contains wrapper functions for Fourier transforms to make shifts automatically and
make it possible to switch between their implementations.
\begin{description}
\sphinxlineitem{Functions:}
\sphinxAtStartPar
wrapped\_fftn: Wrapper for the FFTN function.
wrapped\_ifftn: Wrapper for the IFFTN function.

\end{description}
\index{wrapped\_fft() (in module wrappers)@\spxentry{wrapped\_fft()}\spxextra{in module wrappers}}

\begin{fulllineitems}
\phantomsection\label{\detokenize{source/wrappers:hpc_utils.wrapped_fft}}
\pysigstartsignatures
\pysiglinewithargsret
{\sphinxcode{\sphinxupquote{wrappers.}}\sphinxbfcode{\sphinxupquote{wrapped\_fft}}}
{\sphinxparam{\DUrole{n}{arrays}}\sphinxparamcomma \sphinxparam{\DUrole{o}{*}\DUrole{n}{args}}\sphinxparamcomma \sphinxparam{\DUrole{o}{**}\DUrole{n}{kwargs}}}
{}
\pysigstopsignatures
\end{fulllineitems}

\index{wrapped\_fftn() (in module wrappers)@\spxentry{wrapped\_fftn()}\spxextra{in module wrappers}}

\begin{fulllineitems}
\phantomsection\label{\detokenize{source/wrappers:hpc_utils.wrapped_fftn}}
\pysigstartsignatures
\pysiglinewithargsret
{\sphinxcode{\sphinxupquote{wrappers.}}\sphinxbfcode{\sphinxupquote{wrapped\_fftn}}}
{\sphinxparam{\DUrole{n}{arrays}}\sphinxparamcomma \sphinxparam{\DUrole{o}{*}\DUrole{n}{args}}\sphinxparamcomma \sphinxparam{\DUrole{o}{**}\DUrole{n}{kwargs}}}
{}
\pysigstopsignatures
\end{fulllineitems}

\index{wrapped\_ifft() (in module wrappers)@\spxentry{wrapped\_ifft()}\spxextra{in module wrappers}}

\begin{fulllineitems}
\phantomsection\label{\detokenize{source/wrappers:hpc_utils.wrapped_ifft}}
\pysigstartsignatures
\pysiglinewithargsret
{\sphinxcode{\sphinxupquote{wrappers.}}\sphinxbfcode{\sphinxupquote{wrapped\_ifft}}}
{\sphinxparam{\DUrole{n}{arrays}}\sphinxparamcomma \sphinxparam{\DUrole{o}{*}\DUrole{n}{args}}\sphinxparamcomma \sphinxparam{\DUrole{o}{**}\DUrole{n}{kwargs}}}
{}
\pysigstopsignatures
\end{fulllineitems}

\index{wrapped\_ifftn() (in module wrappers)@\spxentry{wrapped\_ifftn()}\spxextra{in module wrappers}}

\begin{fulllineitems}
\phantomsection\label{\detokenize{source/wrappers:hpc_utils.wrapped_ifftn}}
\pysigstartsignatures
\pysiglinewithargsret
{\sphinxcode{\sphinxupquote{wrappers.}}\sphinxbfcode{\sphinxupquote{wrapped\_ifftn}}}
{\sphinxparam{\DUrole{n}{arrays}}\sphinxparamcomma \sphinxparam{\DUrole{o}{*}\DUrole{n}{args}}\sphinxparamcomma \sphinxparam{\DUrole{o}{**}\DUrole{n}{kwargs}}}
{}
\pysigstopsignatures
\end{fulllineitems}

\index{wrapper\_ft() (in module wrappers)@\spxentry{wrapper\_ft()}\spxextra{in module wrappers}}

\begin{fulllineitems}
\phantomsection\label{\detokenize{source/wrappers:wrappers.wrapper_ft}}
\pysigstartsignatures
\pysiglinewithargsret
{\sphinxcode{\sphinxupquote{wrappers.}}\sphinxbfcode{\sphinxupquote{wrapper\_ft}}}
{\sphinxparam{\DUrole{n}{ft}}}
{}
\pysigstopsignatures
\sphinxAtStartPar
Wrapper for the Fourier transform functions to make shifts automatically.
Currently based on numpy fft implementation.

\end{fulllineitems}



\chapter{API Reference}
\label{\detokenize{index:api-reference}}

\renewcommand{\indexname}{Python Module Index}
\begin{sphinxtheindex}
\let\bigletter\sphinxstyleindexlettergroup
\bigletter{b}
\item\relax\sphinxstyleindexentry{Box}\sphinxstyleindexpageref{source/Box:\detokenize{module-Box}}
\indexspace
\bigletter{c}
\item\relax\sphinxstyleindexentry{compute\_optimal\_lattices}\sphinxstyleindexpageref{source/compute_optimal_lattices:\detokenize{module-compute_optimal_lattices}}
\item\relax\sphinxstyleindexentry{confocal\_ssnr}\sphinxstyleindexpageref{source/confocal_ssnr:\detokenize{module-confocal_ssnr}}
\indexspace
\bigletter{g}
\item\relax\sphinxstyleindexentry{globvar}\sphinxstyleindexpageref{source/globvar:\detokenize{module-globvar}}
\item\relax\sphinxstyleindexentry{GUI}\sphinxstyleindexpageref{source/GUI:\detokenize{module-GUI}}
\item\relax\sphinxstyleindexentry{GUIInitializationWidgets}\sphinxstyleindexpageref{source/GUIInitializationWidgets:\detokenize{module-GUIInitializationWidgets}}
\item\relax\sphinxstyleindexentry{GUIWidgets}\sphinxstyleindexpageref{source/GUIWidgets:\detokenize{module-GUIWidgets}}
\indexspace
\bigletter{i}
\item\relax\sphinxstyleindexentry{Illumination}\sphinxstyleindexpageref{source/Illumination:\detokenize{module-Illumination}}
\item\relax\sphinxstyleindexentry{input\_parser}\sphinxstyleindexpageref{source/input_parser:\detokenize{module-input_parser}}
\indexspace
\bigletter{k}
\item\relax\sphinxstyleindexentry{kernels}\sphinxstyleindexpageref{source/kernels:\detokenize{module-kernels}}
\indexspace
\bigletter{o}
\item\relax\sphinxstyleindexentry{OpticalSystems}\sphinxstyleindexpageref{source/OpticalSystems:\detokenize{module-OpticalSystems}}
\indexspace
\bigletter{p}
\item\relax\sphinxstyleindexentry{ProcessorSIM}\sphinxstyleindexpageref{source/ProcessorSIM:\detokenize{module-ProcessorSIM}}
\indexspace
\bigletter{s}
\item\relax\sphinxstyleindexentry{ShapesGenerator}\sphinxstyleindexpageref{source/ShapesGenerator:\detokenize{module-ShapesGenerator}}
\item\relax\sphinxstyleindexentry{SIMulator}\sphinxstyleindexpageref{source/SIMulator:\detokenize{module-SIMulator}}
\item\relax\sphinxstyleindexentry{Sources}\sphinxstyleindexpageref{source/Sources:\detokenize{module-Sources}}
\item\relax\sphinxstyleindexentry{SSNRBasedFiltering}\sphinxstyleindexpageref{source/SSNRBasedFiltering:\detokenize{module-SSNRBasedFiltering}}
\item\relax\sphinxstyleindexentry{SSNRCalculator}\sphinxstyleindexpageref{source/SSNRCalculator:\detokenize{module-SSNRCalculator}}
\item\relax\sphinxstyleindexentry{stattools}\sphinxstyleindexpageref{source/stattools:\detokenize{module-stattools}}
\indexspace
\bigletter{v}
\item\relax\sphinxstyleindexentry{VectorOperations}\sphinxstyleindexpageref{source/VectorOperations:\detokenize{module-VectorOperations}}
\indexspace
\bigletter{w}
\item\relax\sphinxstyleindexentry{web\_interface}\sphinxstyleindexpageref{source/web_interface:\detokenize{module-web_interface}}
\item\relax\sphinxstyleindexentry{Windowing}\sphinxstyleindexpageref{source/Windowing:\detokenize{module-Windowing}}
\item\relax\sphinxstyleindexentry{wrappers}\sphinxstyleindexpageref{source/wrappers:\detokenize{module-wrappers}}
\end{sphinxtheindex}

\renewcommand{\indexname}{Index}
\printindex
\end{document}